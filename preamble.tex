% Choose the language
\usepackage{polyglossia}
\setmainlanguage{italian}
\usepackage[english=british]{csquotes}	% English quotes

% Creative commons license
\usepackage[
	type={CC},
	modifier={by-sa},
	version={4.0},
]{doclicense}


% Load the bibliography package
\usepackage[sorting=nyt]{kaobiblio}
\addbibresource{compilatori.bib} % Bibliography file

%\usepackage[framed=true]{kaotheorems}
\usepackage{amsmath} % Improved mathematics
\usepackage{amsthm} % Mathematical environments
\usepackage{thmtools} % Theorem styles
\usepackage[framemethod=TikZ]{mdframed}
% From kaobook
\mdfsetup{skipabove=\topskip,skipbelow=-.5\topskip}
\mdfdefinestyle{mdfkao}{
  skipabove=\topskip,
  skipbelow=\topskip, % Does not work :(
  rightmargin=0pt,
  leftmargin=0pt,
  innertopmargin=7pt,
  innerbottommargin=7pt,
  innerrightmargin=7pt,
  innerleftmargin=7pt,
  topline=false,
  bottomline=true,
  rightline=false,
  leftline=false,
  frametitlerule=true,
}




\declaretheoremstyle[
spaceabove=6pt,
spacebelow=6pt,
headfont=\normalfont\bfseries,
notefont=\mdseries, notebraces={(}{)},
bodyfont=\normalfont,
postheadspace=1em,
shaded={bgcolor=Goldenrod!45!white},
headpunct={}
]{mythmstyle}

\theoremstyle{mythmstyle}
\declaretheorem[
name=Teorema,
style=mythmstyle,
refname={Teorema,Teoremi},
Refname={Teorema,Teoremi},
numberwithin=chapter,
]{theorem}
\declaretheorem[
name=Proposizione,
style=mythmstyle,
sibling=theorem,
refname={Proposizione,Proposizioni},
Refname={Proposizione,Proposizioni},
numberwithin=chapter,
]{proposition}
\declaretheorem[
name=Corollario,
style=mythmstyle,
sibling=theorem,
refname={Corollario,Corollari},
Refname={Corollario,Corollari},
numberwithin=chapter,
]{corollary}
\declaretheorem[
name=Lemma,
style=mythmstyle,
sibling=theorem,
refname={Lemma,Lemmi},
Refname={Lemma,Lemmi},
numberwithin=chapter,
]{lemma}
\declaretheorem[
name=Nota,
style=mythmstyle,
sibling=theorem,
refname={Nota,Note},
Refname={Nota,Note},
numberwithin=chapter,
]{observation}


% Quello che segue non ha la stessa numerazione di teoremi, lemmi, ecc.
\declaretheoremstyle[
spaceabove=6pt,
spacebelow=12pt,
headfont=\normalfont\bfseries,
notefont=\mdseries, notebraces={(}{)},
bodyfont=\normalfont,
postheadspace=1em,
shaded={bgcolor=green!15!white},
headpunct={}
]{mydefstyle}
\declaretheorem[
name=Definizione,
style=mydefstyle,
refname={Definizione,Definizioni},
Refname={Definizione,Definizioni},
numberwithin=chapter,
]{definition}

\declaretheorem[
name=Problema,
style=mydefstyle,
refname={Problema,Problemi},
Refname={Problema,Problemi},
]{problem}



\declaretheoremstyle[
spaceabove=6pt,
spacebelow=6pt,
headfont=\normalfont\bfseries,
notefont=\mdseries, notebraces={(}{)},
bodyfont=\normalfont,
postheadspace=1em,
shaded={bgcolor=gray!15!white},
headpunct={}
]{myexamplestyle}
\declaretheorem[
name=Esempio,
style=myexamplestyle,
refname={Esempio,Esempi},
Refname={Esempio,Esempi},
numberwithin=chapter,
]{example}

% \declaretheorem[
% name=Esercizio,
% style=mythmstyle,
% refname={Esercizio,Esercizi},
% Refname={Esercizio,Esercizi},
% numberwithin=chapter,
% ]{exercise}



\newcommand{\parte}[1]{\pagelayout{wide}\addpart{#1}\pagelayout{margin}}


\usepackage{kaorefs}

% Set the paths where to look for images
\usepackage{subcaption}
\graphicspath{{./imgs/}{./img/}}

\makeindex[columns=3, title=Alphabetical Index, intoc] % Make LaTeX produce the files required to compile the index

\makeglossaries% Make LaTeX produce the files required to compile the glossary
\makenomenclature

% Tikz
\usetikzlibrary{automata,positioning}

\newcommand{\keyword}[1]{\textbf{#1}\marginnote{#1}}

% Produzioni
\usepackage{stackrel}
\newcommand{\sprod}[1]{\ensuremath{\stackrel[#1]{}{\Longrightarrow}}}
\newcommand{\eprod}[1]{\ensuremath{\xRightarrow[#1]{*}}}
\newcommand{\prodgen}{\ensuremath{\alpha\to\beta}\xspace}


% Grammatiche
\newcommand{\gramm}{\ensuremath{G=\langle V, T, P, S \rangle}\xspace}

% Automi
\newcommand{\dfa}{\ensuremath{\mathcal{A}=\langle \Sigma, Q, q_{0}, F, \delta \rangle}\xspace}


\usepackage{upref}
\usepackage{siunitx}
\usepackage{upquote}

\usepackage{standalone}



%\usepackage[lastexercise]{exercise}
\usepackage[lastexercise,answerdelayed]{exercise}
\counterwithin{Exercise}{chapter}
\counterwithin{Answer}{chapter}
\renewcounter{Exercise}[chapter]
\renewcommand{\ExerciseHeader}{\noindent\textbf{Esercizio~\ExerciseHeaderNB}.}
\renewcommand{\AnswerHeader}{\noindent\textbf{Soluzione all'Esercizio~\ExerciseHeaderNB}.}

%%% Local Variables:
%%% mode: LaTeX
%%% TeX-engine: luatex
%%% TeX-master: "libro-linguaggi"
%%% End:
