\documentclass[
a4paper,
fontsize=10pt, % Base font size
twoside=false, % Use different layouts for even and odd pages (in particular, if twoside=true, the margin column will be always on the outside)
% open=any, % If twoside=true, uncomment this to force new chapters to start on any page, not only on right (odd) pages
% chapterprefix=true, % Uncomment to use the word "Chapter" before chapter numbers everywhere they appear
% chapterentrydots=true, % Uncomment to output dots from the chapter name to the page number in the table of contents
numbers=auto, % Comment to output dots after chapter numbers; the most common values for this option are: enddot, noenddot and auto (see the KOMAScript documentation for an in-depth explanation)
% draft=true, % If uncommented, rulers will be added in the header and footer
fontmethod = modern,
% overfullrule=true, % If uncommented, overly long lines will be marked by a black box; useful for correcting spacing problems
]{kaobook}

% Choose the language
\usepackage{polyglossia}
\setmainlanguage{italian}
\usepackage[english=british]{csquotes}	% English quotes

% Creative commons license
\usepackage[
	type={CC},
	modifier={by-sa},
	version={4.0},
]{doclicense}


% Load the bibliography package
\usepackage[sorting=nyt]{kaobiblio}
\addbibresource{compilatori.bib} % Bibliography file

%\usepackage[framed=true]{kaotheorems}
\usepackage{amsmath} % Improved mathematics
\usepackage{amsthm} % Mathematical environments
\usepackage{thmtools} % Theorem styles
\usepackage[framemethod=TikZ]{mdframed}
% From kaobook
\mdfsetup{skipabove=\topskip,skipbelow=-.5\topskip}
\mdfdefinestyle{mdfkao}{
  skipabove=\topskip,
  skipbelow=\topskip, % Does not work :(
  rightmargin=0pt,
  leftmargin=0pt,
  innertopmargin=7pt,
  innerbottommargin=7pt,
  innerrightmargin=7pt,
  innerleftmargin=7pt,
  topline=false,
  bottomline=true,
  rightline=false,
  leftline=false,
  frametitlerule=true,
}




\declaretheoremstyle[
spaceabove=6pt,
spacebelow=6pt,
headfont=\normalfont\bfseries,
notefont=\mdseries, notebraces={(}{)},
bodyfont=\normalfont,
postheadspace=1em,
shaded={bgcolor=Goldenrod!45!white},
headpunct={}
]{mythmstyle}

\theoremstyle{mythmstyle}
\declaretheorem[
name=Teorema,
style=mythmstyle,
refname={Teorema,Teoremi},
Refname={Teorema,Teoremi},
numberwithin=chapter,
]{theorem}
\declaretheorem[
name=Proposizione,
style=mythmstyle,
sibling=theorem,
refname={Proposizione,Proposizioni},
Refname={Proposizione,Proposizioni},
numberwithin=chapter,
]{proposition}
\declaretheorem[
name=Corollario,
style=mythmstyle,
sibling=theorem,
refname={Corollario,Corollari},
Refname={Corollario,Corollari},
numberwithin=chapter,
]{corollary}
\declaretheorem[
name=Lemma,
style=mythmstyle,
sibling=theorem,
refname={Lemma,Lemmi},
Refname={Lemma,Lemmi},
numberwithin=chapter,
]{lemma}
\declaretheorem[
name=Nota,
style=mythmstyle,
sibling=theorem,
refname={Nota,Note},
Refname={Nota,Note},
numberwithin=chapter,
]{observation}


% Quello che segue non ha la stessa numerazione di teoremi, lemmi, ecc.
\declaretheoremstyle[
spaceabove=6pt,
spacebelow=12pt,
headfont=\normalfont\bfseries,
notefont=\mdseries, notebraces={(}{)},
bodyfont=\normalfont,
postheadspace=1em,
shaded={bgcolor=green!15!white},
headpunct={}
]{mydefstyle}
\declaretheorem[
name=Definizione,
style=mydefstyle,
refname={Definizione,Definizioni},
Refname={Definizione,Definizioni},
numberwithin=chapter,
]{definition}

\declaretheorem[
name=Problema,
style=mydefstyle,
refname={Problema,Problemi},
Refname={Problema,Problemi},
]{problem}



\declaretheoremstyle[
spaceabove=6pt,
spacebelow=6pt,
headfont=\normalfont\bfseries,
notefont=\mdseries, notebraces={(}{)},
bodyfont=\normalfont,
postheadspace=1em,
shaded={bgcolor=gray!15!white},
headpunct={}
]{myexamplestyle}
\declaretheorem[
name=Esempio,
style=myexamplestyle,
refname={Esempio,Esempi},
Refname={Esempio,Esempi},
numberwithin=chapter,
]{example}

% \declaretheorem[
% name=Esercizio,
% style=mythmstyle,
% refname={Esercizio,Esercizi},
% Refname={Esercizio,Esercizi},
% numberwithin=chapter,
% ]{exercise}



\newcommand{\parte}[1]{\pagelayout{wide}\addpart{#1}\pagelayout{margin}}


\usepackage{kaorefs}

% Set the paths where to look for images
\usepackage{subcaption}
\graphicspath{{./imgs/}{./img/}}

\makeindex[columns=3, title=Alphabetical Index, intoc] % Make LaTeX produce the files required to compile the index

\makeglossaries% Make LaTeX produce the files required to compile the glossary
\makenomenclature

% Tikz
\usetikzlibrary{automata,positioning}

\newcommand{\keyword}[1]{\textbf{#1}\marginnote{#1}}

% Produzioni
\usepackage{stackrel}
\newcommand{\sprod}[1]{\ensuremath{\stackrel[#1]{}{\Longrightarrow}}}
\newcommand{\eprod}[1]{\ensuremath{\xRightarrow[#1]{*}}}
\newcommand{\prodgen}{\ensuremath{\alpha\to\beta}\xspace}


% Grammatiche
\newcommand{\gramm}{\ensuremath{G=\langle V, T, P, S \rangle}\xspace}

% Automi
\newcommand{\dfa}{\ensuremath{\mathcal{A}=\langle \Sigma, Q, q_{0}, F, \delta \rangle}\xspace}


\usepackage{upref}
\usepackage{siunitx}
\usepackage{upquote}

\usepackage{standalone}



%\usepackage[lastexercise]{exercise}
\usepackage[lastexercise,answerdelayed]{exercise}
\counterwithin{Exercise}{chapter}
\counterwithin{Answer}{chapter}
\renewcounter{Exercise}[chapter]
\renewcommand{\ExerciseHeader}{\noindent\textbf{Esercizio~\ExerciseHeaderNB}.}
\renewcommand{\AnswerHeader}{\noindent\textbf{Soluzione all'Esercizio~\ExerciseHeaderNB}.}

%%% Local Variables:
%%% mode: LaTeX
%%% TeX-engine: luatex
%%% TeX-master: "libro-linguaggi"
%%% End:


\begin{document}

\definecolor{shadecolor}{gray}{0.80}

\titlehead{Linguaggi, Grammatiche, Parser}
\subject{}

\title[Linguaggi, Grammatiche, Parser]{Linguaggi, Grammatiche, Parser}
\subtitle{Un libro su: Linguaggi, Grammatiche, Parser}

\author[GDV]{Gianluca Della Vedova}
\date{\today}

\frontmatter

%----------------------------------------------------------------------------------------
%       COPYRIGHT PAGE
%----------------------------------------------------------------------------------------

\makeatletter
\uppertitleback{\@titlehead} % Header

\lowertitleback{\doclicenseThis%
Sei libero di riprodurre, distribuire, comunicare al pubblico, esporre
in pubblico, rappresentare, eseguire, recitare e modificare quest'opera
alle seguenti condizioni:
\begin{itemize}
\item
Attribuzione — Devi attribuire la paternit{\`a} dell'opera nei modi
indicati dall'autore o da chi ti ha dato l'opera in licenza e in modo tale da
non suggerire che essi avallino te o il modo in cui tu usi l'opera.
\item
Condividi allo stesso modo — Se alteri o trasformi quest'opera, o se
la usi per crearne un'altra, puoi distribuire l'opera risultante solo con
una licenza identica o equivalente a  questa.
\end{itemize}
%  \vspace*{1cm}

Tutto il materiale si trova al link
\url{https://github.com/gdv/linguaggi-formali}.
Tutti sono invitati a contribuire a questo libro.
}
\makeatother
\maketitle


%----------------------------------------------------------------------------------------
%       PREFACE
%----------------------------------------------------------------------------------------

%\input{chapters/preface.tex}
%\index{preface}

%----------------------------------------------------------------------------------------
%       TABLE OF CONTENTS & LIST OF FIGURES/TABLES
%----------------------------------------------------------------------------------------

\begingroup % Local scope for the following commands

% Define the style for the TOC, LOF, and LOT
%\setstretch{1} % Uncomment to modify line spacing in the ToC
%\hypersetup{linkcolor=blue} % Uncomment to set the colour of links in the ToC
\setlength{\textheight}{230\hscale} % Manually adjust the height of the ToC pages

% Turn on compatibility mode for the etoc package
\etocstandarddisplaystyle % "toc display" as if etoc was not loaded
\etocstandardlines % "toc lines as if etoc was not loaded

\tableofcontents % Output the table of contents

\endgroup

%----------------------------------------------------------------------------------------
%       MAIN BODY
%----------------------------------------------------------------------------------------

\mainmatter % Denotes the start of the main document content, resets page numbering and uses arabic numbers
\setchapterstyle{kao} % Choose the default chapter heading style

\parte{Introduzione}
\chapter{Introduzione}

Questo testo si prefigge di introdurre i linguaggi formali e descrivere la
rilevanza che questi hanno nel disegno di un compilatore.

In sintesi, un compilatore è un programma che riceve il codice sorgente di un
programma e produce un programma equivalente, detto codice oggetto, in un altro
linguaggio di programmazione: spesso in assembler.

L'anatomia di un compilatore individua tre componenti
principali~\sidecite{engineering-a-compiler}:
\begin{enumerate}
\item
      Frontend: trasforma il codice sorgente in una rappresentazione intermedia
      (IR) indipendente dal linguaggio di programmazione.
\item
      Optimizer: manipola la rappresentazione intermedia per migliorarne le
      caratteristiche (rendere più veloce il programma, usare meno memoria,
      ecc.)
\item
      Backend: riceve la rappresentazione intermedia ed emette il corrispondente
      codice oggetto.
\end{enumerate}

Vedremo che i linguaggi formali sono il fondamento di ogni frontend.

%%% Local Variables:
%%% mode: LaTeX
%%% TeX-master: "../libro-linguaggi"
%%% End:



\parte{Linguaggi e grammatiche}
\setchapterpreamble[u]{\margintoc}
\chapter{Linguaggi e Grammatiche}\label{cha:linguaggi+grammatiche}
\labch{Linguaggi e Grammatiche}

\section{Linguaggi}
iIniziamo con le definizioni di base che saranno fondamentali in tutto il testo.

Un \textbf{simbolo} o \textbf{carattere} è un qualsiasi oggetto.


\begin{definition}[Alfabeto]\label{def:alfabeto}
Un \keyword{alfabeto}, normalmente indicato con $\Sigma$, è un insieme finito e non vuoto di simboli.
\end{definition}

Esempi di alfabeti sono \texttt{abcdefghijklmnopqrstuvwxyz}, e l'insieme di cifre \texttt{0123456789}.
Siccome un alfabeto è un insieme, l'ordine dei simboli non è rilevante.
In altre parole \texttt{9876543210} è sempre l'alfabeto delle cifre.
Nel seguito useremo $\Sigma_{L}$ per indicare l'alfabeto delle lettere minuscole, $\Sigma_{C}$ per l'alfabeto delle cifre,
$\Sigma_{B}$ per l'alfabeto delle cifre binarie.


La giustapposizione (o concatenazione) di simboli permette di creare parole o stringhe, quali ad
esempio \texttt{parola} o \texttt{2301}.
La concatenazione di due simboli viene rappresentata semplicemente scrivendo un simbolo dopo l'altro.


\begin{definition}[Parola]\label{def:parola}
Sia $\Sigma$ un alfabeto.
Allora una \keyword{parola} è una sequenza $\sigma_{1}\cdots\sigma_{n}$ di simboli, non necessariamente
distinti, dell'alfabeto $\Sigma$.
Una parola viene normalmente rappresentata tramite la giustapposizione dei
simboli che compongono la sequenza, rispettando l'ordine.
\end{definition}

Normalmente usiamo le ultime lettere dell'alfabeto latino (ad esempio $w$, $x$,
$y$, $z$) per rappresentare parole.
La \keyword{lunghezza} di una parola è il numero di simboli nella sequenza.
La lunghezza della parola $z$ viene indicata con $|z|$.

Alcuni simboli sono speciali perchè hanno un significato particolare e non fanno parte di nessun alfabeto $\Sigma$.
Il primo simbolo speciale che vediamo è $\epsilon$ e rappresenta la stringa vuota,
formata dalla sequenza di zero simboli.

\begin{definition}[Linguaggio]\label{def:linguaggio}
Dato un alfabeto $\Sigma$, un \keyword{linguaggio} $L$ è un insieme di parole su $\Sigma$.
\end{definition}

Sebbene l'alfabeto sia finito, un linguaggio potrebbe contenere un numero
infinito di parole.
Diventa quindi importante capire quando un linguaggio infinito ha una
rappresentazione finita.
Notare che praticamente tutti i linguaggi rilevanti sono infiniti, ma con
rappresentazione finita.
Prendiamo ad esempio JSON: abbiamo una specifica formale (quindi una sequenza di
caratteri corrisponde ad un documento JSON se e solo se soddisfa la specifica),
e l'insieme dei documenti JSON è infinito.

\begin{example}\label{exa:numeri}
Un numero è formato da una parte intera che consiste unicamente di cifre e,
opzionalmente, da una parte frazionaria che consiste unicamente di cifre.
Se la parte frazionaria esiste, allora la parte intera e la parte frazionaria
sono separate da un punto.
La parte intera non può iniziare con \texttt{0} e la parte frazionaria non può
finire con \texttt{0}.
L'insieme di tutti i numeri è un linguaggio (infinito).
\end{example}

La descrizione nell'\cref{exa:numeri} è di natura insiemistica:  il
linguaggio è visto come l'insieme delle parole che lo compongono.
Quando vogliamo studiare un linguaggio, esistono tre punti di vista alternativi: (1) considerare come sia possibile generare
tutte le parole di un linguaggio, (2) descrivere una procedura che determini se
una parola appartiene al linguaggio, (3) fornire una proprietà che è soddisfatta da tutte e sole le parole del linguaggio.
Si noti che la definizione di linguaggio non cambia.
Nel seguito vedremo come questi tre punti di vista si complementino e
interagiscono fra loro.

L'approccio, basato sulla procedura per determinare se una parola
appartiene al linguaggio, ci porta a definire il primo fondamentale problema
computazionale, detto problema dell'\keyword{appartenenza} ad un linguaggio.


\begin{problem}[Appartenenza]\label{pb:appartenenza}
Siano $L$ un linguaggio e $z$ una parola, entrambi sull'alfabeto $\Sigma$.
Determinare se $z$ \keyword{appartiene} a $L$ o, in altre parole, se $z$ sia una parola del linguaggio $L$.
\end{problem}

Il problema dell'appartenenza è un \keyword{problema di decisione}, perchè
ammette solo due risposte: sì o no.
La nozione di  concatenazione non si applica solo ai caratteri, ma anche a stringhe e linguaggi.


\begin{definition}[Concatenazione]\label{def:concatenazione-parole}
Siano $w$ e $x$ due parole.
La loro \keyword{concatenazione} $wx$ è ottenuta prendendo $w$ e facendo seguire a questa la stringa  $x$.
Più formalmente, se $w=a_{1}\cdots a_{|w|}$ e $x=b_{1}\cdots b_{|w|}$, allora la loro concatenazione $wx$ è la stringa
$a_{1}\cdots a_{|w|} b_{1}\cdots b_{|w|}$.
\end{definition}

\begin{definition}[Concatenazione di linguaggi]\label{def:concatenazione-linguaggi}
Siano $L_{1}$ e $L_{2}$ due linguaggi, rispettivamente su alfabeto $\Sigma_{1}$ e $\Sigma_{2}$.
Allora la loro concatenazione $L_{1}L_{2}$ è uguale alla concatenazione di tutte le coppie ordinate di parole di $L_{1}$
e $L_2$.
Formalmente, $L_{1}L_{2} = \{ w_{1}w_2 : w_{1}\in L_{1}, w_{2}\in L_2\}$.
\end{definition}

Una definizione equivalente di concatenazione di linguaggi è $L_{1}L_{2} = \{ w_{1}w_2 : (w_{1}, w_{2})\in L_{1} \times L_2\}$.
Si noti anche che l'alfabeto del linguaggio $L_1 L_2$ è $\Sigma_{1} \cup \Sigma_{2}$.


\begin{definition}\label{def:potenza-linguaggio}
Sia $L$ un linguaggio e sia $k$ un numero intero strettamente positivo.
Allora la $k$-esima \keyword{potenza} di $L$, denotata con $L^k$, è l'insieme di tutte le stringhe ottenute concatenando
$k$ parole, non necessariamente distinte, del linguaggio $L$.
\end{definition}

Per convenzione, si indica con $L^{0} = \{ epsilon \}$.
Quindi possiamo vedere $L^{1} =L$, mentre $L^{k} = L^{k-1} \Sigma$ se $k>1$.
Una variante della \cref{def:concatenazione-linguaggi} consiste nella definizione di potenza di un alfabeto.

\begin{definition}\label{def:potenza-alfabeto}
Sia $\Sigma$ un alfabeto e sia $k$ un numero intero strettamente positivo.
Allora la $k$-esima \keyword{potenza} di $\Sigma$, denotata con $\Sigma^k$, è l'insieme di tutte le stringhe di
lunghezza $k$ sull'alfabeto $\Sigma$.
\end{definition}

Per convenzione, si indica con $\Sigma^{0} = \{ \epsilon \}$, esattamente come nel caso di potenza di un linguaggio.
Vediamo $\Sigma^{1}$ come il linguaggio formato da tutte le parole di lunghezza $1$ prese dall'alfabeto
$\Sigma$, mentre $\Sigma^{k} = \Sigma^{k-1} \Sigma$ se $k>1$.
Non possiamo dire $\Sigma^{1} = \Sigma$ perchè il primo è un insieme di parole e il secondo è un
insieme di simboli.

\begin{example}\label{exa:L3}
L'alfabeto $\Sigma_{B}^{3}$ è formato dalle stringhe \texttt{000}, \texttt{001}, \texttt{010}, \texttt{011}, \texttt{100}, \texttt{101}, \texttt{110}, \texttt{111}.
\end{example}


La nozione di chiusura permette di generalizzare la potenza al caso $k=\infty$.




\begin{definition}\label{def:punto-fisso}
Sia $A$ un insieme non vuoto e sia $\phi$ una funzione con mappa insiemi in insiemi.
Allora $A$ è un \keyword{punto fisso} per $\phi$ se $\phi(A) \subseteq A$.
\end{definition}

In altre parole, $A$ è un punto fisso se l'applicazione di $\phi$ non ``cambia'' $A$.
Ovviamente una funzione $\phi: X\mapsto Y$ potrebbe non avere punti fissi e potrebbe anche avere più di un punto fisso.
Noi siamo interessati ai \emph{minimi punti fissi} che contengono un insieme $A$: i più piccoli insiemi $B\supseteq A$
tali che $B$ sia un punto fisso per $\phi$.

\begin{definition}\label{def:kleene}
Sia $\Sigma$ un alfabeto.
Allora la sua \keyword{chiusura di Kleene}, denotata con $\Sigma^{*}$, è l'unione infinita di potenze
$\bigcup_{k=0}^{\infty} \Sigma^k = \Sigma^0 \cup \Sigma^1 \cup \cdots$.
\end{definition}

\begin{proposition}\label{prop:kleene-punto-fisso}
Sia $\Sigma$ un alfabeto.
Allora $\Sigma^{*}$ è il minimo punto fisso di $\Sigma \cup \{\epsilon\}$ rispetto alla funzione
chiusura di Kleene come minimo punto fisso di $L\cup \{\epsilon\}$ rispetto alla funzione $\phi$ che mappa ogni
linguaggio $L$ in $\phi(L) = L^{2}$.
\end{proposition}

\begin{example}\label{exa:chiusura-kleene}
$\Sigma_{B}^{*} = \{\mathtt{\epsilon},\mathtt{0},\mathtt{1},\mathtt{00},\mathtt{01},\mathtt{10},\mathtt{100},\mathtt{000},\ldots \}$.
\end{example}

Adesso possiamo introdurre la notazione più utilizzata per specificare un linguaggio costruito su un determinato alfabeto.


\begin{observation}\label{obs:linguaggio}
Un linguaggio $L$ su alfabeto $\Sigma$ è un sottoinsieme di stringhe in $ \Sigma^*$,
quindi $L\subseteq \Sigma^*$.
\end{observation}

Siccome ogni linguaggio è un insieme, il simbolo $\emptyset$ denota anche il \keyword{linguaggio vuoto} (notare che
$\emptyset\in\Sigma^k,\,|\emptyset|=0$), che è
diverso dal linguaggio $\{\epsilon\}$ che consiste esattamente della stringa vuota.
Infatti $|\{\varepsilon\}|=1$.
Infine si noti che $\Sigma^*$ è sempre un insieme infinito.
Vediamo adesso alcuni esempi di linguaggi sull'alfabeto $\Sigma_{B}$.


\begin{example}\label{exa:0n01}
Le stringhe che consistono in $n$ \texttt{0} seguiti da $n$ \texttt{1}:
$\{\varepsilon,\mathtt{01},\mathtt{0011},\mathtt{000111},\ldots\}$.
\end{example}

\begin{example}\label{exa:0=1}
le stringhe con un uguale numero di 0 e di 1:
$\{\varepsilon,\mathtt{01},\mathtt{10},\mathtt{0011},\mathtt{0110},\mathtt{0101},\mathtt{1001},\ldots\}$.
\end{example}

\begin{example}\label{exa:numeri-primi-binari}
le stringhe che codificano un numero primo:
$\{\mathtt{10},\mathtt{11},\mathtt{101}, \mathtt{111}, \ldots\}$.
\end{example}

\begin{example}\label{exa:numeri-primi-unari}
le stringhe che sono una codifica unaria di un numero primo, senza usare \texttt{0}:
$\{\mathtt{11},\mathtt{111},\mathtt{11111}, \mathtt{1111111}, \ldots\}$.
\end{example}

L'\cref{exa:numeri-primi-unari} è interessante perchè l'alfabeto effettivamente usato ha un solo simbolo.
In altre parole, non sono necessari due simboli per le nozioni di alfabeto e linguaggio.


\begin{observation}\label{obs:linguaggi-banali}
Sia $\Sigma$ un alfabeto.
Allora $\emptyset$, $\{\varepsilon\}$, e $\Sigma^{k}$ per ogni intero $k\ge 1$ sono tutti linguaggi per l'alfabeto $\Sigma$.
\end{observation}

Notiamo che tutti gli esempi di linguaggi che abbiamo dato in questa sezione sono basati sulla descrizione di una
proprietà che è vera per tutte e sole le parole del linguaggio.
Questo vuole dire che stiamo considerando una visione \emph{insiemistica} dei linguaggi.
Vedremo presto che questa non è l'unica visione possibile.


\section{Grammatiche}\label{sec:grammatiche}

Mentre la \ref{def:linguaggio} vede un linguaggio come un insieme, in questa sezione andiamo ad introdurre il concetto
di grammatica: ciò ci porterà a vedere come sia possibile \emph{generare} tutte e sole le parole appartenenti ad un linguaggio.


\begin{definition}[Grammatica]\label{def:grammatica}
Una \keyword{grammatica} $G$ è una quadrupla $G=\langle V, T, P, \mathsf{S} \rangle$ dove:
\begin{itemize}
	\item $V$ è un insieme finito di \keyword{variabili} o non terminali;
	\item $T$ è un insieme finito di simboli \keyword{terminali}, ovvero i simboli dell'alfabeto di riferimento,
	  disgiunto dalle variabili.
Quindi $V \cap T=\emptyset$;
	\item $P$ è un insieme finito di \textbf{produzioni} o regole;
	\item $\mathsf{S}\in V$ è il \keyword{simbolo iniziale} ($\mathsf{S}$ rappresenta ``start'').
\end{itemize}
\end{definition}


Dobbiamo ancora fornire la definizione di produzione.

\begin{definition}\label{def:produzione}
Una \keyword{produzione} $\alpha \to \beta$ formata da tre parti:
\begin{itemize}
\item una stringa non vuota $\alpha$, detta \keyword{testa},  sull'alfabeto $V \cup T$,
	  \item il simbolo $\to$ della produzione;
\item una stringa $\beta$, detta \keyword{corpo},  sull'alfabeto $V \cup T$.
\end{itemize}
\end{definition}

Notiamo che sia la testa che il corpo di una produzione possono contenere sia terminali che variabili, senza alcuna
restrizione: possono quindi esserci solo terminali, solo variabili, oppure entrambi.
Anche l'ordine con cui appaiono non ha vincoli.
L'unica restrizione è che la testa non può essere la stringa vuota, mentre il corpo potrebbe essere la stringa vuota
($\beta = \epsilon$).
Convenzionalmente, per evitare ambiguità, il simbolo di produzione $\to$ non appartiene a $V\cup T$.


Ad ogni grammatica $G=\langle V, T, P, S \rangle$ possiamo associare il linguaggio generato da $G$.
Intuitivamente, si parte dal simbolo iniziale $S$ e si applica ripetutamente una produzione.
L'applicazione di una produzione $\alpha\to\beta$ consiste nell'individuare $\alpha$ nella sottostringa attuale e
sostituirla con $\beta$.
Dopo ogni applicazione si ottiene una nuova stringa attuale e il processo può ripetersi, anche applicando una produzione
diversa da quella utilizzata nel passo precedente.
Quando la stringa attuale è formata da soli simboli terminale, tale stringa è una delle parole del linguaggio.
Il processo di ripetute applicazioni di produzioni si chiama derivazione.

Per semplificare alcune parti, introduciamo la definizione di forma sentenziale.

\begin{definition}[Forma sentenziale]\label{def:forma-sentenziale}
Sia \gramm una grammatica.
Una \keyword{forma sentenziale} è una stringa sull'alfabeto $V\cup T$.
\end{definition}


\begin{example}\label{exa:0001}
Consideriamo il linguaggio $L$ su alfabeto $\Sigma_{B}$ delle parola che contengono esattamente un \texttt{1} e questo
carattere is deve trovare in ultima posizione.
Una grammatica $G=\langle V, T, P, S \rangle$ che genera il linguaggio $L$ ha $V=\{\mathsf{Z}, \mathsf{U}\}$,
$T=\{\mathtt{0}, \mathtt{1}\}$, e le produzioni $\mathsf{S}\to \mathsf{Z}\mathsf{U}$, $\mathsf{S}\to \mathsf{U}$,
$\mathsf{U}\to \mathtt{1}$, $\mathsf{Z}\to \mathsf{Z}\mathtt{0}$, and $\mathsf{Z}\to \mathtt{0}$.
\end{example}

Possiamo notare che la stringa \texttt{001} viene generata dalla grammatica $G$ tramite la seguente sequenza di
applicazioni di produzione:

\[\mathsf{S} \sprod{\mathsf{S}\to \mathsf{Z}\mathsf{U}} \mathsf{Z}\mathsf{U}  \sprod{\mathsf{Z}\to \mathsf{Z}\mathtt{0}} \mathsf{Z}\mathtt{0}\mathsf{U} \sprod{\mathsf{U}\to \mathtt{1}} \mathsf{Z}\mathtt{0}\mathtt{1} \sprod{\mathsf{Z}\to \mathtt{0}} \mathtt{0}\mathtt{0}\mathtt{1} \]

Questa derivazione non è l'unica possibile per generare la stringa \texttt{001} e la grammatica $G$ che abbiamo
descritto non è l'unica in grado di generare il linguaggio $G$.
Possiamo adesso dare la definizione formale di applicazione di una produzione.

\begin{definition}Applicazione di una produzione\label{def:applicazione-produzione}
Siano $\alpha$, $\beta$, $\gamma$, $\delta$ stringhe su alfabeto $V\cup T$.
Sia $\gramm$ una grammatica e sia \prodgen una sua produzione (quindi $\alpha\neq\epsilon$).
L'\keyword{applicazione} della produzione \prodgen alla stringa $\gamma\alpha\delta$ è denotata con
$\gamma\alpha\delta \sprod{\prodgen} \gamma\beta\delta$ e rappresenta il fatto che $\alpha$ viene sostituita con $\beta$.
Se la produzione \prodgen è facilmente identificabile, oppure se non ci interessa specificare la produzione usata, possiamo denotare l'applicazione indicando solo la grammatica $G$
con $\gamma\alpha\delta \sprod{G} \gamma\beta\delta$.
Se anche la grammatica coinvolta non è ambigua, possiamo ometterla usando la scrittura $\gamma\alpha\delta \sprod{} \gamma\beta\delta$.
\end{definition}

Per denotare in modo compatto l'effetto di una sequenza di applicazioni di produzioni, e definire il linguaggio generato
da una grammatica, introduciamo il concetto di
produzione estesa.

\begin{definition}[Applicazione estesa di una produzione]\label{def:applicazione-produzione-estesa}
Sia $\gramm$ una grammatica e siano $\alpha$, $\beta$ due forme sentenziali di $G$.
Allora possiamo scrivere $\alpha \eprod{G} \beta$ se esiste una sequenza
$\langle \gamma_{1}, \ldots, \gamma_{n}\rangle $ di $n$ forme sentenziali tali che $\alpha = \gamma_{1}$,
$\beta = \gamma_{n}$, e $\gamma_{i} \sprod{G} \gamma_{i+1}$ per ogni $1\le k< n$.
In questo caso diciamo che $\beta$ è il risultato dell'\keyword{applicazione estesa} di produzioni di $G$ alla stringa $\alpha$.
\end{definition}


\begin{definition}[linguaggio generato]\label{def:linguaggio-generato}
Sia \gramm una grammatica.
Il \keyword{linguaggio generato} da $G$, denotato con $L(G)$ è l'insieme $L(G)= \{ w\in T^{*}: S \eprod{G} w\}$.
\end{definition}

Inoltre la definizione di grammatica è estremamente poco vincolata, rendendo difficile capire quali sia il linguaggio
che una grammatica genera. Per questo motivo sono interessanti studiare grammatiche ristrette dove valgono proprietà più
stringenti di interesse sia per studiare le proprietà dei linguaggi generati che per risolvere più efficientemente il
problema di appartenenza.
Una prima classificazione delle grammatiche corrisponde alla gerarchia di Chomsky che classifica le grammatiche in tipo
0, 1, 2, 3.

\begin{definition}\label{def:tipo-0}
Ogni grammatica \gramm definita secondo la \cref{def:grammatica} è di tipo 0.
\end{definition}

\begin{definition}\label{def:tipo-1}
Ogni grammatica \gramm di tipo 0 per cui tutte le sue produzioni hanno lunghezza almeno pari a alla testa è di tipo 1.
Queste grammatiche sono anche dette \keyword{grammatiche dipendenti dal contesto}.
\end{definition}

Chiaramente le grammatiche di tipo 1 sono incluse in quelle di tipo 1 e si dimostra facilmente che l'inclusione è stretta.
Noi trascureremo queste grammatiche perchè quasi tutti i linguaggi artificiali, quali i linguaggi di programmazione,
sono generati da una grammatica più ristretta.
Le grammatiche di tipo 1 sono invece utilizzate nell'analisi dei linguaggi naturali.


\begin{definition}\label{def:tipo-2}\label{def:CFG}
Ogni grammatica \gramm di tipo 0 per cui tutte le sue produzioni hanno testa formata da esattamente una variabile e per
coda una forma sentenziale non vuota è di tipo 2.
Queste grammatiche sono anche dette \keyword{grammatiche libere dal contesto}.
\end{definition}

Quasi tutti i linguaggi artificiali sono generati da grammatiche di tipo 2: per questo motivo saranno oggetto di uno
studio estensivo in questo testo.
Spesso però siamo interessati ad ulteriori restrizioni.

\begin{definition}\label{def:tipo-3}
Ogni grammatica \gramm di tipo 0 per cui tutte le sue produzioni sono forma $A\to aB$ o $A\to a$, dove $A$ e $B$ sono
variabili e $a$ è un terminale, è di tipo 3.
Queste grammatiche sono anche dette \keyword{grammatiche regolari}.
\end{definition}

In realtà questa gerarchia ammette la possibilità di classi di grammatiche intermedie fra una classe e un'altra.
Vedremo nella \cref{part:parser} diverse classi di grammatiche che sono strettamente incluse in quelle di tipo 2 e
includono strettamente quelle di tipo 3.

Con un leggero abuso di linguaggio, talvolta diremo che un linguaggio è di tipo 0, 1, 2, 3 se è generato da una
grammatica di tipo 0, 1, 2, 3.

\section{Automi}\label{sec:automi}

Nella sezione precedente abbiamo introdotto il concetto di linguaggio generato da una grammatica che collega grammatiche
e linguaggi.
In questa sezione collegheremo classi di linguaggi con il modello di calcolo, o classe di automi, più semplice in grado di
risolvere il corrispondente problema di appartenza.
Non introduciamo una definizione formale di modello di calcolo che unifichi il concetto, ma introdurremo gli specifici
modelli per ogni classe di grammatiche che andremo a studiare.
Sintetizziamo nella \cref{tab:linguaggi-modelli-grammatiche} le relazioni che svilupperemo più avanti.


\begin{table*}[h!]\label{tab:linguaggi-modelli-grammatiche}
	\caption{Grammatiche, linguaggi e modelli di calcolo}
	\begin{tabular}{p{2.8cm} p{4.0cm} p{7.0cm}}
		\toprule
		Linguaggi di tipo   & Grammatiche   & Modello di calcolo \\
		\midrule
		0   & senza restrizioni       & Macchina di Turing        \\
		1     & dipendenti dal contesto     & Macchina di Turing con spazio lineare  \\
		2   & libere dal contesto     & Automi a pila      \\
		3   & regolari     & Automi a stati finiti      \\
		\bottomrule
	\end{tabular}
\end{table*}




\section{Grammatiche Context-Free}\label{sec:CFG}

\begin{example}
	Sia $G=(V,T,O,E)$, con $V=\{E,I\}$ e $T=\{a,b,0,1,(,),+,*\}$
	quindi ho le seguenti regole, è di tipo 3:
	\begin{enumerate}
		\item $E\to I$
		\item $E\to E+E$
		\item $E\to E*E$
		\item $E\to (E)$
		\item $I\to a$
		\item $I\to b$
		\item $I\to Ia$
		\item $I\to Ib$
		\item $I\to I0$
		\item $I\to I1$
	\end{enumerate}
	voglio ottenere $a*(a+b00)$
	sostituisco sempre a destra (right most derivation)
	$$E\to E*E\to E*(E)\to E*(E+E)\to E*(E+I)\to E+(E+I0)$$
	$$\to R+(I+b00)\to E*(a+b00)\to I*(a+b00)\to a*(a+b00)$$

	usiamo ora \textit{l'inferenza ricorsiva}:
	\begin{center}
		\begin{tabular}{|c|c|c|c|c|}
			\hline
			passo & stringa ricorsiva & var & prod & passo stringa impiegata \\
			1     & a                 & I   & 5    & $\backslash$            \\
			\hline
			2     & b                 & I   & 6    & $\backslash$            \\
			\hline
			3     & b0                & I   & 9    & 2                       \\
			\hline
			4     & b00               & I   & 9    & 3                       \\
			\hline
			5     & a                 & E   & 1    & 1                       \\
			\hline
			6     & b00               & E   & 1    & 4                       \\
			\hline
			7     & a+b00             & E   & 2    & 5,6                     \\
			\hline
			8     & (a+b00)           & E   & 4    & 7                       \\
			\hline
			9     & a*(a+b00)         & E   & 3    & 5, 8                    \\
			\hline
		\end{tabular}
	\end{center}
\end{example}
definisco formalmente la derivazione $\to$:
\begin{definition}
	Prendo una grammatica $G=(V,T,P,S)$, grammatica CFG. Se $\alpha A \beta$ è una stringa tale che $\alpha,\beta\in (V\cup T)^*$, appartiene sia a variabili che terminali. Sia $A\in V$ e sia $a\to \gamma$ una produzione di $G$. Allora
	scriviamo:
	$$\alpha A \beta \to \alpha\gamma\beta$$
	con $\gamma\in (V\cup T)^*$.\\
	Le sostituzioni si fanno indipendentemente da $\alpha$ e $\beta$.
	Questa è quindi la definizione di derivazione.
\end{definition}
\begin{definition}
	Definisco il simbolo $\to _*$, ovvero il simbolo di \textit{derivazioni in 0 o più passi}. Può essere definito in modo ricorsivo. Per induzione sul numero di passi.
	\begin{itemize}
		\item la base dice che  $\forall \alpha\in (V\cup T)^*,\, \alpha\to * \,\alpha$
		\item il passo è: se $\alpha\to_{G_*} \,\beta $ e $ \beta \to_{G_*} \,\gamma$ allora $\alpha\to_* \,\gamma$
	\end{itemize}
	Si può anche dire che $\alpha\to_{G_*}\, \beta$ sse esiste una sequenza di stringhe $\gamma_1,...,\gamma_n$ con $n\geq 1$ tale che $\alpha=\gamma_1$, $\beta=\gamma_n$ e $\forall i,\, 1<i<n-1$ si ha che $\gamma_1\to \gamma_{i+1}$
	la derivazione in 0 o più passi è la chiusura transitiva della derivazione
\end{definition}
\begin{definition}
	avendo ora definito questi simboli possiamo definire una forma sentenziale. Infatti è una stringa $\alpha$ tale che:
	$$\forall \alpha\in (V\cup T)^* \mbox{ tale che }S\to_{G_*}\, \alpha$$
\end{definition}
\begin{definition}
	data $G=(V,T,P,S)$ si ha che $L(G)=\{w\in T^* |\, S\to_{G_*}\, w\}$ ovvero composto da stringhe terminali che sono derivabili o 0 o più passi.
\end{definition}
\begin{example}
	formare una grammatica CFG per il linguaggio:
	$$L=\{0^n 1^n| n\geq 1\}=\{01, 0011, 000111,...\}$$
	con $x^n$ intendo una concatenazione di $n$ volte $x$ (che nel nostro caso sono 0 e 1).\\
	posso scrivere:
	$$0^n 1^n =00^{n-1} 1^{n-1}1$$
	il nostro caso base sarà la stringa $01$, Poi si ha:
	$G=(V,T,P,S)$, $T=\{0,1\}$, $V=\{S\}$, il caso base $S\to 01$  e $S\to 0S1$
	il caso passo è quindi: se $w= 0^{n-1}1^{n-1}\in L$ allora $0w1\in L$.\\
	Ora voglio dimostare che $000111\in L$, ovvero $S\to*\, 000111$:\\
	$$S\to\, 0S1 \to 00S11\to 000S111$$
\end{example}
\begin{theorem}
	data la grammatica $G=\{V,T,P,S)$ CFG e $\alpha\in (V\cup T)^*$. Si ha che vale $S\to_*\, \alpha$ sse $S\to_{lm_*}\, \alpha$ sse $S\to_{rm_*}\, \alpha$. Con $\to_{lm_*}$ simbolo di \textit{left most derivation }e $\to_{rm_*}$ simbolo di \textit{right most derivation}
\end{theorem}
\begin{example}
	formare una grammatica CFG per il linguaggio:
	$$L=\{0^n 1^n| n\geq 0\}=\{\varepsilon, 01, 0011, 000111,...\}$$
	stavolta abbiamo anche la stringa vuota. Il caso base stavolta è $S\to\varepsilon| \, 0S1$
\end{example}
\begin{example}
	Fornisco una CFG per $L=\{a^n|n\geq 1\}=\{a, aa, aaa,...\}$.
	La base è $a$ \\il passo è che se $a^{n-1}\in L$ allora $a^{n-1}a\in L$ ( o che $aa^{n-1}\in L$).\\
	Si ha la grammatica $G=\{V,T,P,S)$, $V=\{S\}$, $T=\{a\}$ e si hanno $S\to a|\,Sa$ (o  $S\to a|\,aS$). Dimostro che $a^3\in L$.
	$$S\to Sa \to Saa\to aaa$$
	oppure
	$$S\to aS\to aaS\to aaa$$
\end{example}
\begin{example}
	trovo una CFG per $L=\{(ab)^n|n\geq 1\}=\{ab, abab, ababab,...\}$\\
	La base è $ab$ \\il passo è che se $(ab)^{n-1}\in L$ allora $(ab)^{n-1}ab\in L$.\\
	Si ha la grammatica $G=\{V,T,P,S)$, $V=\{S\}$, $T=\{a,b\}$ (anche se in realtà $T=\{ab\}$) e si hanno $S\to ab|\,Aab$. Poi dimostro come l'esempio sopra
\end{example}
\begin{example}
	trovo una CFG per $L=\{a^n c b^n|n\geq 1\}=acb,aacbb,aaacbbb,...\}$\\
	Il caso base è $acb$ il passo è che se $a^{n-1}cb^{n-1}\in L$ allora $a^{n-1}cb^{n-1}acb\in L$
	Si ha la grammatica $G=\{V,T,P,S)$, $V=\{S\}$, $T=\{a,b,c\}$ e si hanno $S\to aSb|acb$.\\
	dimostro che $aaaacbbbbb\in L$:
	$$S\to aSb\to aaSbb\to aaaaSbbb\to aaaacbbbb$$

	provo a usare anche una grammatica regolare, con le regole $S\to aS|c$, $c\to cB$ e $B\to bB|b$;
	$$S\to aS\to aaS\to aaC\to aacB\to aacb...$$
	non si può dimostrare in quanto non si può imporre una regola adatta
\end{example}
\begin{example}
	$L=\{a^n c b^{n-1}|n\geq 2\}$, con $a^n c b^{n-1}=a^{n-1}acb^{n-1}$. $S\to aSb|aacb$. Quindi:
	$$S\to aSb\to aaaccbb\in L$$
\end{example}
\begin{example}
	cerco CFG per $L=\{a^n c^k b^n|\,n,\,k>0\}$. $a$ e $b$ devono essere uguali, uso quindi una grammatica context free, mentre $c$ genera un linguaggio regolare.\\
	Si ha la grammatica $G=\{V,T,P,S)$, $V=\{S,C\}$, $T=\{a,b,c\}$ e si hanno $S\to aSb|aCb$ e $C\to cC|c$. dimostro che $aaaccbbb\in L, n=3,\, k=2$:
	$$S\to aSb \to aaSbb\to aaaCbbb\to aaacCbbb\to aaaccbbb$$
\end{example}
\begin{example}
	scrivere CFG per $L=\{a^nb^nc^kb^k|\, n,\,k\geq 0\}
	$
	$$=\{w\in\{a,b,c,d\}^*|\,a^nb^nc^kb^k|\, n,\,k\geq 0\}$$
	quindi L concatena due linguaggi $L1$ e $L2$, $X=\{a^nb^n\}$ e $Y=\{c^kd^k\}$:
	$$X\to aXb | \varepsilon$$
	$$Y\to cYd | \varepsilon$$
	$$S\to XY$$
	voglio derivare $abcd$:
	$$S\to XY \to XcYd\to aXbcYd\to aXbc\varepsilon d\to a\varepsilon bc\varepsilon d\to abcd$$
	voglio derivare $cd$
	$$S\to XY\to Y\to cYd\to cd$$
\end{example}
Quindi se ho $w\in L1, L2$, ovvero appartenente ad una concatenazione di linguaggi prima uso le regole di un linguaggio, poi dell'altro e infine ottengo il risultato finale.\\
\begin{example}
	scrivere CFG per $L=\{a^nb^kc^kd^n|\, n>0,\, k\geq 0\}
	$.
	$$S\to aSd|\, aXd$$
	$$X\to bXc| \varepsilon$$
	derivo $aabcdd$:
	$$S\to aSd\to aaXdd\to aabXcdd\to aabcdd$$
\end{example}
\begin{example}
	scrivere CFG per $L=\{a^ncb^nc^mad^m|\, n>0,\, m\geq 1\}
	$.
	$$S\to XY$$
	$$X\to aXb|c$$
	$$Y\to cUd| cad$$
	$$S\to XY\to cY\to ccad$$
\end{example}
\begin{example}
	scrivere CFG per $L=\{a^{n+m}xc^nyd^m|\, n,\, m\geq 0\}
	$. $a^{n+m}=a^na^m \mbox{ o } a^ma^n$. Si hanno 2 casi:
	\begin{enumerate}
		\item $L=\{a^na^m xc^nyd^m|\, n,\, m\geq 0\}
					$
		\item $L=\{a^ma^n xc^nyd^m|\, n,\, m\geq 0\}
					$
	\end{enumerate}
	ma solo  $L=\{a^ma^n xc^nyd^m|\, n,\, m\geq 0\}
	$ può generare una CFG (dove non si possono fare incroci, solo concatenazioni e inclusioni/innesti).
	$$S\to aSd| Y$$
	$$Y\to Xy$$
	$$X\to aXc|x$$
	si può fare in 2:
	$$S\to aSd| Xy$$
	$$X\to aXc|x$$
	derivo con $m=n=1$, $aaxcyd$:
	$$S\to aSd\to aXyd\to aaXcyd\to aaxcyd$$
\end{example}
\begin{example}
	scrivere CFG per $L=\{a^nb^m|\, n\geq m \geq 0\}
	$.$$L=\{\varepsilon, a, ab, aa, aab, aabb, aaa, aaab, aaabb, aaabbb,...\}$$
	Se $n\geq m$ allora $\exists k\geq 0 \to n=m+k$. Quindi:
	$$l=\{a^{m+k}b^m|m,k\geq0\}$$ si può scrivere in 2 modi:
	\begin{enumerate}
		\item $l=\{a^ma^kb^m|m,k\geq0\}$ quindi con innesto
		\item $l=\{a^ka^mb^m|m,k\geq0\}$quindi con concatenazione
	\end{enumerate}
	entrambi possibili per una CFG:
	\begin{enumerate}
		\item
					$$S\to XY$$
					$$X\to aX|\varepsilon \mbox{ si può anche scrivere } X\to Xa|\varepsilon$$
					$$Y\to aYb|\varepsilon$$
					oppure
					$$S\to aS|X$$
					$$X\to aXb| \varepsilon$$
		\item
					$$S\to aSb|\varepsilon$$
					$$X\to aX|\varepsilon$$
	\end{enumerate}
\end{example}
\begin{example}
	scrivere CFG per $L=\{a^nb^{m+n}c^h|\, m>h\geq0,\, n\geq0\}
	$.\\
	Se $n>h$ allora $\exists k \to n= h+k$, quindi:
	$$L=\{a^nb^{m+h+k}c^h|\, m>h\geq0,\, n\geq0\}$$. ovvero:
	$$L=\{a^nb^nb^kb^hc^h|\, m\geq 0, k>0, h\geq 0\}$$
	si ha:
	$$S\to XYZ$$
	$$X\to aXb|\varepsilon$$
	$$Y\to Yb|b$$
	$$Z\to bZc|\varepsilon$$
	si può anche fare:
	$$S\to XY$$
	$$X\to aXb|\varepsilon$$
	$$Y\to bYc|Z$$
	$$Z\to bZ|b$$
\end{example}
\begin{example}
	scrivere CFG per $L=\{a^nb^mc^k|\, k>n+m,\, n,m\geq 0\}
	$.\\
	per $n=m=0,\, k=1$ avrò la stringa $c$.
	se $k>n+m$ allora $\exists l>0\to k=n+m+l$ quindi:
	$$L=\{a^nb^mc^{n+m+l}|\, l>0,\, n,m\geq 0\}
	$$
	$$=L=\{a^nb^mc^nc^mc^l|\, l>0,\, n,m\geq 0\}$$
	sistemando:
	$$=L=\{a^nb^mc^lc^mcnl|\, l>0,\, n,m\geq 0\}$$
	quindi:
	$$S\to aSc|X$$
	$$X\to bXc|Y$$
	$$Y\to cY|c$$
\end{example}
\newpage
\begin{example}
	scrivere CFG per $L=\{a^nxc^{n+m}y^hz^kd^{m+h}|\, n,m,k,h\geq 0\}
	$.\\
	ovvero:
	$$L=\{a^nxc^nc^my^hz^kd^hd^m|\, n,m,k,h\geq 0\}$$
	quindi avrò:
	$$S\to XY$$
	$$X\to aXc|x$$
	$$Y\to cYd|W$$
	$$W\to yWd|X$$
	$$Z\to zZ|\varepsilon$$
\end{example}
\begin{example}
	vediamo un esempio di grammatica dipendente dal contesto:
	$$L=\{a^nb^nc^n|\, n\geq 1\}$$
	$G=\{V,T,P,S\}=\{(S,B,C,X)\}=\{(a,b,c),P,S\}$
	ecco le regole di produzione (qui posso scambiare variabili a differenza delle context free):
	\begin{enumerate}
		\item $S\to aSBC$
		\item $S\to aBC$
		\item $CB\to XB$
		\item $XB\to XC$
		\item $XC\to BC$
		\item $aB\to ab$
		\item $bB\to bb$
		\item $bC\to bc$
		\item $cC\to cc$
	\end{enumerate}
	vediamo un esempio di derivazione:
	per $n=1$ ho $abc$ ovvero:
	$$S\to aBC\to abC\to abc$$
	con $n=2$ ho $aabbcc$:
	$S\to aSBC\to aaBCBC\to aaBXBC\to aaBXCC\to aaBBCC\to aabBCC\to aabbCC\to aabbcC\to aabbcc$\\
	%vedere dimostrazione pag 14 soligo
\end{example}
\newpage
\begin{example}
	vediamo un esempio di grammatica dipendente dal contesto:
	$$L=\{a^nb^mc^nd^m|\, n,m\geq 1\}$$
	Si ha:
	$$G=(\{S,X,C,D,Z\},\{a,b,c,d\},P,S)$$
	con le seguenti regole di produzione:
	\begin{itemize}
		\item $S\to aSc|\, aXc$
		\item $X\to bXD|\, bD$
		\item $DC\to CD$
		\item $DC\to DZ$
		\item $DZ\to CZ$
		\item $XZ\to CD$
		\item $bC\to bc$
		\item $cC\to cc$
		\item $cD\to cd$
		\item $dD\to dd$
	\end{itemize}
	provo a derivare $aabbbccddd$ quindi con $n=2,\,m=3$:\\
	$$S\to aSC\to aaXCC\to aabXDCC\to aabbXDDCC\to $$
	$$aabbbDDDCC\to aabbbCCDDD\to aabbbccddd$$
\end{example}
\begin{example}
	Sia $L=\{w\in\{a,b\}^*|\, \mbox{ w contiene lo stesso numero di a e b}\}$:
	$$S\to aSbS|\,bSaS|\, \varepsilon$$
	dimostro per induzione che è corretto:
	\begin{itemize}
		\item \textbf{caso base:} $|w|=0\to w=\varepsilon$
		\item \textbf{caso passo:} si supponga che $G$ produca tutte le stringhe (di lunghezza $<$ di $n$) di $\{a,b\}^*$ con lo stesso numero di \textit{a} e \textit{b} e dimostro che produce anche quelle di lunghezza $n$, sia:
					$$w\in \{a,b\}^* \mid\, |w|=n \mbox{ con\textit{ a} e \textit{b} in egual numero, }m(a)=m(b) \mbox{ con m() che indica il numero di caratteri}$$
					quindi si ha che:
					$$w=aw_1bw_2\mbox{ o } w=bw_1aw_2$$
					sia.
					$$k_1=m(a)\in w_1=m(b)\in w_1$$
					$$k_2=m(a)\in w_2=m(b)\in w_2$$
					allora:
					$$k_1+k_2+1=m(a)\in w= m(b)\in W$$
					sapendo che $|w_1|<n$ e $|w_2|<n$ allora $w_1$ e $w_2$ sono egnerati da G per ipotesi induttiva
	\end{itemize}
\end{example}

\section{Alberi Sintatici}
\begin{definition}
	Data una grammatica CFG, $G=\{V,T,P,S\}$ un \textbf{albero sintattico} per $G$ soddisfa le seguenti condizioni:
	\begin{itemize}
		\item ogni nodo interno è etichettato con una variabile in $V$
		\item ogni foglia è anch'essa etichettata con una variabile o col simbolo di terminale T o con la stringa vuota $\varepsilon$ (in questo caso la foglia è l'unico figlio del padre)
		\item se un nodo interno è etichettato con A i suoi figli saranno etichettati con X1, ..., Xk e $A\to  X1, ..., Xk$ sarà una produzione di $G$ in $P$. Se un $X_i$ è $\varepsilon$ sarà l'unica figlio e $A\to \varepsilon$ sarà comunque una produzione di $G$
	\end{itemize}
	La concatenazione in ordine delle foglie viene detto \textbf{prodotto dell'albero}
\end{definition}
\newpage
\begin{example}
	Usiamo l'esempio delle stringhe palindrome:
	$$P\to 0P0|\,1P1|\varepsilon$$
	sia il seguente albero sintatico:
	\begin{center}
\includegraphics{example_1.1.1.png
}
%		\psframebox[linestyle=none,framesep=10pt]{%
%			\pstree{\LFTw{t}{\fontspec{Noto Sans}[Script=Latin]P}}{\pstree{\Tp[edge=none]}{%
%					\LFTw{t}{\fontspec{Noto Sans}[Script=Latin]0}
%					\pstree{\LFTw{t}{\fontspec{Noto Sans}[Script=Latin]P}}{\pstree{\Tp[edge=none]}{%
%							\LFTw{t}{\fontspec{Noto Sans}[Script=Latin]1}
%							\pstree{\LFTw{t}{\fontspec{Noto Sans}[Script=Latin]P}}{\pstree{\Tp[edge=none]}{%
%									\LFTw{t}{\fontspec{Noto Sans}[Script=Latin]$\varepsilon$}}}
%							\LFTw{t}{\fontspec{Noto Sans}[Script=Latin]1}}}
%					\LFTw{t}{\fontspec{Noto Sans}[Script=Latin]0}}}}
	\end{center}
\end{example}
\begin{example}
	Si ha:
	$$E\to I|\, E+E|\, E*E|\, (E)$$
	$$I\to a|\,b|\,Ia|\,Ib|\,I0|\,I1$$
	un albero sintattico per $a*(a+b00)$ può essere:
	\begin{center}
\includegraphics{example_1.1.2.png}
%		\psframebox[linestyle=none,framesep=10pt]{%
%			\pstree{\LFTw{t}{\fontspec{Noto Sans}[Script=Latin]E}}{\pstree{\Tp[edge=none]}{%
%					\pstree{\LFTw{t}{\fontspec{Noto Sans}[Script=Latin]E}}{\pstree{\Tp[edge=none]}{%
%							\pstree{\LFTw{t}{\fontspec{Noto Sans}[Script=Latin]I}}{\pstree{\Tp[edge=none]}{%
%									\LFTw{t}{\fontspec{Noto Sans}[Script=Latin]a}}}}}
%					\LFTw{t}{\fontspec{Noto Sans}[Script=Latin]*}
%					\pstree{\LFTw{t}{\fontspec{Noto Sans}[Script=Latin]E}}{\pstree{\Tp[edge=none]}{%
%							\LFTw{t}{\fontspec{Noto Sans}[Script=Latin](}
%								\pstree{\LFTw{t}{\fontspec{Noto Sans}[Script=Latin]E}}{\pstree{\Tp[edge=none]}{%
%										\pstree{\LFTw{t}{\fontspec{Noto Sans}[Script=Latin]E}}{\pstree{\Tp[edge=none]}{%
%												\pstree{\LFTw{t}{\fontspec{Noto Sans}[Script=Latin]I}}{\pstree{\Tp[edge=none]}{%
%														\LFTw{t}{\fontspec{Noto Sans}[Script=Latin]a}}}}}
%										\LFTw{t}{\fontspec{Noto Sans}[Script=Latin]+}
%										\pstree{\LFTw{t}{\fontspec{Noto Sans}[Script=Latin]E}}{\pstree{\Tp[edge=none]}{%
%												\pstree{\LFTw{t}{\fontspec{Noto Sans}[Script=Latin]I}}{\pstree{\Tp[edge=none]}{%
%														\pstree{\LFTw{t}{\fontspec{Noto Sans}[Script=Latin]I}}{\pstree{\Tp[edge=none]}{%
%																\LFTw{t}{\fontspec{Noto Sans}[Script=Latin]b}}}
%														\LFTw{t}{\fontspec{Noto Sans}[Script=Latin]0}}}
%												\LFTw{t}{\fontspec{Noto Sans}[Script=Latin]0}}}}}
%								\LFTw{t}{\fontspec{Noto Sans}[Script=Latin])}}}}}}
	\end{center}
\end{example}
\newpage
Data una CFG si ha che i seguenti cinque enunciati si equivalgono:
\begin{enumerate}
	\item la procedura di inferenza ricorsiva stailisce che una stringa $w$ di simboli terminali appartiene al linguaggio $L(A)$ con $A$ variabile
	\item $A\to ^*w$
	\item $A\to^*_{lm}w$
	\item $A\to^*_{rm}w$
	\item esiste un albero sintattico con radice $A$ e prodotto $w$
\end{enumerate}
queste 5 proposizioni si implicano l'uni l'altra:
\begin{center}
	\begin{tikzpicture}
		\node (top) at (0,0) {5};
		\node (a) at(-1,-0.5) {3};
		\node (b) at(0,-1) {4};
		\node (c) at(-2.0,-1.85) {2};
		\node (d) at(1.5,-2) {1};
		\draw [->] (top) -- (a);
		\draw [->] (top) -- (b);
		\draw [->] (a) -- (c);
		\draw [->] (b) -- (c);
		\draw [->] (c) -- (d);
		\draw [->] (d) -- (top);
	\end{tikzpicture}
\end{center}
vediamo qualche dimostrazione di implicazione tra queste proposizioni:
\begin{proof}[da 1 a 5]
	si procede per induzione:
	\begin{itemize}
		\item \textbf{caso base:} ho un livello solo (una sola riga), $\exists A\to w$:
					$$\overset{A}{\overset{\triangle}w}$$
		\item \textbf{caso passo:} suppongo vero per un numero di righe $\leq n$, lo dimsotro per $n+1$ righe:
					$$A\to X_1,X_2,...,X_k$$
					$$w=w_1,w_2,...,w_k$$
					ovvero, in meno di $n+1$ livelli:
					\begin{center}
\includegraphics{caso_passo.png}
					% \psframebox[linestyle=none,framesep=10pt]{%
%				      \pstree{\LFTw{t}{\fontspec{Noto Sans}[Script=Latin]A}}{\pstree{\Tp[edge=none]}{%
%						      \LFTw{t}{\fontspec{Noto Sans}[Script=Latin]$\overset{X_1}{\overset{\triangle}w_1}$}
%						      \LFTw{t}{\fontspec{Noto Sans}[Script=Latin]$\overset{X_2}{\overset{\triangle}w_2}$}
%						      \LFTw{t}{\fontspec{Noto Sans}[Script=Latin]$\vdots$}
%						      \LFTw{t}{\fontspec{Noto Sans}[Script=Latin]$\overset{X_k}{\overset{\triangle}w_k}$}}}}
					\end{center}
	\end{itemize}
\end{proof}
\begin{proof}[da 5 a 3]
	procedo per induzione:
	\begin{itemize}
		\item \textbf{caso base (n=1): }$\exists A\to w\mbox{ quindi } A\to_{lm}w$, come prima si ha un solo livello:
					$$\overset{A}{\overset{\triangle}w}$$
		\item \textbf{caso passo: }suppongo che la proprierà valga per ogni albero di profondità minore uguale a $n$, dimostro che valga per gli alberi profondi $n+1$:
					$$A\to X_1,X_2,...,X_k$$
					$$w=w_1,w_2,...,w_k$$
					ovvero, in meno di $n+1$ livelli:
					\begin{center}
\includegraphics{caso_passo.png}
%			      \psframebox[linestyle=none,framesep=10pt]{%
%				      \pstree{\LFTw{t}{\fontspec{Noto Sans}[Script=Latin]A}}{\pstree{\Tp[edge=none]}{%
%						      \LFTw{t}{\fontspec{Noto Sans}[Script=Latin]$\overset{X_1}{\overset{\triangle}w_1}$}
%						      \LFTw{t}{\fontspec{Noto Sans}[Script=Latin]$\overset{X_2}{\overset{\triangle}w_2}$}
%						      \LFTw{t}{\fontspec{Noto Sans}[Script=Latin]$\vdots$}
%						      \LFTw{t}{\fontspec{Noto Sans}[Script=Latin]$\overset{X_k}{\overset{\triangle}w_k}$}}}}
					\end{center}
					$$A\to_{lm} X_1,X_2,...,X_k$$
					$$x_1\to^*_{lm}w_1 \mbox{ per ipotesi induttiva si ha un albero al più di n livelli}$$
					quindi:
					$$A\to_{lm}X_1,...,X_k\to^*_{lm}w_1,X_2,...,X_k\to^*_{lm}...\to^*_{lm}w_1,...,w_k=w$$
	\end{itemize}
	\begin{example}
		$$E\to I\to Ib\to ab$$
		$$\alpha E\beta\to\alpha I\beta\to \alpha Ib\beta\to \alpha ab\beta,\,\,\,\alpha,\beta\in(V\cup T)^*$$
	\end{example}
\end{proof}
\begin{example}
	Mostro l'esistenza di una derivazione sinistra dell'albero sintattico di $a*(a+b00)$:
	$$E\to^*_{lm}E*E\to^*_{lm}I*E\to^*_{lm}a*E\to^*_{lm}a*(E)\to^*_{lm}a*(E+E)\to^*_{lm}$$
	$$a*(I+E)\to^*_{lm}a*(a+E)\to^*_{lm}a*(a+I)\to^*_{lm}a+(a+I0)\to^*_{lm}a*(a+I00)\to^*_{lm}a*(a+b00)$$
\end{example}
\section{Grammatiche ambigue}
\begin{definition}
	Una grammatica è definita ambigua se esiste una stringa $w$ di terminali che ha più di un albero sintattico
\end{definition}
\begin{example}
	vediamo un esempio:
	\begin{enumerate}
		\item $E\to E+E\to E+E*E$
					ovvero:
					\begin{center}
\includegraphics{ambigua1.png}
%			      \psframebox[linestyle=none,framesep=10pt]{%
%				      \pstree{\LFTw{t}{\fontspec{Noto Sans}[Script=Latin]E}}{\pstree{\Tp[edge=none]}{%
%						      \LFTw{t}{\fontspec{Noto Sans}[Script=Latin]E}
%						      \LFTw{t}{\fontspec{Noto Sans}[Script=Latin]+}
%						      \pstree{\LFTw{t}{\fontspec{Noto Sans}[Script=Latin]E}}{\pstree{\Tp[edge=none]}{%
%								      \LFTw{t}{\fontspec{Noto Sans}[Script=Latin]E}
%								      \LFTw{t}{\fontspec{Noto Sans}[Script=Latin]*}
%								      \LFTw{t}{\fontspec{Noto Sans}[Script=Latin]E}}}}}}
					\end{center}
		\item $E\to E*E\to E+E*E$
					ovvero:
					\begin{center}
%			      \psframebox[linestyle=none,framesep=10pt]{%
%				      \pstree{\LFTw{t}{\fontspec{Noto Sans}[Script=Latin]E}}{\pstree{\Tp[edge=none]}{%
%						      \pstree{\LFTw{t}{\fontspec{Noto Sans}[Script=Latin]E}}{\pstree{\Tp[edge=none]}{%
%								      \LFTw{t}{\fontspec{Noto Sans}[Script=Latin]E}
%								      \LFTw{t}{\fontspec{Noto Sans}[Script=Latin]+}
%								      \LFTw{t}{\fontspec{Noto Sans}[Script=Latin]E}}}
%						      \LFTw{t}{\fontspec{Noto Sans}[Script=Latin]*}
%						      \LFTw{t}{\fontspec{Noto Sans}[Script=Latin]E}}}}
					\end{center}
	\end{enumerate}
	si arriva a due stringhe uguali ma con alberi diversi. Introduciamo delle categorie sintatiche, dei vincoli alla produzione delle regole:
	\begin{enumerate}
		\item $E\to T|\, E+T$
		\item $T\to F|\, T+F$
		\item $F\to I|\, (E)$
		\item $I\to a|\,b|\,Ia|,Ib|\,I0|\,I1$
	\end{enumerate}
\end{example}
Possono esserci più derivazioni di una stringa ma l'importante è che non ci siano alberi sintattici diversi. Capire se una CFG è ambigua è un problema indecidibile
\begin{example}
	vediamo un esempio:
	$$S\to \varepsilon|\,SS|\, iS|\, iSeS$$
	con S=statement, i=if e e=else. Considero due derivazioni:
	\begin{enumerate}
		\item $S\to iSeS\to iiSeS\to iie$:
					\begin{center}
\includegraphics{ambigua2.png}

%			      \psframebox[linestyle=none,framesep=10pt]{%
%				      \pstree{\LFTw{t}{\fontspec{Noto Sans}[Script=Latin]S}}{\pstree{\Tp[edge=none]}{%
%						      \LFTw{t}{\fontspec{Noto Sans}[Script=Latin]i}
%						      \pstree{\LFTw{t}{\fontspec{Noto Sans}[Script=Latin]S}}{\pstree{\Tp[edge=none]}{%
%								      \LFTw{t}{\fontspec{Noto Sans}[Script=Latin]i}
%								      \pstree{\LFTw{t}{\fontspec{Noto Sans}[Script=Latin]S}}{\pstree{\Tp[edge=none]}{%
%										      \LFTw{t}{\fontspec{Noto Sans}[Script=Latin]$\varepsilon$}}}}}
%						      \LFTw{t}{\fontspec{Noto Sans}[Script=Latin]e}
%						      \pstree{\LFTw{t}{\fontspec{Noto Sans}[Script=Latin]S}}{\pstree{\Tp[edge=none]}{%
%								      \LFTw{t}{\fontspec{Noto Sans}[Script=Latin]$\varepsilon$}}}}}}\end{center}
%		\item $S\to iS\to iiSeS\to iieS\to iie$:
%		      \begin{center}
%
%			      \psframebox[linestyle=none,framesep=10pt]{%
%				      \pstree{\LFTw{t}{\fontspec{Noto Sans}[Script=Latin]S}}{\pstree{\Tp[edge=none]}{%
%						      \LFTw{t}{\fontspec{Noto Sans}[Script=Latin]i}
%						      \pstree{\LFTw{t}{\fontspec{Noto Sans}[Script=Latin]S}}{\pstree{\Tp[edge=none]}{%
%								      \LFTw{t}{\fontspec{Noto Sans}[Script=Latin]i}
%								      \pstree{\LFTw{t}{\fontspec{Noto Sans}[Script=Latin]S}}{\pstree{\Tp[edge=none]}{%
%										      \LFTw{t}{\fontspec{Noto Sans}[Script=Latin]$\varepsilon$}}}
%								      \LFTw{t}{\fontspec{Noto Sans}[Script=Latin]e}
%								      \pstree{\LFTw{t}{\fontspec{Noto Sans}[Script=Latin]S}}{\pstree{\Tp[edge=none]}{%
%										      \LFTw{t}{\fontspec{Noto Sans}[Script=Latin]$\varepsilon$}}}}}}}}
					\end{center}
	\end{enumerate}
	Si ha quindi una grammatica ambigua
\end{example}
\begin{theorem}
	Per ogni CFG, con $G=(V, T, P, S)$, per ogni stringa $w$ di terminali si ha che $w$ ha due alberi sintattici distinti sse ha due derivazioni sinistre da S distinte.\\
	Se la grammatica non è ambigua allora esiste un'unica derivazione sinistra da $S$
\end{theorem}
\subsubsection{Linguaggi inerentemente ambigui}
\begin{definition}
	Un linguaggio $L$ è inerentemente ambiguo se tutte le grammatiche CFG per tale linguaggio sono a loro volta ambigue
\end{definition}
\begin{example}
	Sia $L=\{a^nb^nc^md^m|\, n,m\geq 1\}\cup \{a^nbmnc^md^n|\, n,m\geq 1\}$\\
	si ha quindi un CFL formato dall'unione di due CFL. $L$ è inerentemente ambiguo e generato dalla seguente grammatica:
	\begin{itemize}
		\item $S\to AB|\,C$
		\item $A\to aAb|\,ab$
		\item $B\to cBd|\, cd$
		\item $C\to aCd|\, aDd$
		\item $D\to bDc|\, bc$
	\end{itemize}
	si possono avere due derivazioni:
	\begin{enumerate}
		\item $S\to_{lm}AB\to_{lm} aAbB\to_{lm} aabbB\to_{lm}aabbcBd\to_{lm}aabbccdd$
		\item $S\to_{lm} C\to_{lm} aCd\to_{lm}aaBdd\to_{lm}aabBcdd\to_{lm}aabbccdd$
	\end{enumerate}
	a generare problemi sono le stringhe con n=m perché possono essere prodotte in due modi diversi da entrambi i sottolinguaggi. Dato che l'intersezione tra i due sottolinguaggi non è buota si ha che $L$ è ambiguo
\end{example}
\section{Grammatiche Regolari}
Sono le grammatiche che generano i linguaggi regolari (quelli del terzo tipo) che sono casi particolari dei CFL.\\
Si ha la solita grammatica $G = (V, T, P, S)$ con però vincoli su $P$:
\begin{itemize}
	\item $\varepsilon$ si può ottenere solo con $S\to \varepsilon$
	\item le produzioni sono tutte lineari a destra ($A\to aA$ o $A\to a$) o a sinistra ($A\to Ba$ o $A\to a$)
\end{itemize}
\begin{example}
	$I\to a|\,b|\,Ia|\,Ib|\,I0|\,I1$ è una grammatica con le produzioni lineari a sinistra.\\
	Potremmo pensarlo a destra $I\to a|\,b|\,aI|\,bI|\,0I|\,1I$.\\
	Vediamo esempi di produzione con queste grammatiche:
	\begin{itemize}
		\item con $I\to a|\,b|\,Ia|\,Ib|\,I0|\,I1$ possiamo derivare $ab01b0$:
					$$I\to I0\to Ib0\to I1b0\to I01b0\to Ib01b0\to ab01b0$$
		\item con $I\to a|\,b|\,aI|\,bI|\,0I|\,1I$ invece non riusciamo a generare nulla:
					$$I\to 0I\to 0a$$
	\end{itemize}
	definisco quindi un'altra grammatica (con una nuova categoria sintattica):
	$$I\to aJ|\, bJ$$
	$$J\to a|\,b|\,aJ|\,bJ|\,0J|\,1J$$
	che però non mi permette di terminare le stringhe con 0 e 1, la modifico ancora otterdendo:
	$$I\to aJ|\, bJ$$
	$$J\to a|\,b|\,aJ|\,bJ|\,0J|\,1J|\,0|\,1$$
	e questo è il modo corretto per passare da lineare sinistra a lineare destra
\end{example}
\begin{example}
	Sia $G=(\{S\},\{0,1\},P,S)$ con $S\to \varepsilon|\,0|\,1|\,0S|\,1S$. Si ha quindi:
	$$L(G)=\{0,1\}^*$$
	si hanno comunque due proposizioni ridondanti, riducendo trovo:
	$$S\to \varepsilon|\, 0S|\,1S$$
	con solo produzioni lineari a destra. Con produzioni lineari a sinistra ottengo:
	$$S\to \varepsilon|\, S0|\,S1$$
\end{example}
\begin{example}
	Trovo una grammatica lineare destra e una sinistra per $L=\{a^nb^m|\,n,m\geq 0\}$:
	\begin{itemize}
		\item \textbf{lineare a destra:} si ha $G=(\{S,B\},\{a,b\},P,S)$ e quindi:
					$$S\to \varepsilon|\,aS|\,bB$$
					$$B\to bB|\,b$$
					ma non si possono generare stringhe di sole $b$, infatti:
					$$S\to aS\to abB\to abbB\to abbb$$
					ma aggiungere $\varepsilon$ a B \textbf{non è lecito}. posso però produrre la stessa stringa da due derivazioni diverse:
					$$S\to \varepsilon|\,aS|\,bB|\,b$$
					$$B\to bB|\,b$$
					che risulta quindi la nostra lineare a destra
		\item \textbf{lineare a sinistra:} si ha $G=(\{S,A\},\{a,b\},P,S)$ e quindi:
					$$S\to \varepsilon|\,Sb|\,Ab|\,a$$
					$$A\to Aa|\,a$$
	\end{itemize}
\end{example}
\begin{example}
	Trovo una grammatica lineare destra e una sinistra per $L=\{ab^ncd^me|\,n\geq 0\,,m> 0\}$:
	\begin{itemize}
		\item \textbf{lineare a destra:} si ha  si ha $G=(\{S,A,B,E\},\{a,b,c,d,e\},P,S)$ e quindi:
					$$S\to aA$$
					$$A\to bA|\,cB$$
					$$B\to dB|\, dE$$
					$$E\to e$$
		\item \textbf{lineare a sinistra:} si ha  si ha $G=(\{S,X,Y,Z\},\{a,b,c,d,e\},P,S)$ e quindi:
					$$S\to Xe$$
					$$A\to Xd|\,Yd$$
					$$B\to Zc$$
					$$E\to a|\,Zb$$
	\end{itemize}
	quindi se per esempio ho la stringa "ciao" si ha:
	\begin{itemize}
		\item \textbf{lineare a destra:}
					$$S\to Ao$$
					$$A\to Ba$$
					$$B\to Ei$$
					$$E\to c$$
		\item \textbf{lineare a sinistra:}
					$$S\to cA$$
					$$A\to iB$$
					$$B\to aE$$
					$$E\to o$$
	\end{itemize}
\end{example}
\begin{example}
	A partire da $G=(\{S,T\},\{0,1\},P,S)$ con:
	$$S\to\varepsilon|\,0S|\,1T$$
	$$T\to 0T|\,1S$$

	trovo come è fatto $L(G)$:
	$$L(G)=\{w\in\{0,1\}^*|\, w \mbox{ ha un numero di 1 pari}\}$$
\end{example}
\begin{example}
	fornire una grammatica regolare a destra e sinistra per:
	$$L=\{w\in\{0,1\}^*|\, w \mbox{ ha almeno uno 0 o almeno un 1}\}$$
	Si ah che tutte le stringhe tranne quella vuota ciontengono uno 0 o un 1
	quindi  $G=(\{S\},\{0,1\},P,S)$:
	\begin{itemize}
		\item \textbf{lineare a destra:}
					$$S\to 0|\,1|\,0S|\,1S$$
		\item \textbf{lineare a sinistra:}
					$$S\to 0|\,1|\,S0|\,S1$$
	\end{itemize}
\end{example}



Vediamo ora un esempio di \textit{Context Free Language (CFL)}, costruito a partire da una \textit{Context Free Grammar (CFG)}:
\begin{example}
	Sia $\Sigma=\{0, 1\}$ e $L_{pal}="stringhe\,\, palindrome\,\, binarie"$.
	Quindi, per esempio, $0110\in L,\,\, 11011\in L$ ma $10010\not\in L$. Si ha che $\varepsilon$, la stringa vuota, appartiene a $L$. Diamo una definizione ricorsiva:
	\begin{itemize}
		\item \textbf{base:} $\varepsilon,\, 0\,\ 1\in L_{pal}$
		\item \textbf{passo:} se $w$ è palindroma allora $0w0$ è palindromo e $1w1$ è palindromo
	\end{itemize}
	una variabile generica $S$ può sottostare alle \textit{regole di produzione} di una certa grammatica. In questo caso si ha uno dei seguenti:
	$$S\to\varepsilon,\, S\to 0,\, S\to 1,\, S\to 0S0,\, S\to 1S1$$
\end{example}


riprendiamo l'esempio sopra:
\begin{example}
	$$G_{pal}=(V=\{S\},\, T=\{0, 1\},\, P,\, S)$$
	con:
	$$P=\{S\to\varepsilon,\, S\to 0,\, S\to 1,\, S\to 0S0,\, S\to 1S1\}$$
	Si può ora costruire un algoritmo per creare una stringa palindroma a partire dalla grammatica $G$:
	$$\underbrace{S}_{\mbox{start}}\underbrace{\to}_{\mbox{applico una regola}} 1S1 \to 01S10\to \underbrace{01010}_{\mbox{sostituisco variabile}}$$

	con $S,\, 1S1\,\, e\,\, 01S10$ che sono \textit{forme sentenziali}. Posso così ottenere tutte le possibili stringhe. Esiste anche una forma abbreviata:
	$$S\to \varepsilon|o|1|0S0|1S1$$
	Non si fanno sostituzioni in parallelo, prima una $S$ e poi un'altra
\end{example}

%%% Local Variables:
%%% mode: LaTeX
%%% TeX-master: "../libro-linguaggi"
%%% TeX-engine: luatex
%%% End:

\input{capitoli/automi-stati-finiti.tex}
\setchapterpreamble[u]{\margintoc}
\chapter{Linguaggi liberi dal contesto}\label{cha:CFG}
\labch{Linguaggi liberi dal contesto}







\section{Grammatiche Context-Free}\label{sec:CFG}

\begin{example}
	Sia $G=(V,T,O,E)$, con $V=\{E,I\}$ e $T=\{a,b,0,1,(,),+,*\}$
	quindi ho le seguenti regole, è di tipo 3:
	\begin{enumerate}
		\item $E\to I$
		\item $E\to E+E$
		\item $E\to E*E$
		\item $E\to (E)$
		\item $I\to a$
		\item $I\to b$
		\item $I\to Ia$
		\item $I\to Ib$
		\item $I\to I0$
		\item $I\to I1$
	\end{enumerate}
	voglio ottenere $a*(a+b00)$
	sostituisco sempre a destra (right most derivation)
	$$E\to E*E\to E*(E)\to E*(E+E)\to E*(E+I)\to E+(E+I0)$$
	$$\to R+(I+b00)\to E*(a+b00)\to I*(a+b00)\to a*(a+b00)$$

	usiamo ora \textit{l'inferenza ricorsiva}:
	\begin{center}
		\begin{tabular}{|c|c|c|c|c|}
			\hline
			passo & stringa ricorsiva & var & prod & passo stringa impiegata \\
			1     & a                 & I   & 5    & $\backslash$            \\
			\hline
			2     & b                 & I   & 6    & $\backslash$            \\
			\hline
			3     & b0                & I   & 9    & 2                       \\
			\hline
			4     & b00               & I   & 9    & 3                       \\
			\hline
			5     & a                 & E   & 1    & 1                       \\
			\hline
			6     & b00               & E   & 1    & 4                       \\
			\hline
			7     & a+b00             & E   & 2    & 5,6                     \\
			\hline
			8     & (a+b00)           & E   & 4    & 7                       \\
			\hline
			9     & a*(a+b00)         & E   & 3    & 5, 8                    \\
			\hline
		\end{tabular}
	\end{center}
\end{example}
definisco formalmente la derivazione $\to$:
\begin{definition}
	Prendo una grammatica $G=(V,T,P,S)$, grammatica CFG. Se $\alpha A \beta$ è una stringa tale che $\alpha,\beta\in (V\cup T)^*$, appartiene sia a variabili che terminali. Sia $A\in V$ e sia $a\to \gamma$ una produzione di $G$. Allora
	scriviamo:
	$$\alpha A \beta \to \alpha\gamma\beta$$
	con $\gamma\in (V\cup T)^*$.\\
	Le sostituzioni si fanno indipendentemente da $\alpha$ e $\beta$.
	Questa è quindi la definizione di derivazione.
\end{definition}
\begin{definition}
	Definisco il simbolo $\to _*$, ovvero il simbolo di \textit{derivazioni in 0 o più passi}. Può essere definito in modo ricorsivo. Per induzione sul numero di passi.
	\begin{itemize}
		\item la base dice che  $\forall \alpha\in (V\cup T)^*,\, \alpha\to * \,\alpha$
		\item il passo è: se $\alpha\to_{G_*} \,\beta $ e $ \beta \to_{G_*} \,\gamma$ allora $\alpha\to_* \,\gamma$
	\end{itemize}
	Si può anche dire che $\alpha\to_{G_*}\, \beta$ sse esiste una sequenza di stringhe $\gamma_1,...,\gamma_n$ con $n\geq 1$ tale che $\alpha=\gamma_1$, $\beta=\gamma_n$ e $\forall i,\, 1<i<n-1$ si ha che $\gamma_1\to \gamma_{i+1}$
	la derivazione in 0 o più passi è la chiusura transitiva della derivazione
\end{definition}
\begin{definition}
	avendo ora definito questi simboli possiamo definire una forma sentenziale. Infatti è una stringa $\alpha$ tale che:
	$$\forall \alpha\in (V\cup T)^* \mbox{ tale che }S\to_{G_*}\, \alpha$$
\end{definition}
\begin{definition}
	data $G=(V,T,P,S)$ si ha che $L(G)=\{w\in T^* |\, S\to_{G_*}\, w\}$ ovvero composto da stringhe terminali che sono derivabili o 0 o più passi.
\end{definition}
\begin{example}
	formare una grammatica CFG per il linguaggio:
	$$L=\{0^n 1^n| n\geq 1\}=\{01, 0011, 000111,...\}$$
	con $x^n$ intendo una concatenazione di $n$ volte $x$ (che nel nostro caso sono 0 e 1).\\
	posso scrivere:
	$$0^n 1^n =00^{n-1} 1^{n-1}1$$
	il nostro caso base sarà la stringa $01$, Poi si ha:
	$G=(V,T,P,S)$, $T=\{0,1\}$, $V=\{S\}$, il caso base $S\to 01$  e $S\to 0S1$
	il caso passo è quindi: se $w= 0^{n-1}1^{n-1}\in L$ allora $0w1\in L$.\\
	Ora voglio dimostare che $000111\in L$, ovvero $S\to*\, 000111$:\\
	$$S\to\, 0S1 \to 00S11\to 000S111$$
\end{example}
\begin{theorem}
	data la grammatica $G=\{V,T,P,S)$ CFG e $\alpha\in (V\cup T)^*$. Si ha che vale $S\to_*\, \alpha$ sse $S\to_{lm_*}\, \alpha$ sse $S\to_{rm_*}\, \alpha$. Con $\to_{lm_*}$ simbolo di \textit{left most derivation }e $\to_{rm_*}$ simbolo di \textit{right most derivation}
\end{theorem}
\begin{example}
	formare una grammatica CFG per il linguaggio:
	$$L=\{0^n 1^n| n\geq 0\}=\{\varepsilon, 01, 0011, 000111,...\}$$
	stavolta abbiamo anche la stringa vuota. Il caso base stavolta è $S\to\varepsilon| \, 0S1$
\end{example}
\begin{example}
	Fornisco una CFG per $L=\{a^n|n\geq 1\}=\{a, aa, aaa,...\}$.
	La base è $a$ \\il passo è che se $a^{n-1}\in L$ allora $a^{n-1}a\in L$ ( o che $aa^{n-1}\in L$).\\
	Si ha la grammatica $G=\{V,T,P,S)$, $V=\{S\}$, $T=\{a\}$ e si hanno $S\to a|\,Sa$ (o  $S\to a|\,aS$). Dimostro che $a^3\in L$.
	$$S\to Sa \to Saa\to aaa$$
	oppure
	$$S\to aS\to aaS\to aaa$$
\end{example}
\begin{example}
	trovo una CFG per $L=\{(ab)^n|n\geq 1\}=\{ab, abab, ababab,...\}$\\
	La base è $ab$ \\il passo è che se $(ab)^{n-1}\in L$ allora $(ab)^{n-1}ab\in L$.\\
	Si ha la grammatica $G=\{V,T,P,S)$, $V=\{S\}$, $T=\{a,b\}$ (anche se in realtà $T=\{ab\}$) e si hanno $S\to ab|\,Aab$. Poi dimostro come l'esempio sopra
\end{example}
\begin{example}
	trovo una CFG per $L=\{a^n c b^n|n\geq 1\}=acb,aacbb,aaacbbb,...\}$\\
	Il caso base è $acb$ il passo è che se $a^{n-1}cb^{n-1}\in L$ allora $a^{n-1}cb^{n-1}acb\in L$
	Si ha la grammatica $G=\{V,T,P,S)$, $V=\{S\}$, $T=\{a,b,c\}$ e si hanno $S\to aSb|acb$.\\
	dimostro che $aaaacbbbbb\in L$:
	$$S\to aSb\to aaSbb\to aaaaSbbb\to aaaacbbbb$$

	provo a usare anche una grammatica regolare, con le regole $S\to aS|c$, $c\to cB$ e $B\to bB|b$;
	$$S\to aS\to aaS\to aaC\to aacB\to aacb...$$
	non si può dimostrare in quanto non si può imporre una regola adatta
\end{example}
\begin{example}
	$L=\{a^n c b^{n-1}|n\geq 2\}$, con $a^n c b^{n-1}=a^{n-1}acb^{n-1}$. $S\to aSb|aacb$. Quindi:
	$$S\to aSb\to aaaccbb\in L$$
\end{example}
\begin{example}
	cerco CFG per $L=\{a^n c^k b^n|\,n,\,k>0\}$. $a$ e $b$ devono essere uguali, uso quindi una grammatica context free, mentre $c$ genera un linguaggio regolare.\\
	Si ha la grammatica $G=\{V,T,P,S)$, $V=\{S,C\}$, $T=\{a,b,c\}$ e si hanno $S\to aSb|aCb$ e $C\to cC|c$. dimostro che $aaaccbbb\in L, n=3,\, k=2$:
	$$S\to aSb \to aaSbb\to aaaCbbb\to aaacCbbb\to aaaccbbb$$
\end{example}
\begin{example}
	scrivere CFG per $L=\{a^nb^nc^kb^k|\, n,\,k\geq 0\}
	$
	$$=\{w\in\{a,b,c,d\}^*|\,a^nb^nc^kb^k|\, n,\,k\geq 0\}$$
	quindi L concatena due linguaggi $L1$ e $L2$, $X=\{a^nb^n\}$ e $Y=\{c^kd^k\}$:
	$$X\to aXb | \varepsilon$$
	$$Y\to cYd | \varepsilon$$
	$$S\to XY$$
	voglio derivare $abcd$:
	$$S\to XY \to XcYd\to aXbcYd\to aXbc\varepsilon d\to a\varepsilon bc\varepsilon d\to abcd$$
	voglio derivare $cd$
	$$S\to XY\to Y\to cYd\to cd$$
\end{example}
Quindi se ho $w\in L1, L2$, ovvero appartenente ad una concatenazione di linguaggi prima uso le regole di un linguaggio, poi dell'altro e infine ottengo il risultato finale.\\
\begin{example}
	scrivere CFG per $L=\{a^nb^kc^kd^n|\, n>0,\, k\geq 0\}
	$.
	$$S\to aSd|\, aXd$$
	$$X\to bXc| \varepsilon$$
	derivo $aabcdd$:
	$$S\to aSd\to aaXdd\to aabXcdd\to aabcdd$$
\end{example}
\begin{example}
	scrivere CFG per $L=\{a^ncb^nc^mad^m|\, n>0,\, m\geq 1\}
	$.
	$$S\to XY$$
	$$X\to aXb|c$$
	$$Y\to cUd| cad$$
	$$S\to XY\to cY\to ccad$$
\end{example}
\begin{example}
	scrivere CFG per $L=\{a^{n+m}xc^nyd^m|\, n,\, m\geq 0\}
	$. $a^{n+m}=a^na^m \mbox{ o } a^ma^n$. Si hanno 2 casi:
	\begin{enumerate}
		\item $L=\{a^na^m xc^nyd^m|\, n,\, m\geq 0\}
					$
		\item $L=\{a^ma^n xc^nyd^m|\, n,\, m\geq 0\}
					$
	\end{enumerate}
	ma solo  $L=\{a^ma^n xc^nyd^m|\, n,\, m\geq 0\}
	$ può generare una CFG (dove non si possono fare incroci, solo concatenazioni e inclusioni/innesti).
	$$S\to aSd| Y$$
	$$Y\to Xy$$
	$$X\to aXc|x$$
	si può fare in 2:
	$$S\to aSd| Xy$$
	$$X\to aXc|x$$
	derivo con $m=n=1$, $aaxcyd$:
	$$S\to aSd\to aXyd\to aaXcyd\to aaxcyd$$
\end{example}
\begin{example}
	scrivere CFG per $L=\{a^nb^m|\, n\geq m \geq 0\}
	$.$$L=\{\varepsilon, a, ab, aa, aab, aabb, aaa, aaab, aaabb, aaabbb,...\}$$
	Se $n\geq m$ allora $\exists k\geq 0 \to n=m+k$. Quindi:
	$$l=\{a^{m+k}b^m|m,k\geq0\}$$ si può scrivere in 2 modi:
	\begin{enumerate}
		\item $l=\{a^ma^kb^m|m,k\geq0\}$ quindi con innesto
		\item $l=\{a^ka^mb^m|m,k\geq0\}$quindi con concatenazione
	\end{enumerate}
	entrambi possibili per una CFG:
	\begin{enumerate}
		\item
					$$S\to XY$$
					$$X\to aX|\varepsilon \mbox{ si può anche scrivere } X\to Xa|\varepsilon$$
					$$Y\to aYb|\varepsilon$$
					oppure
					$$S\to aS|X$$
					$$X\to aXb| \varepsilon$$
		\item
					$$S\to aSb|\varepsilon$$
					$$X\to aX|\varepsilon$$
	\end{enumerate}
\end{example}
\begin{example}
	scrivere CFG per $L=\{a^nb^{m+n}c^h|\, m>h\geq0,\, n\geq0\}
	$.\\
	Se $n>h$ allora $\exists k \to n= h+k$, quindi:
	$$L=\{a^nb^{m+h+k}c^h|\, m>h\geq0,\, n\geq0\}$$. ovvero:
	$$L=\{a^nb^nb^kb^hc^h|\, m\geq 0, k>0, h\geq 0\}$$
	si ha:
	$$S\to XYZ$$
	$$X\to aXb|\varepsilon$$
	$$Y\to Yb|b$$
	$$Z\to bZc|\varepsilon$$
	si può anche fare:
	$$S\to XY$$
	$$X\to aXb|\varepsilon$$
	$$Y\to bYc|Z$$
	$$Z\to bZ|b$$
\end{example}
\begin{example}
	scrivere CFG per $L=\{a^nb^mc^k|\, k>n+m,\, n,m\geq 0\}
	$.\\
	per $n=m=0,\, k=1$ avrò la stringa $c$.
	se $k>n+m$ allora $\exists l>0\to k=n+m+l$ quindi:
	$$L=\{a^nb^mc^{n+m+l}|\, l>0,\, n,m\geq 0\}
	$$
	$$=L=\{a^nb^mc^nc^mc^l|\, l>0,\, n,m\geq 0\}$$
	sistemando:
	$$=L=\{a^nb^mc^lc^mcnl|\, l>0,\, n,m\geq 0\}$$
	quindi:
	$$S\to aSc|X$$
	$$X\to bXc|Y$$
	$$Y\to cY|c$$
\end{example}
\newpage
\begin{example}
	scrivere CFG per $L=\{a^nxc^{n+m}y^hz^kd^{m+h}|\, n,m,k,h\geq 0\}
	$.\\
	ovvero:
	$$L=\{a^nxc^nc^my^hz^kd^hd^m|\, n,m,k,h\geq 0\}$$
	quindi avrò:
	$$S\to XY$$
	$$X\to aXc|x$$
	$$Y\to cYd|W$$
	$$W\to yWd|X$$
	$$Z\to zZ|\varepsilon$$
\end{example}
\begin{example}
	vediamo un esempio di grammatica dipendente dal contesto:
	$$L=\{a^nb^nc^n|\, n\geq 1\}$$
	$G=\{V,T,P,S\}=\{(S,B,C,X)\}=\{(a,b,c),P,S\}$
	ecco le regole di produzione (qui posso scambiare variabili a differenza delle context free):
	\begin{enumerate}
		\item $S\to aSBC$
		\item $S\to aBC$
		\item $CB\to XB$
		\item $XB\to XC$
		\item $XC\to BC$
		\item $aB\to ab$
		\item $bB\to bb$
		\item $bC\to bc$
		\item $cC\to cc$
	\end{enumerate}
	vediamo un esempio di derivazione:
	per $n=1$ ho $abc$ ovvero:
	$$S\to aBC\to abC\to abc$$
	con $n=2$ ho $aabbcc$:
	$S\to aSBC\to aaBCBC\to aaBXBC\to aaBXCC\to aaBBCC\to aabBCC\to aabbCC\to aabbcC\to aabbcc$\\
	%vedere dimostrazione pag 14 soligo
\end{example}
\newpage
\begin{example}
	vediamo un esempio di grammatica dipendente dal contesto:
	$$L=\{a^nb^mc^nd^m|\, n,m\geq 1\}$$
	Si ha:
	$$G=(\{S,X,C,D,Z\},\{a,b,c,d\},P,S)$$
	con le seguenti regole di produzione:
	\begin{itemize}
		\item $S\to aSc|\, aXc$
		\item $X\to bXD|\, bD$
		\item $DC\to CD$
		\item $DC\to DZ$
		\item $DZ\to CZ$
		\item $XZ\to CD$
		\item $bC\to bc$
		\item $cC\to cc$
		\item $cD\to cd$
		\item $dD\to dd$
	\end{itemize}
	provo a derivare $aabbbccddd$ quindi con $n=2,\,m=3$:\\
	$$S\to aSC\to aaXCC\to aabXDCC\to aabbXDDCC\to $$
	$$aabbbDDDCC\to aabbbCCDDD\to aabbbccddd$$
\end{example}
\begin{example}
	Sia $L=\{w\in\{a,b\}^*|\, \mbox{ w contiene lo stesso numero di a e b}\}$:
	$$S\to aSbS|\,bSaS|\, \varepsilon$$
	dimostro per induzione che è corretto:
	\begin{itemize}
		\item \textbf{caso base:} $|w|=0\to w=\varepsilon$
		\item \textbf{caso passo:} si supponga che $G$ produca tutte le stringhe (di lunghezza $<$ di $n$) di $\{a,b\}^*$ con lo stesso numero di \textit{a} e \textit{b} e dimostro che produce anche quelle di lunghezza $n$, sia:
					$$w\in \{a,b\}^* \mid\, |w|=n \mbox{ con\textit{ a} e \textit{b} in egual numero, }m(a)=m(b) \mbox{ con m() che indica il numero di caratteri}$$
					quindi si ha che:
					$$w=aw_1bw_2\mbox{ o } w=bw_1aw_2$$
					sia.
					$$k_1=m(a)\in w_1=m(b)\in w_1$$
					$$k_2=m(a)\in w_2=m(b)\in w_2$$
					allora:
					$$k_1+k_2+1=m(a)\in w= m(b)\in W$$
					sapendo che $|w_1|<n$ e $|w_2|<n$ allora $w_1$ e $w_2$ sono egnerati da G per ipotesi induttiva
	\end{itemize}
\end{example}

\section{Alberi Sintatici}
\begin{definition}
	Data una grammatica CFG, $G=\{V,T,P,S\}$ un \textbf{albero sintattico} per $G$ soddisfa le seguenti condizioni:
	\begin{itemize}
		\item ogni nodo interno è etichettato con una variabile in $V$
		\item ogni foglia è anch'essa etichettata con una variabile o col simbolo di terminale T o con la stringa vuota $\varepsilon$ (in questo caso la foglia è l'unico figlio del padre)
		\item se un nodo interno è etichettato con A i suoi figli saranno etichettati con X1, ..., Xk e $A\to  X1, ..., Xk$ sarà una produzione di $G$ in $P$. Se un $X_i$ è $\varepsilon$ sarà l'unica figlio e $A\to \varepsilon$ sarà comunque una produzione di $G$
	\end{itemize}
	La concatenazione in ordine delle foglie viene detto \textbf{prodotto dell'albero}
\end{definition}
\newpage
\begin{example}
	Usiamo l'esempio delle stringhe palindrome:
	$$P\to 0P0|\,1P1|\varepsilon$$
	sia il seguente albero sintatico:
	\begin{center}
\includegraphics{example_1.1.1.png
}
%		\psframebox[linestyle=none,framesep=10pt]{%
%			\pstree{\LFTw{t}{\fontspec{Noto Sans}[Script=Latin]P}}{\pstree{\Tp[edge=none]}{%
%					\LFTw{t}{\fontspec{Noto Sans}[Script=Latin]0}
%					\pstree{\LFTw{t}{\fontspec{Noto Sans}[Script=Latin]P}}{\pstree{\Tp[edge=none]}{%
%							\LFTw{t}{\fontspec{Noto Sans}[Script=Latin]1}
%							\pstree{\LFTw{t}{\fontspec{Noto Sans}[Script=Latin]P}}{\pstree{\Tp[edge=none]}{%
%									\LFTw{t}{\fontspec{Noto Sans}[Script=Latin]$\varepsilon$}}}
%							\LFTw{t}{\fontspec{Noto Sans}[Script=Latin]1}}}
%					\LFTw{t}{\fontspec{Noto Sans}[Script=Latin]0}}}}
	\end{center}
\end{example}
\begin{example}
	Si ha:
	$$E\to I|\, E+E|\, E*E|\, (E)$$
	$$I\to a|\,b|\,Ia|\,Ib|\,I0|\,I1$$
	un albero sintattico per $a*(a+b00)$ può essere:
	\begin{center}
\includegraphics{example_1.1.2.png}
%		\psframebox[linestyle=none,framesep=10pt]{%
%			\pstree{\LFTw{t}{\fontspec{Noto Sans}[Script=Latin]E}}{\pstree{\Tp[edge=none]}{%
%					\pstree{\LFTw{t}{\fontspec{Noto Sans}[Script=Latin]E}}{\pstree{\Tp[edge=none]}{%
%							\pstree{\LFTw{t}{\fontspec{Noto Sans}[Script=Latin]I}}{\pstree{\Tp[edge=none]}{%
%									\LFTw{t}{\fontspec{Noto Sans}[Script=Latin]a}}}}}
%					\LFTw{t}{\fontspec{Noto Sans}[Script=Latin]*}
%					\pstree{\LFTw{t}{\fontspec{Noto Sans}[Script=Latin]E}}{\pstree{\Tp[edge=none]}{%
%							\LFTw{t}{\fontspec{Noto Sans}[Script=Latin](}
%								\pstree{\LFTw{t}{\fontspec{Noto Sans}[Script=Latin]E}}{\pstree{\Tp[edge=none]}{%
%										\pstree{\LFTw{t}{\fontspec{Noto Sans}[Script=Latin]E}}{\pstree{\Tp[edge=none]}{%
%												\pstree{\LFTw{t}{\fontspec{Noto Sans}[Script=Latin]I}}{\pstree{\Tp[edge=none]}{%
%														\LFTw{t}{\fontspec{Noto Sans}[Script=Latin]a}}}}}
%										\LFTw{t}{\fontspec{Noto Sans}[Script=Latin]+}
%										\pstree{\LFTw{t}{\fontspec{Noto Sans}[Script=Latin]E}}{\pstree{\Tp[edge=none]}{%
%												\pstree{\LFTw{t}{\fontspec{Noto Sans}[Script=Latin]I}}{\pstree{\Tp[edge=none]}{%
%														\pstree{\LFTw{t}{\fontspec{Noto Sans}[Script=Latin]I}}{\pstree{\Tp[edge=none]}{%
%																\LFTw{t}{\fontspec{Noto Sans}[Script=Latin]b}}}
%														\LFTw{t}{\fontspec{Noto Sans}[Script=Latin]0}}}
%												\LFTw{t}{\fontspec{Noto Sans}[Script=Latin]0}}}}}
%								\LFTw{t}{\fontspec{Noto Sans}[Script=Latin])}}}}}}
	\end{center}
\end{example}
\newpage
Data una CFG si ha che i seguenti cinque enunciati si equivalgono:
\begin{enumerate}
	\item la procedura di inferenza ricorsiva stailisce che una stringa $w$ di simboli terminali appartiene al linguaggio $L(A)$ con $A$ variabile
	\item $A\to ^*w$
	\item $A\to^*_{lm}w$
	\item $A\to^*_{rm}w$
	\item esiste un albero sintattico con radice $A$ e prodotto $w$
\end{enumerate}
queste 5 proposizioni si implicano l'uni l'altra:
\begin{center}
	\begin{tikzpicture}
		\node (top) at (0,0) {5};
		\node (a) at(-1,-0.5) {3};
		\node (b) at(0,-1) {4};
		\node (c) at(-2.0,-1.85) {2};
		\node (d) at(1.5,-2) {1};
		\draw [->] (top) -- (a);
		\draw [->] (top) -- (b);
		\draw [->] (a) -- (c);
		\draw [->] (b) -- (c);
		\draw [->] (c) -- (d);
		\draw [->] (d) -- (top);
	\end{tikzpicture}
\end{center}
vediamo qualche dimostrazione di implicazione tra queste proposizioni:
\begin{proof}[da 1 a 5]
	si procede per induzione:
	\begin{itemize}
		\item \textbf{caso base:} ho un livello solo (una sola riga), $\exists A\to w$:
					$$\overset{A}{\overset{\triangle}w}$$
		\item \textbf{caso passo:} suppongo vero per un numero di righe $\leq n$, lo dimsotro per $n+1$ righe:
					$$A\to X_1,X_2,...,X_k$$
					$$w=w_1,w_2,...,w_k$$
					ovvero, in meno di $n+1$ livelli:
					\begin{center}
\includegraphics{caso_passo.png}
					% \psframebox[linestyle=none,framesep=10pt]{%
%				      \pstree{\LFTw{t}{\fontspec{Noto Sans}[Script=Latin]A}}{\pstree{\Tp[edge=none]}{%
%						      \LFTw{t}{\fontspec{Noto Sans}[Script=Latin]$\overset{X_1}{\overset{\triangle}w_1}$}
%						      \LFTw{t}{\fontspec{Noto Sans}[Script=Latin]$\overset{X_2}{\overset{\triangle}w_2}$}
%						      \LFTw{t}{\fontspec{Noto Sans}[Script=Latin]$\vdots$}
%						      \LFTw{t}{\fontspec{Noto Sans}[Script=Latin]$\overset{X_k}{\overset{\triangle}w_k}$}}}}
					\end{center}
	\end{itemize}
\end{proof}
\begin{proof}[da 5 a 3]
	procedo per induzione:
	\begin{itemize}
		\item \textbf{caso base (n=1): }$\exists A\to w\mbox{ quindi } A\to_{lm}w$, come prima si ha un solo livello:
					$$\overset{A}{\overset{\triangle}w}$$
		\item \textbf{caso passo: }suppongo che la proprierà valga per ogni albero di profondità minore uguale a $n$, dimostro che valga per gli alberi profondi $n+1$:
					$$A\to X_1,X_2,...,X_k$$
					$$w=w_1,w_2,...,w_k$$
					ovvero, in meno di $n+1$ livelli:
					\begin{center}
\includegraphics{caso_passo.png}
%			      \psframebox[linestyle=none,framesep=10pt]{%
%				      \pstree{\LFTw{t}{\fontspec{Noto Sans}[Script=Latin]A}}{\pstree{\Tp[edge=none]}{%
%						      \LFTw{t}{\fontspec{Noto Sans}[Script=Latin]$\overset{X_1}{\overset{\triangle}w_1}$}
%						      \LFTw{t}{\fontspec{Noto Sans}[Script=Latin]$\overset{X_2}{\overset{\triangle}w_2}$}
%						      \LFTw{t}{\fontspec{Noto Sans}[Script=Latin]$\vdots$}
%						      \LFTw{t}{\fontspec{Noto Sans}[Script=Latin]$\overset{X_k}{\overset{\triangle}w_k}$}}}}
					\end{center}
					$$A\to_{lm} X_1,X_2,...,X_k$$
					$$x_1\to^*_{lm}w_1 \mbox{ per ipotesi induttiva si ha un albero al più di n livelli}$$
					quindi:
					$$A\to_{lm}X_1,...,X_k\to^*_{lm}w_1,X_2,...,X_k\to^*_{lm}...\to^*_{lm}w_1,...,w_k=w$$
	\end{itemize}
	\begin{example}
		$$E\to I\to Ib\to ab$$
		$$\alpha E\beta\to\alpha I\beta\to \alpha Ib\beta\to \alpha ab\beta,\,\,\,\alpha,\beta\in(V\cup T)^*$$
	\end{example}
\end{proof}
\begin{example}
	Mostro l'esistenza di una derivazione sinistra dell'albero sintattico di $a*(a+b00)$:
	$$E\to^*_{lm}E*E\to^*_{lm}I*E\to^*_{lm}a*E\to^*_{lm}a*(E)\to^*_{lm}a*(E+E)\to^*_{lm}$$
	$$a*(I+E)\to^*_{lm}a*(a+E)\to^*_{lm}a*(a+I)\to^*_{lm}a+(a+I0)\to^*_{lm}a*(a+I00)\to^*_{lm}a*(a+b00)$$
\end{example}
\section{Grammatiche ambigue}
\begin{definition}
	Una grammatica è definita ambigua se esiste una stringa $w$ di terminali che ha più di un albero sintattico
\end{definition}
\begin{example}
	vediamo un esempio:
	\begin{enumerate}
		\item $E\to E+E\to E+E*E$
					ovvero:
					\begin{center}
\includegraphics{ambigua1.png}
%			      \psframebox[linestyle=none,framesep=10pt]{%
%				      \pstree{\LFTw{t}{\fontspec{Noto Sans}[Script=Latin]E}}{\pstree{\Tp[edge=none]}{%
%						      \LFTw{t}{\fontspec{Noto Sans}[Script=Latin]E}
%						      \LFTw{t}{\fontspec{Noto Sans}[Script=Latin]+}
%						      \pstree{\LFTw{t}{\fontspec{Noto Sans}[Script=Latin]E}}{\pstree{\Tp[edge=none]}{%
%								      \LFTw{t}{\fontspec{Noto Sans}[Script=Latin]E}
%								      \LFTw{t}{\fontspec{Noto Sans}[Script=Latin]*}
%								      \LFTw{t}{\fontspec{Noto Sans}[Script=Latin]E}}}}}}
					\end{center}
		\item $E\to E*E\to E+E*E$
					ovvero:
					\begin{center}
%			      \psframebox[linestyle=none,framesep=10pt]{%
%				      \pstree{\LFTw{t}{\fontspec{Noto Sans}[Script=Latin]E}}{\pstree{\Tp[edge=none]}{%
%						      \pstree{\LFTw{t}{\fontspec{Noto Sans}[Script=Latin]E}}{\pstree{\Tp[edge=none]}{%
%								      \LFTw{t}{\fontspec{Noto Sans}[Script=Latin]E}
%								      \LFTw{t}{\fontspec{Noto Sans}[Script=Latin]+}
%								      \LFTw{t}{\fontspec{Noto Sans}[Script=Latin]E}}}
%						      \LFTw{t}{\fontspec{Noto Sans}[Script=Latin]*}
%						      \LFTw{t}{\fontspec{Noto Sans}[Script=Latin]E}}}}
					\end{center}
	\end{enumerate}
	si arriva a due stringhe uguali ma con alberi diversi. Introduciamo delle categorie sintatiche, dei vincoli alla produzione delle regole:
	\begin{enumerate}
		\item $E\to T|\, E+T$
		\item $T\to F|\, T+F$
		\item $F\to I|\, (E)$
		\item $I\to a|\,b|\,Ia|,Ib|\,I0|\,I1$
	\end{enumerate}
\end{example}
Possono esserci più derivazioni di una stringa ma l'importante è che non ci siano alberi sintattici diversi. Capire se una CFG è ambigua è un problema indecidibile
\begin{example}
	vediamo un esempio:
	$$S\to \varepsilon|\,SS|\, iS|\, iSeS$$
	con S=statement, i=if e e=else. Considero due derivazioni:
	\begin{enumerate}
		\item $S\to iSeS\to iiSeS\to iie$:
					\begin{center}
\includegraphics{ambigua2.png}

%			      \psframebox[linestyle=none,framesep=10pt]{%
%				      \pstree{\LFTw{t}{\fontspec{Noto Sans}[Script=Latin]S}}{\pstree{\Tp[edge=none]}{%
%						      \LFTw{t}{\fontspec{Noto Sans}[Script=Latin]i}
%						      \pstree{\LFTw{t}{\fontspec{Noto Sans}[Script=Latin]S}}{\pstree{\Tp[edge=none]}{%
%								      \LFTw{t}{\fontspec{Noto Sans}[Script=Latin]i}
%								      \pstree{\LFTw{t}{\fontspec{Noto Sans}[Script=Latin]S}}{\pstree{\Tp[edge=none]}{%
%										      \LFTw{t}{\fontspec{Noto Sans}[Script=Latin]$\varepsilon$}}}}}
%						      \LFTw{t}{\fontspec{Noto Sans}[Script=Latin]e}
%						      \pstree{\LFTw{t}{\fontspec{Noto Sans}[Script=Latin]S}}{\pstree{\Tp[edge=none]}{%
%								      \LFTw{t}{\fontspec{Noto Sans}[Script=Latin]$\varepsilon$}}}}}}\end{center}
%		\item $S\to iS\to iiSeS\to iieS\to iie$:
%		      \begin{center}
%
%			      \psframebox[linestyle=none,framesep=10pt]{%
%				      \pstree{\LFTw{t}{\fontspec{Noto Sans}[Script=Latin]S}}{\pstree{\Tp[edge=none]}{%
%						      \LFTw{t}{\fontspec{Noto Sans}[Script=Latin]i}
%						      \pstree{\LFTw{t}{\fontspec{Noto Sans}[Script=Latin]S}}{\pstree{\Tp[edge=none]}{%
%								      \LFTw{t}{\fontspec{Noto Sans}[Script=Latin]i}
%								      \pstree{\LFTw{t}{\fontspec{Noto Sans}[Script=Latin]S}}{\pstree{\Tp[edge=none]}{%
%										      \LFTw{t}{\fontspec{Noto Sans}[Script=Latin]$\varepsilon$}}}
%								      \LFTw{t}{\fontspec{Noto Sans}[Script=Latin]e}
%								      \pstree{\LFTw{t}{\fontspec{Noto Sans}[Script=Latin]S}}{\pstree{\Tp[edge=none]}{%
%										      \LFTw{t}{\fontspec{Noto Sans}[Script=Latin]$\varepsilon$}}}}}}}}
					\end{center}
	\end{enumerate}
	Si ha quindi una grammatica ambigua
\end{example}
\begin{theorem}
	Per ogni CFG, con $G=(V, T, P, S)$, per ogni stringa $w$ di terminali si ha che $w$ ha due alberi sintattici distinti sse ha due derivazioni sinistre da S distinte.\\
	Se la grammatica non è ambigua allora esiste un'unica derivazione sinistra da $S$
\end{theorem}
\subsubsection{Linguaggi inerentemente ambigui}
\begin{definition}
	Un linguaggio $L$ è inerentemente ambiguo se tutte le grammatiche CFG per tale linguaggio sono a loro volta ambigue
\end{definition}
\begin{example}
	Sia $L=\{a^nb^nc^md^m|\, n,m\geq 1\}\cup \{a^nbmnc^md^n|\, n,m\geq 1\}$\\
	si ha quindi un CFL formato dall'unione di due CFL. $L$ è inerentemente ambiguo e generato dalla seguente grammatica:
	\begin{itemize}
		\item $S\to AB|\,C$
		\item $A\to aAb|\,ab$
		\item $B\to cBd|\, cd$
		\item $C\to aCd|\, aDd$
		\item $D\to bDc|\, bc$
	\end{itemize}
	si possono avere due derivazioni:
	\begin{enumerate}
		\item $S\to_{lm}AB\to_{lm} aAbB\to_{lm} aabbB\to_{lm}aabbcBd\to_{lm}aabbccdd$
		\item $S\to_{lm} C\to_{lm} aCd\to_{lm}aaBdd\to_{lm}aabBcdd\to_{lm}aabbccdd$
	\end{enumerate}
	a generare problemi sono le stringhe con n=m perché possono essere prodotte in due modi diversi da entrambi i sottolinguaggi. Dato che l'intersezione tra i due sottolinguaggi non è buota si ha che $L$ è ambiguo
\end{example}
\section{Grammatiche Regolari}
Sono le grammatiche che generano i linguaggi regolari (quelli del terzo tipo) che sono casi particolari dei CFL.\\
Si ha la solita grammatica $G = (V, T, P, S)$ con però vincoli su $P$:
\begin{itemize}
	\item $\varepsilon$ si può ottenere solo con $S\to \varepsilon$
	\item le produzioni sono tutte lineari a destra ($A\to aA$ o $A\to a$) o a sinistra ($A\to Ba$ o $A\to a$)
\end{itemize}
\begin{example}
	$I\to a|\,b|\,Ia|\,Ib|\,I0|\,I1$ è una grammatica con le produzioni lineari a sinistra.\\
	Potremmo pensarlo a destra $I\to a|\,b|\,aI|\,bI|\,0I|\,1I$.\\
	Vediamo esempi di produzione con queste grammatiche:
	\begin{itemize}
		\item con $I\to a|\,b|\,Ia|\,Ib|\,I0|\,I1$ possiamo derivare $ab01b0$:
					$$I\to I0\to Ib0\to I1b0\to I01b0\to Ib01b0\to ab01b0$$
		\item con $I\to a|\,b|\,aI|\,bI|\,0I|\,1I$ invece non riusciamo a generare nulla:
					$$I\to 0I\to 0a$$
	\end{itemize}
	definisco quindi un'altra grammatica (con una nuova categoria sintattica):
	$$I\to aJ|\, bJ$$
	$$J\to a|\,b|\,aJ|\,bJ|\,0J|\,1J$$
	che però non mi permette di terminare le stringhe con 0 e 1, la modifico ancora otterdendo:
	$$I\to aJ|\, bJ$$
	$$J\to a|\,b|\,aJ|\,bJ|\,0J|\,1J|\,0|\,1$$
	e questo è il modo corretto per passare da lineare sinistra a lineare destra
\end{example}
\begin{example}
	Sia $G=(\{S\},\{0,1\},P,S)$ con $S\to \varepsilon|\,0|\,1|\,0S|\,1S$. Si ha quindi:
	$$L(G)=\{0,1\}^*$$
	si hanno comunque due proposizioni ridondanti, riducendo trovo:
	$$S\to \varepsilon|\, 0S|\,1S$$
	con solo produzioni lineari a destra. Con produzioni lineari a sinistra ottengo:
	$$S\to \varepsilon|\, S0|\,S1$$
\end{example}
\begin{example}
	Trovo una grammatica lineare destra e una sinistra per $L=\{a^nb^m|\,n,m\geq 0\}$:
	\begin{itemize}
		\item \textbf{lineare a destra:} si ha $G=(\{S,B\},\{a,b\},P,S)$ e quindi:
					$$S\to \varepsilon|\,aS|\,bB$$
					$$B\to bB|\,b$$
					ma non si possono generare stringhe di sole $b$, infatti:
					$$S\to aS\to abB\to abbB\to abbb$$
					ma aggiungere $\varepsilon$ a B \textbf{non è lecito}. posso però produrre la stessa stringa da due derivazioni diverse:
					$$S\to \varepsilon|\,aS|\,bB|\,b$$
					$$B\to bB|\,b$$
					che risulta quindi la nostra lineare a destra
		\item \textbf{lineare a sinistra:} si ha $G=(\{S,A\},\{a,b\},P,S)$ e quindi:
					$$S\to \varepsilon|\,Sb|\,Ab|\,a$$
					$$A\to Aa|\,a$$
	\end{itemize}
\end{example}
\begin{example}
	Trovo una grammatica lineare destra e una sinistra per $L=\{ab^ncd^me|\,n\geq 0\,,m> 0\}$:
	\begin{itemize}
		\item \textbf{lineare a destra:} si ha  si ha $G=(\{S,A,B,E\},\{a,b,c,d,e\},P,S)$ e quindi:
					$$S\to aA$$
					$$A\to bA|\,cB$$
					$$B\to dB|\, dE$$
					$$E\to e$$
		\item \textbf{lineare a sinistra:} si ha  si ha $G=(\{S,X,Y,Z\},\{a,b,c,d,e\},P,S)$ e quindi:
					$$S\to Xe$$
					$$A\to Xd|\,Yd$$
					$$B\to Zc$$
					$$E\to a|\,Zb$$
	\end{itemize}
	quindi se per esempio ho la stringa "ciao" si ha:
	\begin{itemize}
		\item \textbf{lineare a destra:}
					$$S\to Ao$$
					$$A\to Ba$$
					$$B\to Ei$$
					$$E\to c$$
		\item \textbf{lineare a sinistra:}
					$$S\to cA$$
					$$A\to iB$$
					$$B\to aE$$
					$$E\to o$$
	\end{itemize}
\end{example}
\begin{example}
	A partire da $G=(\{S,T\},\{0,1\},P,S)$ con:
	$$S\to\varepsilon|\,0S|\,1T$$
	$$T\to 0T|\,1S$$

	trovo come è fatto $L(G)$:
	$$L(G)=\{w\in\{0,1\}^*|\, w \mbox{ ha un numero di 1 pari}\}$$
\end{example}
\begin{example}
	fornire una grammatica regolare a destra e sinistra per:
	$$L=\{w\in\{0,1\}^*|\, w \mbox{ ha almeno uno 0 o almeno un 1}\}$$
	Si ah che tutte le stringhe tranne quella vuota ciontengono uno 0 o un 1
	quindi  $G=(\{S\},\{0,1\},P,S)$:
	\begin{itemize}
		\item \textbf{lineare a destra:}
					$$S\to 0|\,1|\,0S|\,1S$$
		\item \textbf{lineare a sinistra:}
					$$S\to 0|\,1|\,S0|\,S1$$
	\end{itemize}
\end{example}



Vediamo ora un esempio di \textit{Context Free Language (CFL)}, costruito a partire da una \textit{Context Free Grammar (CFG)}:
\begin{example}
	Sia $\Sigma=\{0, 1\}$ e $L_{pal}="stringhe\,\, palindrome\,\, binarie"$.
	Quindi, per esempio, $0110\in L,\,\, 11011\in L$ ma $10010\not\in L$. Si ha che $\varepsilon$, la stringa vuota, appartiene a $L$. Diamo una definizione ricorsiva:
	\begin{itemize}
		\item \textbf{base:} $\varepsilon,\, 0\,\ 1\in L_{pal}$
		\item \textbf{passo:} se $w$ è palindroma allora $0w0$ è palindromo e $1w1$ è palindromo
	\end{itemize}
	una variabile generica $S$ può sottostare alle \textit{regole di produzione} di una certa grammatica. In questo caso si ha uno dei seguenti:
	$$S\to\varepsilon,\, S\to 0,\, S\to 1,\, S\to 0S0,\, S\to 1S1$$
\end{example}


riprendiamo l'esempio sopra:
\begin{example}
	$$G_{pal}=(V=\{S\},\, T=\{0, 1\},\, P,\, S)$$
	con:
	$$P=\{S\to\varepsilon,\, S\to 0,\, S\to 1,\, S\to 0S0,\, S\to 1S1\}$$
	Si può ora costruire un algoritmo per creare una stringa palindroma a partire dalla grammatica $G$:
	$$\underbrace{S}_{\mbox{start}}\underbrace{\to}_{\mbox{applico una regola}} 1S1 \to 01S10\to \underbrace{01010}_{\mbox{sostituisco variabile}}$$

	con $S,\, 1S1\,\, e\,\, 01S10$ che sono \textit{forme sentenziali}. Posso così ottenere tutte le possibili stringhe. Esiste anche una forma abbreviata:
	$$S\to \varepsilon|o|1|0S0|1S1$$
	Non si fanno sostituzioni in parallelo, prima una $S$ e poi un'altra
\end{example}


\section{Automi a Pila}\label{sec:automi-pila}

	Si introduce un nuovo tipo di automa, il PDA (push down automata) che può essere pensato come un $\varepsilon-NFA$ col supporto di una pila (stack):
	\begin{center}
	\includegraphics{2025-01-16_14-59.png}
		% \psscalebox{1.0 1.0} % Change this value to rescale the drawing.
		% {
		% 	\begin{pspicture}(0,-2.04)(5.83,2.04)
		% 		\psframe[linecolor=black, linewidth=0.04, dimen=outer](3.2,2.04)(1.58,1.16)
		% 		\psline[linecolor=black, linewidth=0.04, arrowsize=0.05291667cm 2.0,arrowlength=1.4,arrowinset=0.0]{->}(0.0,1.58)(1.6,1.58)
		% 		\psline[linecolor=black, linewidth=0.04, arrowsize=0.05291667cm 2.0,arrowlength=1.4,arrowinset=0.0]{->}(3.2,1.58)(4.8,1.58)
		% 		\psline[linecolor=black, linewidth=0.04, arrowsize=0.05291667cm 2.0,arrowlength=1.4,arrowinset=0.0]{->}(2.4,0.38)(2.4,-0.42)
		% 		\psline[linecolor=black, linewidth=0.04, arrowsize=0.05291667cm 2.0,arrowlength=1.4,arrowinset=0.0]{->}(2.4,0.38)(2.4,1.18)
		% 		\psline[linecolor=black, linewidth=0.04](1.6,-0.42)(1.6,-2.02)
		% 		\psline[linecolor=black, linewidth=0.04](3.2,-2.02)(1.6,-2.02)
		% 		\psline[linecolor=black, linewidth=0.04](3.2,-2.02)(3.2,-0.42)
		% 		\psline[linecolor=black, linewidth=0.04](1.6,-0.82)(3.2,-0.82)
		% 		\psline[linecolor=black, linewidth=0.04](1.6,-1.22)(3.2,-1.22)
		% 		\psline[linecolor=black, linewidth=0.04](1.6,-1.62)(3.2,-1.62)
		% 		\rput[bl](4.9,1.32){$\frac{accetta}{rifiuta}$}
		% 		\rput[bl](0.4,1.74){input}
		% 		\rput[bl](3.78,0.22){ stati finiti}
		% 		\rput[bl](3.6,-1.22){stack}
		% 		\psline[linecolor=black, linewidth=0.04, arrowsize=0.05291667cm 2.0,arrowlength=1.4,arrowinset=0.0]{->}(3.64,0.72)(2.38,1.62)
		% 		\rput[bl](3.78,0.56){controllo a}
		% 	\end{pspicture}
		% }
	Tikz
	\end{center}
	e viene definito un PDA $P$ come:
	$$P=(Q,\Sigma,\Gamma,\delta,q_0,z_0,F)$$
	con;
	\begin{itemize}
		\item $Q$: insieme finito e non vuoto di stati
		\item $\Sigma$: alfabeto di simboli di input
		\item $\Gamma$: alfabeto di simboli di stack
		\item $q_0\in Q$: stato iniziale
		\item $z_0\in \Gamma\backslash \Sigma$: simbolo iniziale dello stack
		\item $F\in Q$: insieme degli stati accettanti o finali
	\end{itemize}
	si ha che:
	$$\delta:Q\times(\Sigma\cup\{\varepsilon\}\times \Gamma\to 2^{Q\times \Gamma^*}$$
	quindi:
	$$\delta(q_0,a,X)=\{(p_1,X_1),(p_2,X_2),...\}\,insieme\,\,\,finito\,\,p_i\in Q\,\,X_i\in \Gamma^*$$
	si hanno dei casi particolari:
	\begin{itemize}
		\item lo stato $p$ potrebbe coincidere con $Q$ e si avrebbe un cappio
		\item se $\Gamma=\varepsilon$ si ha il pop di $X$ dallo stack
		\item se $\Gamma=X$ si lascia lo stack invariato
		\item se $\Gamma=Y\neq X$ si ha la sostituzione di $X$ con $Y$ in cima allo stack
		\item se $\Gamma$ è una stringa di simboli si ha il la rimozione di $X$ dallo stack e l'aggiunta a uno a uno dei simboli nello stack
	\end{itemize}
	\newpage
	\begin{example}
		Trovo PDA per il linguaggio delle stringhe binarie  palindrome di lunghezza pari: $L=\{ww^R|w\in\{0,1\}^*\}$. Con $R$ che indica rovesciato\\
		Si ha la CFG $G=(\{P\},\{0,1\},Prod,P)$ tale che:
		$$P\to 0P0|1P1|\varepsilon$$
		si hanno quindi tre stati:
		\begin{itemize}
			\item $q_0$ che è quello iniziale che legge $w$ e spinge i dati sullo stack
			\item $q_1$ che letta $w$ legge i simboli di $wR$ e li confronta con quelli dello stack
			\item $q_2$ sarà la stringa accettata
		\end{itemize}
		descriviamo formalmente l'automa con la funzione di transizione $\delta$.
		PDA $P=(\{q_0,q_1,q_2\},\{0,1\}, \{o,1,z_0\},\delta,q_0,z_0,\{q_2\})$
		ovvero:
		$$\delta(q_0,0,z_0)=\{(q_0,1z_0)\}$$
		$$\delta(q_0,1,z_0)=\{(q_0,0z_0)\}$$
		$$\delta(q_0,0,0)=\{(q_0,00)\}$$
		$$\delta(q_0,0,1)=\{(q_0,01)\}$$
		$$\delta(q_0,1,0)=\{(q_0,10)\}$$
		$$\delta(q_0,1,1)=\{(q_0,11)\}$$
		$$\delta(q_0,\varepsilon,z_0)=\{(q_0,z_0)\}$$
		$$\delta(q_0,\varepsilon,0)=\{(q_0,0)\}$$
		$$\delta(q_0,\varepsilon,1)=\{(q_0,1)\}$$
		$$\delta(q_1,0,0)=\{(q_1,\varepsilon)\}$$
		$$\delta(q_2,1,1)=\{(q_1,\varepsilon)\}$$
		$$\delta(q_2,\varepsilon,z_0)=\{(q_2,0z_0)\}$$
		\newpage
		otteniamo il seguente PDA:
		\begin{center}
			\begin{tikzpicture}[shorten >=1pt,node distance=3cm,on grid,auto]
				\node[state, initial] (q_0) {$q_0$};
				\node[state] (q_1) [right=of q_0] {$q_1$};
				\node[state, accepting] (q_2) [right =of q_1] {$q_2$};
				\path[->]
				(q_0) edge node [align=center] {$\varepsilon,0/0$\\$\varepsilon,1/1$\\$\varepsilon,z_0/z_0$} (q_1)
				edge [loop below] node [align=center] {$0,z_0/0z_0$\\$1,z_0/1z_0$\\$0,0/00$\\$0,1/01$\\$1,0/10$\\$1,1/11$} ()
				(q_1) edge [loop below] node [align=center] {$0,0/\varepsilon$\\$1,1/\varepsilon$} ()
				edge  node [align=center] {$\varepsilon,z_0/z_0$} (q_2);
			\end{tikzpicture}
		\end{center}
		e si definisce questa notazione per gli archi:
		$$(p,\alpha)\in\delta(q,a,X)$$
		\begin{center}
			\begin{tikzpicture}[shorten >=1pt,node distance=3cm,on grid,auto]
				\node[state, initial] (q_0) {$q_0$};
				\node[state] (q_1) [right=of q_0] {$q_1$};
				\path[->]
				(q_0) edge node [align=center] {$a,X/\alpha$} (q_1);
			\end{tikzpicture}
		\end{center}
	\end{example}
	analizziamo meglio i PDA. Si ha che la \textbf{descrizione istantanea (ID)} di un PDA è una tripla:
	$$ID:(q,w,\gamma)$$
	con $q\in Q$ stato attuale $w\in\Sigma^*$ input rimanente e $\gamma\in\Gamma^*$ contenuto attuale dello stack.\\
	Definiamo ora il concetto di \textbf{mossa in un passo}
	dato $P=(Q,\Sigma,\Gamma,\delta,q_0,z_0,F)$ la mossa è una relazione $\vdash_p$:
	$$(p,\alpha)\in\delta(q,a,X)\,\,allora\,\,\forall w\in\Sigma^*\,\,e\,\, \forall\beta\in\Gamma^*\to (q,aw,X\beta)\vdash(p,w,\alpha\beta$$
	e
	$$(p,\alpha)\in\delta(q,\varepsilon,X)\,\,allora\,\,\forall w\in\Sigma^*\,\,e\,\, \forall\beta\in\Gamma^*\to (q,w,X\beta)\vdash(p,w,\alpha\beta)$$
	\newpage
	ora possiamo anche definire la relazione con 0 o più mosse
	in forma induttiva $\stackrel{*}{\vdash_p}$:
	\begin{itemize}
		\item \textbf{caso base:} $\forall ID\,\,I, I \stackrel{*}{\vdash} I$
		\item \textbf{caso passo:} $I \stackrel{*}{\vdash} J$ se $\exists ID\,\,K$ tale che $ I\vdash K \,\,e\,\, K \stackrel{*}{\vdash} J$
	\end{itemize}
	vediamo un esempio con un PDA che accetta $ww^R|\,w\in\{0,1\}^*$:
	\begin{center}
		\begin{tikzpicture}[shorten >=1pt,node distance=3cm,on grid,auto]
			\node[state, initial] (q_0) {$q_0$};
			\node[state] (q_1) [right=of q_0] {$q_1$};
			\node[state, accepting] (q_2) [right =of q_1] {$q_2$};
			\path[->]
			(q_0) edge node [align=center] {$\varepsilon,0/0$\\$\varepsilon,1/1$\\$\varepsilon,z_0/z_0$} (q_1)
			edge [loop below] node [align=center] {$0,z_0/0z_0$\\$1,z_0/1z_0$\\$0,0/00$\\$0,1/01$\\$1,0/10$\\$1,1/11$} ()
			(q_1) edge [loop below] node [align=center] {$0,0/\varepsilon$\\$1,1/\varepsilon$} ()
			edge  node [align=center] {$\varepsilon,z_0/z_0$} (q_2);
		\end{tikzpicture}
	\end{center}
	e prendiamo la stringa $1111$:\\
	\includegraphics{2025-01-16_15-01.png}
	% \\
	% \psscalebox{1.0 1.0} % Change this value to rescale the drawing.
	% {
	% 	\begin{pspicture}(0,-2.51)(3.08,2.51)
	% 		\rput[bl](0.0,2.29){$q_0,1111,z_0$}
	% 		\rput[bl](0.0,1.49){$q_0,111,1z_0$}
	% 		\rput[bl](0.0,0.69){$q_0,11,11z_0$}
	% 		\rput[bl](0.0,-0.11){$q_0,1,111z_0$}
	% 		\rput[bl](0.0,-0.91){$q_0,\varepsilon,1111z_0$}
	% 		\rput[bl](0.0,-1.71){$q_1,\varepsilon,1111z_0$}
	% 		\rput[bl](0.25,-2.51){$X$}
	% 		\rput[bl](3.0,1.49){$(q_1,1111,z_0)\to(q_2,1111,z_0)\to X$}
	% 		\rput[bl](3.0,0.69){$(q_1,111,1z_0)\to (q_1,11,z_0)\to(q_2,11,z_0)\to X$}
	% 		\rput[bl](3.0,-0.11){$(q_1,11,11z_0)\to(q_1,1,1z_0)\to(q_1,\varepsilon,z_0)\to(q_2,\varepsilon,z_0)\to V$}
	% 		\rput[bl](3.0,-0.91){$(q_1,1,111z_0)\to(q_1,\varepsilon,11z_0)\to X$}
	% 		\psline[linecolor=black, linewidth=0.04, arrowsize=0.05291667cm 2.0,arrowlength=1.4,arrowinset=0.0]{->}(0.4,2.29)(0.4,1.89)
	% 		\psline[linecolor=black, linewidth=0.04, arrowsize=0.05291667cm 2.0,arrowlength=1.4,arrowinset=0.0]{->}(0.4,1.49)(0.4,1.09)
	% 		\psline[linecolor=black, linewidth=0.04, arrowsize=0.05291667cm 2.0,arrowlength=1.4,arrowinset=0.0]{->}(0.4,0.69)(0.4,0.29)
	% 		\psline[linecolor=black, linewidth=0.04, arrowsize=0.05291667cm 2.0,arrowlength=1.4,arrowinset=0.0]{->}(0.4,-0.11)(0.4,-0.51)
	% 		\psline[linecolor=black, linewidth=0.04, arrowsize=0.05291667cm 2.0,arrowlength=1.4,arrowinset=0.0]{->}(0.4,-0.91)(0.4,-1.31)
	% 		\psline[linecolor=black, linewidth=0.04, arrowsize=0.05291667cm 2.0,arrowlength=1.4,arrowinset=0.0]{->}(0.4,-1.71)(0.4,-2.11)
	% 		\psline[linecolor=black, linewidth=0.04, arrowsize=0.05291667cm 2.0,arrowlength=1.4,arrowinset=0.0]{->}(0.8,1.49)(2.8,1.09)
	% 		\psline[linecolor=black, linewidth=0.04, arrowsize=0.05291667cm 2.0,arrowlength=1.4,arrowinset=0.0]{->}(0.8,2.29)(2.8,1.89)
	% 		\psline[linecolor=black, linewidth=0.04, arrowsize=0.05291667cm 2.0,arrowlength=1.4,arrowinset=0.0]{->}(0.8,0.69)(2.8,0.29)
	% 		\psline[linecolor=black, linewidth=0.04, arrowsize=0.05291667cm 2.0,arrowlength=1.4,arrowinset=0.0]{->}(0.8,-0.11)(2.8,-0.51)
	% 	\end{pspicture}
	% }
	\\
	chiamiamo \textbf{computazione }una sequenza di mosse, non necessariamente di successo. Si hanno alcune proprietà:
	\begin{itemize}
		\item se una se una sequenza di ID è lecita per un PDA P allora è lecita anche la sequenza di Id ottenuta concatenando $w\in\Sigma^*$ in ogni ID
		\item se una se una sequenza di ID è lecita per un PDA P e resta una coda di input non consumata allora posso rimuovere tale coda in ogni ID e ottenere un'altra sequenza lecita
		\item se una se una sequenza di ID è lecita per un PDA P allora  è lecita la sequenza ottenuta aggiungendo $\gamma\in\Gamma^*$ in coda alla terza sequenza di ogni ID
	\end{itemize}
	del resto però:
	$$(q,Xw,\alpha\gamma) \stackrel{*}{\vdash_p} (p,Yw,\beta\gamma)\not\to (q,X,\alpha) \stackrel{*}{\vdash}(p,y,\beta),\,\,x,w,y\in\Sigma^*\,\,\alpha,\beta,\gamma\in\Gamma^*$$
	per queste proprietà valgono i seguenti teoremi:
	\begin{theorem}
		per la seconda:
		Se $P=(Q,\Sigma,\Gamma,\delta,q_0,z_0,F)$ è un PDA e $(q,Xw,\alpha) \stackrel{*}{\vdash_p} (p,Yw,\beta)$ allora vale anche:
		$$(q,X,\alpha) \stackrel{*}{\vdash_p} (p,Y,\beta)$$
	\end{theorem}
	\begin{theorem}
		per la prima e la terza:
		Se $P=(Q,\Sigma,\Gamma,\delta,q_0,z_0,F)$ è un PDA e $(q,X,\alpha) \vdash_p^* (p,Y,\beta)$ allora:
		$$\forall\gamma\in\Gamma^*\,\,\,vale\,\,\,anche\,\,\,(q,Xw,\alpha\gamma) \vdash_p^* (p,Yw,\beta\gamma)$$
	\end{theorem}
	Si definiscono due modalità di accettazione per i PDA:
	\begin{enumerate}
		\item \textbf{per stato finale:} sia $P=(Q,\Sigma,\Gamma,\delta,q_0,z_0,F)$ si ha che:
		      $$L(P)=\{w\in\Sigma^*|\,(q_0,w,z_0) \vdash_p^* (q,\varepsilon,\alpha)\}$$
		      con $q\in F$ e $\forall \alpha\in \Gamma^*$
		\item \textbf{per stack vuoto:} sia $P=(Q,\Sigma,\Gamma,\delta,q_0,z_0,F)$ si ha che:
		      $$N(P)=\{w\in\Sigma^*|\,(q_0,w,z_0) \vdash_p^* (q,\varepsilon,\varepsilon)\}$$
		      con $q\in Q$ e in questo caso l'insieme degli stati finali $F$ non ha alcuna influenza
	\end{enumerate}
	In realtà si ha che la classe di linguaggi accettati dai PDA per stato finale è uguale a quella per stack vuoto, anche se passare da un tipo all'altro di PDA è complesso. SI ha il seguente teorema per la trasformazione:
	\begin{theorem}
		se $L=N(P_N)$ per un PDA  $P_N=(Q,\Sigma,\Gamma,\delta,q_0,z_0,F)$ allora $\exists \,\,PDA\,\,P_F\,\,\,tale\,\,\,che\,\,\,L=L(P_F)$
	\end{theorem}
	\begin{proof}
		Sia $x_0\in\Gamma$, che indica la fine dello stack di $P_F$. Si ha:
		$$\delta(p_0,\varepsilon,x_0)=\{(q_0,z_0x_0\}$$
		e:
		$$\forall q\in Q,\,\forall a\in\Sigma\cup\{\varepsilon\},,\forall y\in\Sigma:\,\,\delta_F(q,a,y)\mbox{ contiene tutte le coppie di }\delta_N(q,a,y)$$
		$$\forall q\in Q,\delta_F(q,\varepsilon,x_0)=\{(P_F,\varepsilon)\}$$
		quindi graficamente:

		\includegraphics{2025-01-16_15-03.png}
		Bisogna dimostrare che effettivamente $w\in L(P_F)\longleftrightarrow\in N(P_N)$.\\
		se $w\in N(P_N)$ $\exists$ una sequenza di $ID\,\,\,(q_0,w,z_0)\vdash_{P_N}^*(q,\varepsilon,\varepsilon)$ per un qualche $q\in Q$:
		$$(q_0,w,z_0x_0)\vdash_{P_N}^*(q,\varepsilon,x_0)$$
		inoltre:
		$$(q_0,w,z_0x_0)\vdash_{P_F}^*(q,\varepsilon,x_0)$$
		e quindi:
		$$(p_0,w,x_0)\vdash_{P_F}(q_0,w,z_0x_0)\vdash_{P_F}^*(q,\varepsilon,x_0)\vdash_{P_F}(P_F,\varepsilon,\varepsilon)$$
		solo se togliendo il primo e l'ultimo passo di $P_F$ ripercorro all'indietro quanto scritto sopra.
	\end{proof}
	\begin{example}
		trasformazione da accettante per stack vuoto a accettante per stato finale. Siano:
		$$\Sigma=\{i,e\}$$
		$$P_n=(\{q\}, \{i,e\},\{Z\},\delta_M,q,Z)$$
		$$\delta_N(q,i,Z)=\{(q,ZZ)\}$$
		$$\delta_N(q,e,Z)=\{(q,\varepsilon)\}$$
		quindi:
		\begin{center}
			\begin{tikzpicture}[shorten >=1pt,node distance=3cm,on grid,auto]
				\node[state, initial] (q_0) {$q$};
				\path[->]
				(q_0) edge [loop above] node [align=center] {$i,Z/ZZ$\\$e,Z/\varepsilon$} ();
			\end{tikzpicture}
		\end{center}
		quindi inseriamo una $Z$ quando leggiamo $i$ e ne rimuoviamo una se leggiamo $e$ e si parte con una $Z$ nello stack.\\
		Costruisco ora il PDA $P_F$ che accetta lo stesso linguaggio ma per stato finale, introduco lo stato iniziale $p$ e quello  accettante $r$, uso $x_0$ come segnale della fine dello stack:
		\begin{center}
			\begin{tikzpicture}[shorten >=1pt,node distance=3cm,on grid,auto]
				\node[state, initial] (q_0) {$p$};
				\node[state] (q_1) [right=of q_0] {$q$};
				\node[state, accepting] (q_2) [right =of q_1] {$r$} ;
				\path[->]
				(q_0) edge  node [align=center] {$\varepsilon x_0/zx_0$} (q_1)
				(q_1) edge node {$\varepsilon,x_0/\varepsilon$} (q_2)
				edge [loop above] node [align=center] {$i,Z/ZZ$\\$e,Z/\varepsilon$} ();
			\end{tikzpicture}
		\end{center}
		e si ha formalmente:
		$$P_F=(\{p,q,r\}, \{i,e\},\{Z,x_0\},\delta_F,p,x_0,\{r\})$$
		con $\delta_F$ che rappresenta le seguenti quattro regole:
		\begin{enumerate}
			\item $\delta_F(p,\varepsilon,x_0)=\{(q,Zx_0)\}$ regola che fa partire $P_F$ con $x_o$ come segnalatore dello stack
			\item $\delta_F(p,i,Z)=\{(q,ZZ)\}$ regola che inserisce $Z$ quando si ha $i$ simulando $P_N$
			\item $\delta_F(p,e,Z)=\{(q,Z\varepsilon)\}$ regola che rimuove $Z$ quando si ha $e$ simulando $P_N$
			\item $\delta_F(p,e,x_0)=\{(r,\varepsilon)\}$ regola che permette a $P_F$ di accettare quando $P_N$ esaurisce lo stack
		\end{enumerate}
	\end{example}
	Si può anche effettuare la trasformazione inversa:
	\begin{theorem}
		Sia $P_F=(Q,\Sigma,\Gamma,\delta_f,q_0,Z_0,F)$.\\
		Si aggiunge una transizione $\varepsilon$  a un nuovo stato $p$ da ogni accettante di $P_F$. quando si ha $p$ $P_N$ svuota lo stack senza consumare input. Quindi se $P:F$ entra in uno stato accettante dopo aver consumato l'input $w$, $P_N$ svuota lo stack dopo aver consumato $w$. Per evitare che si svuoti lo stack per una stringa non accettata uso $x_0$ per indicare il fondo dello stack. Il nuovo $P_N$ parte da $p_0$ che ha il solo scopo di inserire il simbolo iniziale di $P_F$ e passare al suo stato iniziale. Si ottiene quindi:
		\begin{center}
		\includegraphics{2025-01-16_15-04.png}
		\end{center}
		e si ha formalmente:
		$$P_F=(Q\cup\{p_0,p\}, \Sigma,\Gamma\cup\{x_0\},\delta_N,p_0,x_0)$$
		dove $\delta_N$ è così definita:
		\begin{enumerate}
			\item $\delta_N(p_0,\varepsilon,x_0)=\{(q_0,Z_0x_0)\}$ inserisce il simbolo iniziale di $P_F$ nello stack e va allo stato iniziale di $P_F$
			\item $\forall q\in Q$ ogni simbolo di input $a\in\Sigma$, compreso l'input vuoto, e $\forall y\in \Gamma$, $\delta_N(q,a,y)$  contiene tutte le coppie di $\delta_F(q,a,y)$. Quindi $P_N$ simula $P_F$
			\item per tutti gli stati accettanti $q\in F$ e i simboli di stack $y\in\Gamma$, compreso $x_0$, si ha che $\delta_N(q,\varepsilon,y)$ contiene $(p,\varepsilon)$, quindi ogni volta che $P_F$ accetta $P_N$ inizia scaricare lo stack senza consumare ulteriori input
			\item per tutti  i simboli di stack $y\in\Gamma$, compreso $x_0$, si ha che $\delta_N(q,\varepsilon,y)=\{(p,\varepsilon)\}$, quindi giunti allo stato $p$, ovvero quando $P_F$ ha accettato, $P_N$ elimina ogni simbolo nel suo stack fino a svuotarlo
		\end{enumerate}
		inoltre formalmente voglio dimostrare che:
		$$w\in L(P_F)\to w\in N(P_N)$$
		e quindi ho le seguenti mosse:
		$$(q_0,w,z_0)\vdash_{P_F}^*(q,\varepsilon,\alpha)\,\,q\in F,\,\,\alpha\in\Gamma^*$$
		$$(p_o,w,x_0)\vdash(q_0,w,z_0x_0)\vdash_{P_N}^*(q,\varepsilon,\alpha,x_0)\vdash_{P_N}^*(p,\varepsilon,\varepsilon)$$
	\end{theorem}
	\begin{example}
		si ha una CFG $G=(\{i,e\},\{a,b,0,1,*,+,(,)\},P,E)$
		con:
		$$P:I\to a|b|Ia|Ib|I0|I1$$
		$$\,\,E\to E+E|E*E|(E)$$
		si ha il PDA $P_G=(\{q\},\Sigma,\Sigma\cup\{i,e\},\delta,q,E)$:
		\begin{center}
			\begin{tikzpicture}[shorten >=1pt,node distance=3cm,on grid,auto]
				\node[state, initial] (q_0) {$q$};
				\path[->]
				(q_0) edge [loop above] node [align=center] {$\mbox{ }$} ();
			\end{tikzpicture}
		\end{center}
		si ha quindi:
		$$\delta(q,\varepsilon, i)=\{(q,a),(q,b),(q,Ia),(q,Ib),(q,I0),(q,I1)\}$$
		$$\delta(q,\varepsilon, E)=\{(q,I),(q,E+E),(q,E*E),(q,I(E))\}$$
		$$\delta(q,a,a)=\{(q,\varepsilon)\}$$
		$$\delta(q,b,b)=\{(q,\varepsilon)\}$$
		$$\cdots=\{(q,\varepsilon)\}$$
		quindi si ha:
		$$E\to E+E\to i+E\to a+(E)\to a+(i)\to a+(i0)\to a+(b0)$$
		$$(q, a+(b0),E)\vdash (q, a+(b0),E+E)\vdash (q, a+(b0),i+E)\vdash (q, a+(b0),a+E)$$
		$$\vdash (q, +(b0),+E)\vdash (q, (b0),E)\vdash (q, (b0),(E))\vdash (q, b0),E))$$
		$$\vdash (q, b0),i))\vdash (q, b0),i0))\vdash (q, b0),b0)\vdash (q, 0),0))$$
		$$\vdash (q, ),))\vdash (q, \varepsilon,\varepsilon)$$
	\end{example}
	questo esempio è generalizzabile ad ogni CFG


	\begin{theorem}
		sia $G=(V,T,P,S)$ una CFG:
		$$\exists\,\, PDA\,\,Q=(\{q\},T,V\cup T,\delta,q,S)\mbox{ tale che }N(Q)=L(G)$$
		$$\forall A\in V\,\,\delta(q,\varepsilon,A)=\{(q,\beta)|\,A\to B\mbox{ e' una produzione di G}\}$$
		$$\forall a\in T\,\,\delta(q,a,a)=\{(a,\varepsilon)\}$$
	\end{theorem}
	Questo dimostra che ogni CFL può essere accettato da un PDA accettante per stack vuoto. Per il teorema visto in precedenza, posso sempre costruire un altro PDA accettante per stati finale. I PDA accettano tutti e soli i linguaggi CF. Mostrare che accettano solo linguaggi di tipo 2 è complicato.\\
	Un tipo di PDA interessante, soprattutto per i parse, è il PDA deterministico, il \textbf{DPDA}.\\
	Un PDA $P=(Q,\Sigma,\Gamma,\delta,q_0,z_0,F)$ è deterministico se:
	\begin{enumerate}
		\item $|\delta(q,a,x)|\leq 1$ $\forall q\in Q,\forall a\in\Sigma\cup\{\varepsilon\},\forall x\in \Gamma$
		\item se $|\delta(q,a,x)|\neq 0$ per qualche $a\in \Sigma$ allora $|\delta(q,\varepsilon,x)|=0$
	\end{enumerate}
	\begin{example}
		abbiamo il linguaggio $L_{wcw^R}=\{wcw^R|\,w\in\{0,1\}^*\}$\\
		Gli automi a pila deterministici non riconosco tutti i CFL, ma solo una classe strettamente più piccola. Ad esempio non potrebbero riconoscere il linguaggio delle palindrome senza "il segnalibro" c. SI ha quindi:
		\begin{center}
			\begin{tikzpicture}[shorten >=1pt,node distance=3cm,on grid,auto]
				\node[state, initial] (q_0) {$q_0$};
				\node[state] (q_1) [right=of q_0] {$q_1$};
				\node[state, accepting] (q_2) [right =of q_1] {$q_2$} ;
				\path[->]
				(q_0) edge [loop above] node [align=center] {$0,z_0/0z_0$\\$1,z_0/1z_0$\\$0,0/00$\\$0,1/01$\\$1,0/10$\\$1,1/11$} ()
				edge node [align=center] {$c,z_0/0z_0$\\$c,0/0$\\$c,1/1$} (q_1)
				(q_1) edge node {$\varepsilon,z_0/z_0$} (q_2)
				edge [loop above] node [align=center] {$0,0/\varepsilon$\\$1,1/\varepsilon$} ();
			\end{tikzpicture}
		\end{center}
		quindi:
		\begin{center}
		\includegraphics{2025-01-16_15-03.png}
			% \psscalebox{1.0 1.0} % Change this value to rescale the drawing.
			% {
			% 	\begin{pspicture}(0,-2.0)(7.6,2.0)
			% 		\psframe[linecolor=black, linewidth=0.04, dimen=outer](7.6,2.0)(0.0,-2.0)
			% 		\psframe[linecolor=black, linewidth=0.04, dimen=outer](4.8,0.8)(0.8,-1.2)
			% 		\psframe[linecolor=black, linewidth=0.04, dimen=outer](2.4,0.0)(1.2,-0.8)
			% 		\rput[bl](0.4,1.6){CFL}
			% 		\rput[bl](1.46,-0.52){REG}
			% 		\rput[bl](3.68,0.28){DPDA}
			% 		\rput[bl](3.18,1.18){PDA per stato finale}
			% 		\rput[bl](3.2,1.6){PDA per stack vuoto}
			% 		\rput[bl](3.42,-0.86){$Lwcw^R$}
			% 		\rput[bl](5.96,-1.5){$Lww^R$}
			% 	\end{pspicture}
			% }
		\end{center}
	\end{example}
	si ha infatti il seguente teorema:
	\begin{theorem}
		$L\in REG\to\exists PDA\,\,P\,\,tale\,\,che\,\,L=L(P)$
	\end{theorem}
	\begin{proof}
		$$L\in REG\to\exists DFA\,\,A=(Q,\Sigma,\delta_A,q_0,F)\,\,tale\,\,che\,\, L=L(A)$$
		costruisco il DPDA $P=(Q,\Sigma,\{z_0\},\delta_p,q_0,z_0,F)$ con:
		$$\delta_p(q,a,z_0)=\{p,z_0\}\,\,\forall p,q\in Q\,\,tali\,\,che\,\,\delta_A(q,a)=0$$
		vale:
		$$(q_0,w,z_0)\stackrel{A}{\vdash_p}(p,\varepsilon,z_0)\longleftrightarrow \stackrel{\wedge}{\delta_A}(q_0,w)=p$$
	\end{proof}
	si ha inoltre il seguente teorema:
	\begin{theorem}
		$L$ è $N(P)$ per un DPDA $P$ sse $L$ è $L(P^{'})$ per un DPDA $P^{'}$ e $L$ ha le proprietà di prefisso \textbf{prefix-free}
	\end{theorem}
	definiamo così la proprietà di prefisso:
	$$\not\exists x,y\in L\,\,tali\,\,che\,\,x\neq y\,\,e\,\, x \mbox{ è prefisso di } y$$
	per esempio $L=\{0\}^0=\{\varepsilon,0,00,000,...\}$ non ha la proprietà di prefisso. Osserviamo che se la stringa vuota appartiene al linguaggio, tale stringa è prefissa di tutte le altre e quindi il linguaggio non può avere la proprietà di prefisso. Affermiamo che L è regolare, quindi è accettato da un DPDA per stati finali ma non da uno per stack vuoto. Completiamo il diagramma precedente sulle classi di
	linguaggi:
	\begin{center}
\includegraphics{2025-01-16_15-07.png}
		% \psscalebox{1.0 1.0} % Change this value to rescale the drawing.
		% {
		% 	\begin{pspicture}(0,-2.0)(7.6,2.0)
		% 		\psframe[linecolor=black, linewidth=0.04, dimen=outer](7.6,2.0)(0.0,-2.0)
		% 		\psframe[linecolor=black, linewidth=0.04, dimen=outer](5.0,0.76)(0.7,-1.56)
		% 		\psframe[linecolor=black, linewidth=0.04, dimen=outer](2.38,-0.02)(0.92,-0.82)
		% 		\rput[bl](0.4,1.6){CFL}
		% 		\rput[bl](1.066,-0.54){REG}
		% 		\rput[bl](2.76,0.36){DPDA}
		% 		\rput[bl](3.18,1.18){PDA per stato finale}
		% 		\rput[bl](3.2,1.6){PDA per stack vuoto}
		% 		\rput[bl](5.96,-1.5){$Lww^R$}
		% 		\psframe[linecolor=black, linewidth=0.04, dimen=outer](4.52,0.28)(2.14,-1.12)
		% 		\rput[bl](2.45,-0.56){$N(DPDA)$}
		% 	\end{pspicture}
		% }

	\end{center}
	SI ha che $L_{wcw^R}$ gode della proprietà di prefisso:
	$$y=wcw^R\in L\,\, Se\,\,x\neq y,\mbox{ prefisso di } y,x\not\in L$$
	tornando alle grammatiche si hanno ora due teoremi:
	\begin{theorem}
		se $L=N(P)$ per un DPDA P, allora L ha una CFG non ambigua
	\end{theorem}
	\begin{theorem}
		se $L=L(P)$ per un DPDA P, allora L ha una CFG non ambigua
	\end{theorem}
	dimostriamo il secondo:
	\begin{proof}
		$L=L(P)$ per un DPDA P, costruiamo $L^{'}=L$, quindi $L^{'}$ ha la proprietà di prefisso. Esiste quindi un DPDA $P^{'}$ tale che $L^{'}N(P)$, esiste quindi per il teorema sopra una CFG $G^{'}$ tale che $L(G^{'})=L^{'}$ che non è ambigua.\\
		Costruiamo $G$ per $L$ con le stesse produzioni di $G^{'}$ più $\$\to\varepsilon$, applicata solo all'ultimo passo.
	\end{proof}
	Vogliamo scoprire se è vero il viceversa: per ogni L che ha una CFG non ambigua è vero che L è
	accettato da un DPDA? No, mostriamo infatti un controesempio:\\
$S\to 0S0|1S1|\varepsilon$ produce $L_{ww^R}$ che non è accettato da alcun PDA

	\begin{Exercise}
		costruire un PDA per $L=\{0^n  n| n\geq 1\}$:
		CONTROLLARE LINGUAGGIO
		\begin{center}
			\begin{tikzpicture}[shorten >=1pt,node distance=3cm,on grid,auto]
				\node[state, initial] (q_0) {$q_0$};
				\node[state] (q_1) [right=of q_0] {$q_1$};
				\node[state, accepting] (q_2) [right =of q_1] {$q_2$} ;
				\path[->]
				(q_0) edge [loop above] node [align=center] {$0,0/00$\\$0,z_0/0z_0$} ()
				edge node [align=center] {$1,0/\varepsilon$} (q_1)
				(q_1) edge node {$\varepsilon,z_0/\varepsilon$} (q_2)
				edge [loop above] node [align=center] {$1,0/\varepsilon$} ();
			\end{tikzpicture}
		\end{center}
		Osserviamo che questo è un DPDA e L è accettato sia per stato finale che per stack vuoto.
		Osserviamo anche che L ha la proprietà di prefisso:
		$$y=0^n1^n\in L,x\neq y$$
		e $x$ è prefissa di y e quindi x ha un numero di 0 diverso da quelli di 1 e quindi $x\not\in L$
	\end{Exercise}

	\begin{Exercise}
		costruire un PDA per $L=\{0^n  n|\,n\geq 0\}$:
		\begin{center}
			\begin{tikzpicture}[shorten >=1pt,node distance=3cm,on grid,auto]
				\node[state, initial] (q_0) {$q_0$};
				\node[state] (q_1) [right=of q_0] {$q_1$};
				\node[state, accepting] (q_2) [right =of q_1] {$q_2$} ;
				\path[->]
				(q_0) edge [loop above] node [align=center] {$0,0/00$\\$0,z_0/0z_0$} ()
				edge node [align=center] {$1,0/\varepsilon$} (q_1)
				edge [bend left = 25] node {$\varepsilon,z_0/\varepsilon$} (q_2)
				(q_1) edge node {$\varepsilon,z_0/\varepsilon$} (q_2)
				edge [loop below] node [align=center] {$1,0/\varepsilon$} ();
			\end{tikzpicture}
		\end{center}
		Ora la stringa vuota appartiene al linguaggio. Il linguaggio non ha la proprietà del prefisso. Si può dimostrare che non esiste un DPDA per L.
	\end{Exercise}
	\begin{Exercise}
		considero il linguaggio generato da $B\to (BB)|(B)|()$.\\
		Il linguaggio ha la proprietà di prefisso perché (BB). Se fosse B -> BB allora non lo avrebbe, perché potremmo costruire le stringhe () e ()().
		\\Abbiamo quindi il DPDA per stato finale:
		\begin{center}
			\begin{tikzpicture}[shorten >=1pt,node distance=3cm,on grid,auto]
				\node[state, initial] (q_0) {$q_0$};
				\node[state] (q_1) [right=of q_0] {$q_1$};
				\node[state, accepting] (q_2) [right =of q_1] {$q_2$} ;
				\path[->]
				(q_0) edge [loop above] node [align=center] {$(,z_0/(z_0$\\$(,(/(($} ()
				edge [bend left = 25] node [align=center] {$),(/\varepsilon$} (q_1)

				(q_1) edge node {$\varepsilon,z_0/\varepsilon$} (q_2)
				edge [bend left = 25] node {$(,(/(($} (q_0)
				edge [loop above] node [align=center] {$),(/\varepsilon$} ();
			\end{tikzpicture}
		\end{center}
		volendo accettare solo per stack vuoto:
		\begin{center}
			\begin{tikzpicture}[shorten >=1pt,node distance=3cm,on grid,auto]
				\node[state, initial] (q_0) {$q_0$};
				\node[state] (q_1) [right=of q_0] {$q_1$};
				\path[->]
				(q_0) edge [loop above] node [align=center] {$(,z_0/(z_0$\\$(,(/(($} ()
				edge [bend left = 25] node [align=center] {$),(/\varepsilon$} (q_1)

				(q_1) edge [bend left = 25] node {$(,(/(($} (q_0)
				edge [loop above] node [align=center] {$),(/\varepsilon$ \\ $\varepsilon,z_0/\varepsilon$} ();
			\end{tikzpicture}
		\end{center}
	\end{Exercise}
	\begin{Exercise}
		Si ha $B\to BB|(B)|()$. La stringa vuota non appartiene al linguaggio ma L non ha la proprietà del prefisso (come abbiamo
		mostrato nell'esercizio precedente). Possiamo quindi realizzare solo un DPDA per stato finale:
		\begin{center}
			\begin{tikzpicture}[shorten >=1pt,node distance=3cm,on grid,auto]
				\node[state, initial] (q_0) {$q_0$};
				\node[state] (q_1) [right=of q_0] {$q_1$};
				\node[state] (q_2) [right =of q_1] {$q_2$};
				\node[state,accepting] (q_3) [below right =of q_1] {$q_3$};
				\path[->]
				(q_0) edge node [align=center] {$(,z_0/(z_0$} (q_1)

				(q_1) edge [bend left = 25] node {$),(/\varepsilon$} (q_2)
				edge [loop above] node [align=center] {$(,(/(($} ()
				(q_2) edge [bend left = 25] node {$(,(/(($} (q_1)
				edge [loop above] node [align=center] {$),(/\varepsilon$} ()
				edge node {$\varepsilon,z_0/z_0$} (q_3)
				(q_3) edge node {$(,z_0/(z_0$} (q_1)  ;
			\end{tikzpicture}
		\end{center}
	\end{Exercise}
	\begin{Exercise}
		Si ha $B\to BB|(B)|\varepsilon$ ho un DPDA per stato finale perché L non ha la proprietà del prefisso:
		\begin{center}
			\begin{tikzpicture}[shorten >=1pt,node distance=3cm,on grid,auto]
				\node[state, initial,accepting] (q_0) {$q_0$};
				\node[state] (q_1) [right=of q_0] {$q_1$};
				\node[state] (q_2) [right =of q_1] {$q_2$};

				\path[->]
				(q_0) edge node [align=center] {$(,z_0/(z_0$} (q_1)

				(q_1) edge [bend left = 25] node {$),(/\varepsilon$} (q_2)
				edge [loop above] node [align=center] {$(,(/(($} ()
				(q_2) edge [bend left = 25] node {$(,(/(($} (q_1)
				edge [loop above] node [align=center] {$),(/\varepsilon$} ()
				edge [bend left = 65] node {$\varepsilon,z_0/z_0$} (q_0);
			\end{tikzpicture}
		\end{center}
	\end{Exercise}
	\begin{Exercise}
		sia $L=\{q\in\{a,b\}^*| numero\,\,uguale\,\,di\,\,a\,\,e\,\,b\}$ DPDA per stato finale (L non ha la proprietà di prefisso):
		\begin{center}
			\begin{tikzpicture}[shorten >=1pt,node distance=3cm,on grid,auto]
				\node[state, initial,accepting] (q_0) {$q_0$};
				\node[state] (q_1) [right=of q_0] {$q_1$};
				\path[->]
				(q_0) edge [bend left=25] node [align=center] {$b,z_0/bz_0$\\$a,z_0/az_0$} (q_1)

				(q_1) edge [bend left = 25] node {$\varepsilon,z_0/z_0$} (q_0)
				edge [loop above] node [align=center] {$a,a/aa$\\$b,b/bb$\\$a,b/\varepsilon$ \\$b,a/\varepsilon$} ()
				;
			\end{tikzpicture}
		\end{center}
	\end{Exercise}
	\newpage
	\begin{Exercise}
		sia dato il CFL $L=\{a^ncb^n|n\geq 0\}$\\
		è generato da :
		$$G=(\{S\},\{a,b,c\},P,S)$$
		$$S\to aSb|c$$
		si ha il seguente automa non deterministico:
		\begin{center}
			\begin{tikzpicture}[shorten >=1pt,node distance=3cm,on grid,auto]
				\node[state, initial,accepting] (q_0) {$q_0$};
				\path[->]
				(q_0) edge [loop above] node [align=center] {$\varepsilon,S/aSb$\\$\varepsilon,S/c$\\$a,a/\varepsilon$\\$b,b/\varepsilon$\\$c,c/\varepsilon$} ();
			\end{tikzpicture}
		\end{center}
		con:
		$$\delta(q,\varepsilon,S)=\{(q,aSb),(q,c)\}$$
		$$\delta(q,a,a)=\{(q,\varepsilon)\}$$
		$$\delta(q,b,b)=\{(q,\varepsilon)\}$$
		$$\delta(q,c,c)=\{(q,\varepsilon)\}$$
		mostro la derivazione per n=3, aaacbbb e il comportamento dell'automa:
		$$(q,aaacbbb,S)\vdash(q,aaacbbb,aSb)\vdash(q,aacbb,Sb)\vdash(q,aacbb,aSb)$$
		$$\vdash(q,acbbb,Sbb)\vdash(q,acbbb,aSbbb)\vdash(q,cbbb,Sbbbb)\vdash(q,cbbb,cbbb)$$
		$$\vdash(q,bbb,bbb)\vdash(q,bb,bb)\vdash(q,b,b)\vdash(q,\varepsilon,\varepsilon)\to \,\, accetta$$
		si ha che vale la proprietà del prefisso e si ha il seguente DPDA:
		\begin{center}
			\begin{tikzpicture}[shorten >=1pt,node distance=3cm,on grid,auto]
				\node[state, initial,accepting] (q_0) {$q_0$};
				\node[state] (q_1) [right=of q_0] {$q_1$};
				\node[state] (q_2) [right =of q_1] {$q_2$};
				\path[->]
				(q_0) edge node [align=center] {$c,z_0/cz_0$\\$c,a/a$} (q_1)
				edge [loop above] node [align=center] {$a,z_0/az_0$\\$a,a/aa$} ()
				(q_1) edge  node {$\varepsilon,z_0/z_0$} (q_2)
				edge [loop above] node [align=center] {$b,a/\varepsilon$} ();
			\end{tikzpicture}
		\end{center}
	\end{Exercise}
	\begin{Exercise}
		realizzare il pda per $L=\{a^nb^mcd^mef^n|n,m\geq 0\}$;
		si ha quindi:
		$$a^ncef^n\,\,n>0$$
		$$b^mcd^me\,\,m>0$$
		$$a^nb^mcd^mef^n\,\,n,m> 0$$
		$$ce\,\,n=0=m$$
		\begin{center}
			\begin{tikzpicture}[shorten >=1pt,node distance=3cm,on grid,auto]
				\node[state, initial] (q_0) {$q_0$};
				\node[state] (q_1) [above right=of q_0] {$\mbox{ }$};
				\node[state] (q_2) [right =of q_0] {$\mbox{ }$};
				\node[state] (q_3) [below right =of q_0] {$\mbox{ }$};
				\node[state] (q_4) [ right =of q_1] {$\mbox{ }$};
				\node[state] (q_5) [above =of q_1] {$\mbox{ }$};
				\node[state] (q_6) [right =of q_2] {$\mbox{ }$};
				\node[state,accepting] (q_7) [right =of q_6] {$\mbox{ }$};
				\node[state] (q_8) [right =of q_5] {$\mbox{ }$};
				\node[state] (q_9) [right =of q_8] {$\mbox{ }$};
				\node[state] (q_a) [right =of q_4] {$\mbox{ }$};

				\path[->]
				(q_0) edge node [align=center] {$a,z_0/az_0$} (q_1)
				(q_0) edge node [align=center] {$b,z_0/bz_0$} (q_2)
				(q_0) edge node [left] {$c,z_0/cz_0$} (q_3)
				(q_2) edge node [align=center] {$c,b/b$} (q_6)
				edge [loop below] node [align=center] {$b,b/bb$} ()
				(q_6) edge node [align=center] {$\varepsilon,z_0/z_0$} (q_7)
				edge [loop below] node [align=center] {$d,b/\varepsilon$} ()

				(q_3) edge [bend right =65] node [align=center] {$\varepsilon,z_0/z_0$} (q_7)
				(q_1) edge node [align=center] {$b,a/ba$} (q_5)
				edge node [align=center] {$c,a/ca$} (q_4)
				edge [loop left] node [align=center] {$a,a/aa$} ()
				(q_5) edge node [align=center] {$c,b/cb$} (q_8)
				edge [loop above] node [align=center] {$b,b/bb$} ()
				(q_8) edge node [align=center] {$e,a/a$} (q_9)
				edge [loop above] node [align=center] {$d,b/\varepsilon$} ()
				(q_9) edge [bend left =95] node [align=center] {$\varepsilon,z_0/z_0$} (q_7)
				edge [loop above] node [align=center] {$f,a/\varepsilon$} ()
				(q_4) edge node [align=center] {$e,a/a$} (q_a)
				(q_a) edge node [align=center] {$\varepsilon,z_0/z_0$} (q_7)
				edge [loop right] node [align=center] {$f,a/\varepsilon$} ()
				;


			\end{tikzpicture}
		\end{center}
	\end{Exercise}
	\newpage
%%% Local Variables:
%%% mode: LaTeX
%%% TeX-master: ../libro-linguaggi
%%% End:


%\setchapterpreamble[u]{\margintoc}
	\chapter{Macchine di Turing}
	Nascono in risposta al problema di Hilbert, che si chiedeva se esiste un algoritmo per dimostrare teoremi. Non ci sono comunque funzioni non calcolabili con le macchine di Turing:
	$f:\mathbb{N}\to\{0,1\}\,\,s^{|\mathbb{N}}|>\mathbb{N}$
	Siamo interessati a studiare la calcolabilità (computabilità).
	Esiste un programma che calcola una certa funzione? Se non esiste, $f$ è indecidibile. Se sì, in quanto tempo è calcolabile (complessità computazionale), quante mosse e quante celle del nastro sono necessarie?\\
	vediamo due esempi:
	\begin{example}
		data una CFG $G=(V,T,P,S)$ stabilire se è ambigua. Questo problema è indecidibile, cioè non esiste nessun algoritmo che dia una risposta.
	\end{example}
	\begin{example}
		Problema "ciao mondo": dato un sorgente di un programma in C/Java/ , i primi undici caratteri
		stampati dal programma sono "ciao, mondo"?
		Consideriamo il seguente algoritmo, che prende in input un numero intero N e stampa "ciao mondo" sse
		$\exists x,y,z\in \mathbb{N}$ tali che
		$x^n+y^n=z^n$.
	\end{example}
	quest'ultimo problema è indecidibile:
	\begin{proof}
		procediamo per assurdo.\\Supponiamo che esista un algoritmo che risolva il problema.
		Assumiamo che P sia corretto (e quindi compilabile) e che stampi solo stringhe sulla console:
		\begin{center}
		\includegraphics{2025-01-16_15-09.png}
			%\psscalebox{1.0 1.0} % Change this value to rescale the drawing.
			%{
			%	\begin{pspicture}(0,-0.965)(4.71,0.965)
			%		\psframe[linecolor=black, linewidth=0.04, dimen=outer](3.16,0.775)(1.56,-0.825)
			%		\psline[linecolor=black, linewidth=0.04, arrowsize=0.05291667cm 2.0,arrowlength=1.4,arrowinset=0.0]{->}(0.36,-0.425)(1.56,-0.425)
			%		\psline[linecolor=black, linewidth=0.04, arrowsize=0.05291667cm 2.0,arrowlength=1.4,arrowinset=0.0]{->}(0.36,0.375)(1.56,0.375)
			%		\psline[linecolor=black, linewidth=0.04, arrowsize=0.05291667cm 2.0,arrowlength=1.4,arrowinset=0.0]{->}(3.16,-0.025)(4.36,0.775)
			%		\psline[linecolor=black, linewidth=0.04, arrowsize=0.05291667cm 2.0,arrowlength=1.4,arrowinset=0.0]{->}(3.16,-0.025)(4.36,-0.825)
			%		\rput[bl](0.0,0.235){P}
			%		\rput[bl](0.02,-0.545){I}
			%		\rput[bl](2.28,-0.085){H}
			%		\rput[bl](4.4,0.735){si}
			%		\rput[bl](4.36,-0.965){no}
			%	\end{pspicture}
			%}
		\end{center}
		quindi:
		\includegraphics{2025-01-16_15-10.png}
		suppongo di dare in input solo programmi, sse $H_1$ stampa con $P=I$:
		Allora posso pensare di dare ad $H_2$ il programma $H_2$ stesso:
		Ma questo è assurdo: perché $H_2$ dice si quando in realtà stampa ciao mondo e viceversa, quindi
		non funziona. Ma allora in conclusione H non può esistere (perché i passaggi da H a $H_1$ e da $H_1$ a
		$H_2$ sono leciti) e quindi il problema non è decidibile.
	\end{proof}
	\section{Riduzioni}
	Supponiamo $P_2$ indicibile. Considerato un nuovo problema $P_2$, vorrei stabilire se anch'esso è
	indecidibile. Vorremmo quindi fare una "riduzione" da $P_1$ a $P_2$:
		\includegraphics{2025-01-16_15-12.png}

	Supponiamo per assurdo che $P_2$ sia decidibile. Allora esiste l'algoritmo di decisione. Esistendo il
	processo di riduzione, allora l'intero rettangolo tratteggiato sarebbe un algoritmo di decisione per
$P_1$, assurdo perché $P_1$ è indecidibile per ipotesi.
	\begin{example}
		$P_2$ (Problema della chiamata): dato un programma $Q$, esso chiama il metodo $m()$?
		\\Dobbiamo quindi trovare un algoritmo di riduzione. Procediamo:
		\begin{itemize}
			\item rinominiamo tutte le istanze di $m$ in qualcosa d'altro
			\item aggiungiamo un metodo m che non verrà quindi mai chiamato
			\item modifichiamo questo programma in modo che salvi in un array i primi 11 caratteri che stampa a video
			\item modifichiamo questo programma in modo che se sfora gli 11 caratteri e controlla tali caratteri. Se sono esattamente "ciao mondo", chiama m
		\end{itemize}
		quindi anche questo problema è indecidibile
	\end{example}
	È importante che la riduzione sia fatta nel verso corretto. Se facessimo la riduzione da $P_2$ a $P_1$,
	staremmo mostrando che se $P_1$ è decidibile allora $P_2$ è decidibile. Ma sappiamo che $P_1$ è
	indecidibile, quindi non staremmo dimostrando niente.
	Nel caso contrario stiamo invece dicendo che se $P_2$ è decidibile allora $P_1$ è decidibile. Ma essendo
$P_1$ indecidibile, allora l'antecedente è falsa e quindi $P_2$ non può essere decidibile, quindi è
	indecidibile.\\
	Diamo ora la definizione formale della Macchina di Turing:
	\begin{definition}
		Si definisce MdT:
		$$M=(Q,\Sigma,\Gamma,\delta,q_0,B,F)$$
		con:
		\begin{itemize}
			\item $Q$ insieme finito non vuoto di stati
			\item $\Sigma$ insieme finito non vuoto di simboli di input
			\item $\Gamma$ insieme finito non vuoto di simboli sul nastro
			\item $\delta:Q\times\Gamma\to Q\times\Gamma\times\{L,R\}$
			\item $q_0\in Q$ stato iniziale
			\item $B\in\Gamma\backslash\Sigma$ simboli di blank
			\item $F\subseteq Q$ insieme di stati finali
		\end{itemize}
		inoltre si ha che:
		$$\Sigma\subseteq \Gamma\,\,\,e\,\,\,\Gamma\cap\Sigma\neq\emptyset$$
		quindi si ha per esempio questa rappresentazione, per l'input $x_1,\ldots ,x_n\in\Sigma^*$:
		\begin{center}
		\includegraphics{2025-01-16_15-13.png}

			%\psscalebox{1.0 1.0} % Change this value to rescale the drawing.
			%{
			%	\begin{pspicture}(0,-2.41)(9.540835,2.41)
			%		\psframe[linecolor=black, linewidth=0.04, dimen=outer](6.0,2.41)(3.6,0.01)
			%		\psframe[linecolor=black, linewidth=0.04, dimen=outer](8.8,-1.59)(0.8,-2.39)
			%		\psline[linecolor=black, linewidth=0.04](4.8,0.01)(4.8,-0.79)(4.0,-0.79)
			%		\psline[linecolor=black, linewidth=0.04](1.6,-1.59)(1.6,-2.39)
			%		\psline[linecolor=black, linewidth=0.04](2.4,-1.59)(2.4,-2.39)
			%		\psline[linecolor=black, linewidth=0.04](3.2,-1.59)(3.2,-2.39)
			%		\psline[linecolor=black, linewidth=0.04](4.0,-1.59)(4.0,-2.39)
			%		\psline[linecolor=black, linewidth=0.04](4.8,-1.59)(4.8,-2.39)
			%		\psline[linecolor=black, linewidth=0.04](5.6,-1.59)(5.6,-2.39)
			%		\psline[linecolor=black, linewidth=0.04](6.4,-1.59)(6.4,-2.39)
			%		\psline[linecolor=black, linewidth=0.04](7.2,-1.59)(7.2,-2.39)
			%		\psline[linecolor=black, linewidth=0.04](8.0,-1.59)(8.0,-2.39)
			%		\psline[linecolor=black, linewidth=0.04, linestyle=dashed, dash=0.17638889cm 0.10583334cm](0.0,-1.59)(0.8,-1.59)
			%		\psline[linecolor=black, linewidth=0.04, linestyle=dashed, dash=0.17638889cm 0.10583334cm](0.0,-2.39)(0.8,-2.39)
			%		\psline[linecolor=black, linewidth=0.04, linestyle=dashed, dash=0.17638889cm 0.10583334cm](8.8,-1.59)(9.6,-1.59)
			%		\psline[linecolor=black, linewidth=0.04, linestyle=dashed, dash=0.17638889cm 0.10583334cm](8.8,-2.39)(9.6,-2.39)
			%		\psdots[linecolor=black, dotsize=0.08](4.0,2.01)
			%		\psdots[linecolor=black, dotsize=0.08](5.6,2.01)
			%		\psdots[linecolor=black, dotsize=0.08](5.6,1.21)
			%		\psdots[linecolor=black, dotsize=0.08](5.2,0.41)
			%		\psdots[linecolor=black, dotsize=0.08](4.4,0.41)
			%		\psdots[linecolor=black, dotsize=0.08](4.8,2.01)
			%		\psdots[linecolor=black, dotsize=0.08](4.0,1.21)
			%		\psdots[linecolor=black, dotsize=0.08](4.0,0.41)
			%		\psdots[linecolor=black, dotsize=0.08](5.6,0.41)
			%		\psdots[linecolor=black, dotsize=0.08](4.8,0.41)
			%		\psdots[linecolor=black, dotsize=0.08](4.4,2.01)
			%		\psdots[linecolor=black, dotsize=0.08](5.2,2.01)
			%		\psdots[linecolor=black, dotsize=0.08](5.6,1.61)
			%		\psdots[linecolor=black, dotsize=0.08](5.6,0.81)
			%		\psdots[linecolor=black, dotsize=0.08](4.0,1.61)
			%		\psdots[linecolor=black, dotsize=0.08](4.0,0.81)
			%		\psline[linecolor=black, linewidth=0.04, arrowsize=0.05291667cm 2.0,arrowlength=1.4,arrowinset=0.0]{->}(3.6,-0.79)(3.6,-1.59)
			%		\psline[linecolor=black, linewidth=0.04, arrowsize=0.05291667cm 2.0,arrowlength=1.4,arrowinset=0.0]{->}(4.8,1.21)(4.0,2.01)
			%		\psline[linecolor=black, linewidth=0.04](4.0,-0.79)(3.6,-0.79)
			%		\rput[bl](3.44,-2.09){$x_1$}
			%		\rput[bl](6.62,-2.07){$x_n$}
			%		\rput[bl](4.3,-2.11){$x_2$}
			%		\rput[bl](5.04,-2.01){\ldots }
			%		\rput[bl](5.92,-1.99){\ldots }
			%		\rput[bl](7.42,-2.11){B}
			%		\rput[bl](8.26,-2.11){B}
			%		\rput[bl](2.74,-2.11){B}
			%		\rput[bl](1.96,-2.09){B}
			%		\rput[bl](1.06,-2.07){B}
			%	\end{pspicture}
			%}

		\end{center}
	\end{definition}
	\newpage
	Questa era una definizione deterministica, passiamo ora ad una definizione \textit{istantanea}:
	\begin{definition}
		suppongo di avere lo stato $q$, il nastro $x_1,\ldots ,x_n$ con la testina $x_i$. SI ha:
		$$ID:\,\,x_1x_2,\ldots qx_ix_{i+1}\ldots x_n$$
		e suppongo $Q\cap\Gamma=\emptyset$ senza perdere generalità e uso la simbologia dei $\vdash$ introdotta coi PDA. \\
		Se $\delta(q,x_i)=(p,y,L)$ allora $\vdash\,\, x_1x_2\ldots x_{i-2}px_{i-1}yx_{i+1}\ldots x_n$.\\
		Si hanno dei casi particolari:
		\begin{itemize}
			\item se $i=1$ si ha $qx_1x_2\ldots x_n\vdash pByx_2\ldots x_n$
			\item se $i=n\,\,y=B$ si ha $qx_1x_2\ldots qx_n\vdash x_1x_2\ldots x_{n-2}px_{n-1}$
		\end{itemize}
		a destra abbiamo invece:
		Se $\delta(q,x_i)=(p,y,R)$ allora $\vdash\,\, x_1x_2\ldots x_{i-1}Ypx_{i+1}\ldots x_n$.\\
		Si hanno dei casi particolari:
		\begin{itemize}
			\item se $i=1$ si ha $q_1x_2\ldots x_{n-1}\vdash ypB$
			\item se $i=n\,\,y=B$ si ha $qx_1x_2\ldots qx_n\vdash px_2\ldots x_n$
		\end{itemize}
	\end{definition}
	\begin{example}
		sia $L=\{0^n1^n|\,n\geq 1\}$ un CFL. vediamo una tabella per capire l'azione della macchina:
		\begin{center}
			\begin{tabular}{c|c|c|c|c|c|}
				      & 0           & 1           & x           & y           & B           \\
				\hline
				$q_0$ & $(q_1,x,R)$ & --          & --          & $(q_3,y,R)$ & --          \\
				$q_1$ & $(q_1,0,R)$ & $(q_2,y,L)$ & --          & $(q_1,y,R)$ & --          \\
				$q_2$ & $(q_2,0,L)$ & --          & $(q_0,x,R)$ & $(q_2,y,L)$ & --          \\
				$q_3$ & --          & --          & --          & $(q_2,y,L)$ & $(q_4,B,R)$ \\
				$q_4$ & --          & --          & --          & --          & --
			\end{tabular}
		\end{center}
		ovvero, per esempio:
		$$000111$$
		$$x00111$$
		$$x00y11$$
		$$xx0yy1$$
		$$xxxyyy$$
		\newpage
		ma la tabella non è comodissima, usiamo quindi i diagrammi di transizione:
		\begin{center}
			\begin{tikzpicture}[shorten >=1pt,node distance=3cm,on grid,auto]
				\node[state, initial,accepting] (q_0) {$q_0$};
				\node[state] (q_1) [right=of q_0] {$q_1$};
				\node[state] (q_2) [right =of q_1] {$q_2$};
				\node[state] (q_3) [below =of q_0] {$q_3$};
				\node[state] (q_4) [right =of q_3] {$q_4$};

				\path[->]
				(q_0) edge node [align=center] {$0/x\rightarrow$} (q_1)
				edge node [align=center] {$y/y\rightarrow$} (q_3)
				(q_1) edge node {$1/y\leftarrow$} (q_2)
				edge [loop above] node [align=center] {$y/y\rightarrow$\\$0/0\rightarrow$} ()
				(q_2) edge [bend left =40] node {$x/x\rightarrow$} (q_0)
				edge [loop above] node [align=center] {$y/y\leftarrow$\\$0/0\leftarrow$} ()
				(q_3) edge  node {$B/B\rightarrow$} (q_4)
				edge [loop below] node [align=center] {$y/y\rightarrow$} ()
				;
			\end{tikzpicture}
		\end{center}
		e per l'input 0010 si hanno i seguenti passi:
		$$q_0010\vdash xq_1010\vdash x0q_110\vdash xq_20y0\vdash  q_2x0y0$$
		$$\vdash xxq_2y0\vdash xxyq_10\vdash xxy0q_1B$$
	\end{example}
	definiamo quini il linguaggio accettato da una macchina di Turing:
	$$M=(Q,\Sigma,\Gamma,\delta,q_0,B,F)$$
	$$L(M)=\{w\in\Sigma^*|q_0w\stackrel{*}{\vdash}\alpha p\beta\,\,p\in F\,\,\alpha,\beta\in\Gamma^*\}$$
	La classe dei linguaggi accettati dalle MdT sono i ricorsivamente enumerabili (RE).
	Le MdT possono anche calcolare le funzioni:
	\begin{example}
		$$f(n,m)=m-n=max(m-n,0)=\begin{cases}
				m-n\,\,m\geq n \\
				0\,\,\,\,\,m<n
			\end{cases}$$
		quindi se l'input fosse $0^m10^n$ si avrebbe in output $0^{m-n}$
	\end{example}
	si ha che se la MdT accetta, allora si ferma e che se la MdT non accetta, allora non si può dire se si ferma oppure no. Se non si ferma: linguaggi ricorsivamente enumerabili. Se si ferma sempre, otteniamo una sottoclasse che sono i linguaggi ricorsivi. Inoltre il problema dell'arresto, \textbf{halting problem}, è indecidibile e quindi i linguaggi ricorsivamente enumerabili sono quindi semidecibili.\\
	Si possono avere delle estensioni della macchina di Turing:
	\begin{itemize}
		\item macchina non deterministica; anziché una sequenza di ID avremmo un albero. Non si aggiunge comunque potenza di calcolo, perché posso simulare una nondet con una det che esplora questo albero
		\item MdT multinastro
		      Hanno un numero finito di nastri. In una mossa guarda lo stato del controllo e il simbolo sotto ciascuna delle testine, si muove in un nuovo stato, scrive un nuovo simbolo per ogni nastro e si muove indipendentemente su ogni nastro:
		      \begin{center}
		\includegraphics{2025-01-16_15-15.png}


			    %  \psscalebox{1.0 1.0} % Change this value to rescale the drawing.
			    %  {
				%      \begin{pspicture}(0,-3.01)(9.540835,3.01)
				%	      \psframe[linecolor=black, linewidth=0.04, dimen=outer](6.0,3.01)(3.6,0.61)
				%	      \psframe[linecolor=black, linewidth=0.04, dimen=outer](8.8,-0.99)(0.8,-1.79)
				%	      \psline[linecolor=black, linewidth=0.04](4.8,0.61)(4.8,-0.19)(4.0,-0.19)
				%	      \psline[linecolor=black, linewidth=0.04](1.6,-0.99)(1.6,-1.79)
				%	      \psline[linecolor=black, linewidth=0.04](2.4,-0.99)(2.4,-1.79)
				%	      \psline[linecolor=black, linewidth=0.04](3.2,-0.99)(3.2,-1.79)
				%	      \psline[linecolor=black, linewidth=0.04](4.0,-0.99)(4.0,-1.79)
				%	      \psline[linecolor=black, linewidth=0.04](4.8,-0.99)(4.8,-1.79)
				%	      \psline[linecolor=black, linewidth=0.04](5.6,-0.99)(5.6,-1.79)
				%	      \psline[linecolor=black, linewidth=0.04](6.4,-0.99)(6.4,-1.79)
				%	      \psline[linecolor=black, linewidth=0.04](7.2,-0.99)(7.2,-1.79)
				%	      \psline[linecolor=black, linewidth=0.04](8.0,-0.99)(8.0,-1.79)
				%%%%%%%%%%	      \psline[linecolor=black, linewidth=0.04, linestyle=dashed, dash=0.17638889cm 0.10583334cm](0.0,-0.99)(0.8,-0.99)
				%	      \psline[linecolor=black, linewidth=0.04, linestyle=dashed, dash=0.17638889cm 0.10583334cm](0.0,-1.79)(0.8,-1.79)
				%	      \psline[linecolor=black, linewidth=0.04, linestyle=dashed, dash=0.17638889cm 0.10583334cm](8.8,-0.99)(9.6,-0.99)
				%	      \psline[linecolor=black, linewidth=0.04, linestyle=dashed, dash=0.17638889cm 0.10583334cm](8.8,-1.79)(9.6,-1.79)
				%	      \psdots[linecolor=black, dotsize=0.08](4.0,2.61)
				%	      \psdots[linecolor=black, dotsize=0.08](5.6,2.61)
				%	      \psdots[linecolor=black, dotsize=0.08](5.6,1.81)
				%	      \psdots[linecolor=black, dotsize=0.08](5.2,1.01)
				%	      \psdots[linecolor=black, dotsize=0.08](4.4,1.01)
				%	      \psdots[linecolor=black, dotsize=0.08](4.8,2.61)
				%	      \psdots[linecolor=black, dotsize=0.08](4.0,1.81)
				%	      \psdots[linecolor=black, dotsize=0.08](4.0,1.01)
				%	      \psdots[linecolor=black, dotsize=0.08](5.6,1.01)
				%	      \psdots[linecolor=black, dotsize=0.08](4.8,1.01)
				%	      \psdots[linecolor=black, dotsize=0.08](4.4,2.61)
				%	      \psdots[linecolor=black, dotsize=0.08](5.2,2.61)
				%	      \psdots[linecolor=black, dotsize=0.08](5.6,2.21)
				%	      \psdots[linecolor=black, dotsize=0.08](5.6,1.41)
				%	      \psdots[linecolor=black, dotsize=0.08](4.0,2.21)
				%	      \psdots[linecolor=black, dotsize=0.08](4.0,1.41)
				%	      \psline[linecolor=black, linewidth=0.04, arrowsize=0.05291667cm 2.0,arrowlength=1.4,arrowinset=0.0]{->}(3.6,-0.19)(3.6,-0.99)
				%	      \psline[linecolor=black, linewidth=0.04, arrowsize=0.05291667cm 2.0,arrowlength=1.4,arrowinset=0.0]{->}(4.8,1.81)(4.0,2.61)
				%	      \psline[linecolor=black, linewidth=0.04](4.0,-0.19)(3.6,-0.19)
				%	      \rput[bl](6.62,-1.47){$x_n$}
				%	      \rput[bl](5.04,-1.41){\ldots }
				%	      \rput[bl](5.92,-1.39){\ldots }
				%	      \rput[bl](7.42,-1.51){B}
				%	      \rput[bl](8.26,-1.51){B}
				%	      \rput[bl](2.74,-1.51){B}
				%	      \rput[bl](1.96,-1.49){B}
				%	      \rput[bl](1.06,-1.47){B}
				%	      \psframe[linecolor=black, linewidth=0.04, dimen=outer](8.76,-2.19)(0.76,-2.99)
				%	      \psline[linecolor=black, linewidth=0.04](1.56,-2.19)(1.56,-2.99)
				%	      \psline[linecolor=black, linewidth=0.04](2.36,-2.19)(2.36,-2.99)
				%	      \psline[linecolor=black, linewidth=0.04](3.16,-2.19)(3.16,-2.99)
				%	      \psline[linecolor=black, linewidth=0.04](3.96,-2.19)(3.96,-2.99)
				%	      \psline[linecolor=black, linewidth=0.04](4.76,-2.19)(4.76,-2.99)
				%	      \psline[linecolor=black, linewidth=0.04](5.56,-2.19)(5.56,-2.99)
				%	      \psline[linecolor=black, linewidth=0.04](6.36,-2.19)(6.36,-2.99)
				%	      \psline[linecolor=black, linewidth=0.04](7.16,-2.19)(7.16,-2.99)
				%	      \psline[linecolor=black, linewidth=0.04](7.96,-2.19)(7.96,-2.99)
				%	      \psline[linecolor=black, linewidth=0.04, linestyle=dashed, dash=0.17638889cm 0.10583334cm](8.76,-2.19)(9.56,-2.19)
				%	      \psline[linecolor=black, linewidth=0.04, linestyle=dashed, dash=0.17638889cm 0.10583334cm](8.76,-2.99)(9.56,-2.99)
				%	      \psline[linecolor=black, linewidth=0.04, arrowsize=0.05291667cm 2.0,arrowlength=1.4,arrowinset=0.0]{->}(4.16,0.59)(4.16,-2.21)
				%	      \rput[bl](6.58,-2.67){$x_n$}
				%	      \rput[bl](5.0,-2.61){\ldots }
				%	      \rput[bl](5.88,-2.59){\ldots }
				%	      \rput[bl](7.38,-2.71){B}
				%	      \rput[bl](8.22,-2.71){B}
				%	      \rput[bl](2.7,-2.71){B}
				%	      \rput[bl](1.92,-2.69){B}
				%	      \rput[bl](1.02,-2.67){B}
				%      \end{pspicture}
			    %  }

		      \end{center}
	\end{itemize}
	si ha il seguente teorema:
	\begin{theorem}
		ogni linguaggio accettato da una MdT multinastro è Ricorsivamente Enumerabile
	\end{theorem}
	Ora possiamo definire in maniera più precisa cosa si intendeva per simulazione di una macchina
	nondet: uso due nastri, il primo con l'input e il secondo gestito come una coda di ID da elaborare
	\begin{theorem}
		Se Mn è una NTM (Nondeterministic Turing Machine) allora esiste una DTM
		(Deterministic Turing Machine) tale che il linguaggio accettato dalla NTM è uguale a quello accettato
		dalla DTM
	\end{theorem}
	\section{Restrizioni delle macchine di Turing}
	Consideriamo una MdT con nastro semiinfinito (ovvero è infinito solo da un lato):
	\begin{center}
			\includegraphics{2025-01-16_15-16.png}

		% \psscalebox{1.0 1.0} % Change this value to rescale the drawing.
		% {
		% 	\begin{pspicture}(0,-0.43)(4.7609334,0.43)
		% 		\psline[linecolor=black, linewidth=0.04](0.020098267,0.39)(0.020098267,-0.41)
		% 		\psline[linecolor=black, linewidth=0.04](0.8200983,0.39)(0.8200983,-0.41)
		% 		\psline[linecolor=black, linewidth=0.04](1.6200982,0.39)(1.6200982,-0.41)
		% 		\psline[linecolor=black, linewidth=0.04](2.4200983,0.39)(2.4200983,-0.41)
		% 		\psline[linecolor=black, linewidth=0.04](3.2200983,0.39)(3.2200983,-0.41)
		% 		\psline[linecolor=black, linewidth=0.04, linestyle=dashed, dash=0.17638889cm 0.10583334cm](4.020098,0.39)(4.8200984,0.39)
		% 		\psline[linecolor=black, linewidth=0.04, linestyle=dashed, dash=0.17638889cm 0.10583334cm](4.020098,-0.41)(4.8200984,-0.41)
		% 		\rput[bl](1.8400983,-0.09){$x_n$}
		% 		\rput[bl](0.24009827,-0.07){$x_1$}
		% 		\rput[bl](1.1400982,-0.01){\ldots }
		% 		\rput[bl](2.6400983,-0.13){B}
		% 		\rput[bl](3.4800982,-0.13){B}
		% 		\psline[linecolor=black, linewidth=0.04](0.0,0.41)(4.080098,0.39)
		% 		\psline[linecolor=black, linewidth=0.04](0.020098267,-0.41)(4.040098,-0.41)(4.020098,0.39)
		% 	\end{pspicture}
		% }
	\end{center}
	si ha il seguente teorema:
	\begin{theorem}
		Ogni linguaggio accettato da una DTM M2 è anche accettato da una DTM M1 tale che:
		\begin{itemize}
			\item La testina di M1 non va mai a sinistra della posizione iniziale
			\item M1 non scrive mai un Blank:
			      $$B^{'}\not\in\Gamma \mbox{  e scrive }B^{'} \mbox{ al posto di } B$$
		\end{itemize}
	\end{theorem}
	Consideriamo ora una MdT tradizionale con il nastro:
			\includegraphics{2025-01-16_15-18.png}
	% \begin{center}

	% 	\psscalebox{1.0 1.0} % Change this value to rescale the drawing.
	% 	{
	% 		\begin{pspicture}(0,-0.43)(7.181278,0.43)
	% 			\psline[linecolor=black, linewidth=0.04](1.5008337,0.39)(1.5008337,-0.41)
	% 			\psline[linecolor=black, linewidth=0.04](2.3008337,0.39)(2.3008337,-0.41)
	% 			\psline[linecolor=black, linewidth=0.04](3.1008337,0.39)(3.1008337,-0.41)
	% 			\psline[linecolor=black, linewidth=0.04](3.9008338,0.39)(3.9008338,-0.41)
	% 			\psline[linecolor=black, linewidth=0.04](4.700834,0.39)(4.700834,-0.41)
	% 			\psline[linecolor=black, linewidth=0.04, linestyle=dashed, dash=0.17638889cm 0.10583334cm](5.5008335,0.39)(6.3008337,0.39)
	% 			\psline[linecolor=black, linewidth=0.04, linestyle=dashed, dash=0.17638889cm 0.10583334cm](5.5008335,-0.41)(6.3008337,-0.41)
	% 			\rput[bl](3.3208337,-0.09){$x_0$}
	% 			\rput[bl](1.6808337,-0.13){$x_{-2}$}
	% 			\rput[bl](2.5208337,-0.11){$x_{-1}$}
	% 			\rput[bl](4.120834,-0.13){$x_1$}
	% 			\rput[bl](4.9608335,-0.13){$x_2$}
	% 			\psline[linecolor=black, linewidth=0.04](1.4808338,0.41)(5.560834,0.39)
	% 			\psline[linecolor=black, linewidth=0.04](1.5008337,-0.41)(5.520834,-0.41)(5.5008335,0.39)
	% 			\psline[linecolor=black, linewidth=0.04](1.5408337,0.39)(0.74083376,0.39)(0.74083376,-0.41)(1.5408337,-0.41)
	% 			\psline[linecolor=black, linewidth=0.04, linestyle=dashed, dash=0.17638889cm 0.10583334cm](0.74083376,0.39)(-0.05916626,0.39)
	% 			\psline[linecolor=black, linewidth=0.04, linestyle=dashed, dash=0.17638889cm 0.10583334cm](0.74083376,-0.41)(-0.05916626,-0.41)
	% 			\rput[bl](0.94083375,-0.07){$B$}
	% 			\psline[linecolor=black, linewidth=0.04, linestyle=dashed, dash=0.17638889cm 0.10583334cm](6.2408338,0.39)(7.140834,0.41)
	% 			\psline[linecolor=black, linewidth=0.04, linestyle=dashed, dash=0.17638889cm 0.10583334cm](6.2808337,-0.39)(7.180834,-0.41)
	% 			\psline[linecolor=black, linewidth=0.04](5.540834,0.39)(6.3408337,0.39)(6.3408337,-0.41)(5.540834,-0.41)
	% 			\rput[bl](5.8408337,-0.13){B}
	% 		\end{pspicture}
	% 	}

	% \end{center}
	% e con un nastro semiinfinito:
	% \begin{center}

	% 	\psscalebox{1.0 1.0} % Change this value to rescale the drawing.
	% 	{
	% 		\begin{pspicture}(0,-0.82)(3.1608338,0.82)
	% 			\psline[linecolor=black, linewidth=0.04](0.02,-0.8)(2.42,-0.8)(2.42,0.8)(0.02,0.8)(0.02,-0.8)(0.82,-0.8)(0.82,0.8)(1.62,0.8)(1.62,-0.8)(2.42,-0.8)(2.42,0.0)(0.02,0.0)
	% 			\psline[linecolor=black, linewidth=0.04, linestyle=dashed, dash=0.17638889cm 0.10583334cm](2.42,0.8)(3.22,0.8)
	% 			\psline[linecolor=black, linewidth=0.04, linestyle=dashed, dash=0.17638889cm 0.10583334cm](2.42,0.0)(3.22,0.0)
	% 			\psline[linecolor=black, linewidth=0.04, linestyle=dashed, dash=0.17638889cm 0.10583334cm](2.42,-0.8)(3.22,-0.8)
	% 			\rput[bl](0.36,0.36){$x_0$}
	% 			\rput[bl](1.08,0.3){$x_1$}
	% 			\rput[bl](1.92,0.34){$x_2$}
	% 			\rput[bl](2.68,0.42){$\ldots $}
	% 			\rput[bl](0.34,-0.52){*}
	% 			\rput[bl](0.98,-0.48){$x_{-1}$}
	% 			\rput[bl](1.78,-0.5){$x_{-2}$}
	% 			\rput[bl](2.74,-0.4){\ldots }
	% 		\end{pspicture}
	% 	}

	% \end{center}
	% dove si ha l'alfabeto $\Gamma\times (\Gamma\cup\{*\})$\\
	% Macchine multi stack, che sono DPDA con un controllo a stati finiti e un numero finito di pile:
	% \begin{center}
	% 	\psscalebox{1.0 1.0} % Change this value to rescale the drawing.
	% 	{
	% 		\begin{pspicture}(0,-2.41)(5.64,2.41)
	% 			\psframe[linecolor=black, linewidth=0.04, dimen=outer](4.0,2.41)(1.6,0.01)
	% 			\psline[linecolor=black, linewidth=0.04, arrowsize=0.05291667cm 2.0,arrowlength=1.4,arrowinset=0.0]{->}(0.0,1.21)(1.6,1.21)
	% 			\psline[linecolor=black, linewidth=0.04, arrowsize=0.05291667cm 2.0,arrowlength=1.4,arrowinset=0.0]{->}(4.0,1.21)(5.6,1.21)
	% 			\psline[linecolor=black, linewidth=0.04, arrowsize=0.05291667cm 2.0,arrowlength=1.4,arrowinset=0.0]{->}(2.0,0.01)(2.0,-0.79)
	% 			\psline[linecolor=black, linewidth=0.04, arrowsize=0.05291667cm 2.0,arrowlength=1.4,arrowinset=0.0]{->}(2.0,-0.79)(2.0,0.01)
	% 			\psline[linecolor=black, linewidth=0.04, arrowsize=0.05291667cm 2.0,arrowlength=1.4,arrowinset=0.0]{->}(3.6,0.01)(3.6,-0.79)
	% 			\psline[linecolor=black, linewidth=0.04, arrowsize=0.05291667cm 2.0,arrowlength=1.4,arrowinset=0.0]{->}(3.6,-0.79)(3.6,0.01)
	% 			\psline[linecolor=black, linewidth=0.04](1.6,-0.79)(1.6,-2.39)(2.4,-2.39)(2.4,-0.79)(2.4,-1.19)(1.6,-1.19)(1.6,-1.59)(2.4,-1.59)(2.4,-1.99)(1.6,-1.99)(2.4,-1.99)
	% 			\psline[linecolor=black, linewidth=0.04](3.2,-0.79)(3.2,-2.39)(4.0,-2.39)(4.0,-0.79)(4.0,-1.19)(3.2,-1.19)(3.2,-1.59)(4.0,-1.59)(4.0,-1.99)(3.2,-1.99)(4.0,-1.99)
	% 			\rput[bl](0.42,1.27){input}
	% 			\rput[bl](4.4,1.61){accetta/}
	% 			\rput[bl](4.46,1.37){rifiuta}
	% 			\rput[bl](2.18,1.81){controlli}
	% 			\rput[bl](2.44,1.21){stati}
	% 			\rput[bl](2.42,0.67){finiti}
	% 		\end{pspicture}
	% 	}
	% \end{center}
	e si ha:
	$$\delta(q,a,x_1,x_2,\ldots ,x_k)=(p,\gamma_1,\gamma_2,\ldots ,\gamma_k)$$
	% \begin{example}
	% 	Legge l'input e lo spinge tutto sul primo stack (sfruttando un simbolo ausiliario di fine input):
	% 	\begin{center}
	% 		\psscalebox{1.0 1.0} % Change this value to rescale the drawing.
	% 		{
	% 			\begin{pspicture}(0,-2.98)(4.02,2.98)
	% 				\psframe[linecolor=black, linewidth=0.04, dimen=outer](4.0,2.98)(1.6,0.58)
	% 				\psline[linecolor=black, linewidth=0.04, arrowsize=0.05291667cm 2.0,arrowlength=1.4,arrowinset=0.0]{->}(0.0,1.78)(1.6,1.78)
	% 				\psline[linecolor=black, linewidth=0.04, arrowsize=0.05291667cm 2.0,arrowlength=1.4,arrowinset=0.0]{->}(2.0,0.58)(2.0,-0.22)
	% 				\psline[linecolor=black, linewidth=0.04, arrowsize=0.05291667cm 2.0,arrowlength=1.4,arrowinset=0.0]{->}(2.0,-0.22)(2.0,0.58)
	% 				\psline[linecolor=black, linewidth=0.04, arrowsize=0.05291667cm 2.0,arrowlength=1.4,arrowinset=0.0]{->}(3.6,0.58)(3.6,-0.22)
	% 				\psline[linecolor=black, linewidth=0.04, arrowsize=0.05291667cm 2.0,arrowlength=1.4,arrowinset=0.0]{->}(3.6,-0.22)(3.6,0.58)
	% 				\psline[linecolor=black, linewidth=0.04](1.6,-0.22)(1.6,-1.82)(2.4,-1.82)(2.4,-0.22)(2.4,-0.62)(1.6,-0.62)(1.6,-1.02)(2.4,-1.02)(2.4,-1.42)(1.6,-1.42)(2.4,-1.42)
	% 				\psline[linecolor=black, linewidth=0.04](3.2,-0.22)(3.2,-1.82)(4.0,-1.82)(4.0,-0.22)(4.0,-0.62)(3.2,-0.62)(3.2,-1.02)(4.0,-1.02)(4.0,-1.42)(3.2,-1.42)(4.0,-1.42)
	% 				\rput[bl](0.42,1.84){input}
	% 				\rput[bl](1.94,-0.44){a}
	% 				\rput[bl](1.92,-0.88){s}
	% 				\rput[bl](1.92,-1.28){a}
	% 				\rput[bl](1.92,-1.72){c}
	% 				\rput[bl](3.5,-0.44){c}
	% 				\rput[bl](3.5,-0.92){a}
	% 				\rput[bl](3.56,-1.3){s}
	% 				\rput[bl](3.56,-1.68){a}
	% 				\psline[linecolor=black, linewidth=0.04](1.6,-1.82)(1.6,-2.22)(2.4,-2.22)(2.4,-1.82)
	% 				\psline[linecolor=black, linewidth=0.04](3.2,-1.82)(3.2,-2.22)(4.0,-2.22)(4.0,-1.82)
	% 				\rput[bl](1.8,-2.14){$z_0$}
	% 				\rput[bl](3.44,-2.2){$z_0$}
	% 				\rput[bl](0.36,1.42){$casa\$$}
	% 				\psline[linecolor=black, linewidth=0.04](2.0,-2.22)(2.0,-2.62)
	% 				\psline[linecolor=black, linewidth=0.04](3.6,-2.62)(3.6,-2.22)
	% 				\psline[linecolor=black, linewidth=0.04](2.0,-2.62)(3.6,-2.62)
	% 				\rput[bl](2.26,-2.98){ribalto}
	% 				\psline[linecolor=black, linewidth=0.04, arrowsize=0.05291667cm 2.0,arrowlength=1.4,arrowinset=0.0]{->}(3.6,-2.62)(3.6,-2.22)
	% 			\end{pspicture}
	% 		}

	% 	\end{center}
	%\end{example}
	si ha un'ulteriore restrizione:
	Supponiamo che i simboli sulle pile possano essere solo $x$ o
$z_0$. Si riescono ancora a simulare le macchine di Turing
			\includegraphics{2025-01-16_15-19.png}


% \begin{center}
% 		\psscalebox{1.0 1.0} % Change this value to rescale the drawing.
% 		{
% 			\begin{pspicture}(0,-1.23)(4.04,1.23)
% 				\psline[linecolor=black, linewidth=0.04](0.02,0.79)(0.02,-0.81)(0.82,-0.81)(0.82,0.79)(0.82,0.39)(0.02,0.39)
% 				\psline[linecolor=black, linewidth=0.04](0.02,-0.01)(0.82,-0.01)
% 				\psline[linecolor=black, linewidth=0.04](0.02,-0.41)(0.82,-0.41)
% 				\psline[linecolor=black, linewidth=0.04](1.62,0.79)(1.62,-0.81)(2.42,-0.81)(2.42,0.79)
% 				\psline[linecolor=black, linewidth=0.04](1.62,-0.01)(2.42,-0.01)
% 				\psline[linecolor=black, linewidth=0.04](1.62,-0.41)(2.42,-0.41)
% 				\psline[linecolor=black, linewidth=0.04](3.22,0.79)(3.22,-0.81)(4.02,-0.81)(4.02,0.79)(4.02,0.39)(3.22,0.39)
% 				\psline[linecolor=black, linewidth=0.04](3.22,-0.01)(4.02,-0.01)
% 				\psline[linecolor=black, linewidth=0.04](3.22,-0.41)(4.02,-0.41)
% 				\rput[bl](0.32,0.91){$i_1$}
% 				\rput[bl](0.28,0.11){x}
% 				\rput[bl](0.32,-0.29){x}
% 				\rput[bl](0.26,-0.71){$z_0$}
% 				\rput[bl](1.96,-0.23){x}
% 				\rput[bl](1.94,-0.71){$z_0$}
% 				\rput[bl](3.54,0.55){x}
% 				\rput[bl](3.48,0.19){\ldots }
% 				\rput[bl](3.52,-0.33){x}
% 				\rput[bl](3.48,-0.71){$z_0$}
% 				\rput[bl](1.96,0.95){$i_2$}
% 				\rput[bl](3.6,0.89){$i_k$}
% 				\rput[bl](0.3,-1.17){$^{i_1}$}
% 				\rput[bl](1.96,-1.21){$^{i_2}$}
% 				\rput[bl](3.48,-1.23){$^{i_k}$}
% 			\end{pspicture}
% 		}
% 	\end{center}
% 	e il modello diventa:
% 	\begin{center}


% 		\psscalebox{1.0 1.0} % Change this value to rescale the drawing.
% 		{
% 			\begin{pspicture}(0,-2.0)(5.64,2.0)
% 				\psframe[linecolor=black, linewidth=0.04, dimen=outer](4.0,2.0)(1.6,-0.4)
% 				\psline[linecolor=black, linewidth=0.04, arrowsize=0.05291667cm 2.0,arrowlength=1.4,arrowinset=0.0]{->}(0.0,0.8)(1.6,0.8)
% 				\psline[linecolor=black, linewidth=0.04, arrowsize=0.05291667cm 2.0,arrowlength=1.4,arrowinset=0.0]{->}(4.0,0.8)(5.6,0.8)
% 				\psline[linecolor=black, linewidth=0.04, arrowsize=0.05291667cm 2.0,arrowlength=1.4,arrowinset=0.0]{->}(2.8,-0.4)(1.6,-1.6)
% 				\psline[linecolor=black, linewidth=0.04, arrowsize=0.05291667cm 2.0,arrowlength=1.4,arrowinset=0.0]{->}(2.8,-0.4)(2.8,-1.6)
% 				\psline[linecolor=black, linewidth=0.04, arrowsize=0.05291667cm 2.0,arrowlength=1.4,arrowinset=0.0]{->}(2.8,-0.4)(4.4,-1.6)
% 				\rput[bl](0.32,0.9){input}
% 				\rput[bl](2.02,0.68){controlli}
% 				\rput[bl](4.4,1.26){accetta/}
% 				\rput[bl](4.4,0.92){rifiuta}
% 				\rput[bl](1.3,-2.0){$c_1$}
% 				\rput[bl](2.64,-1.98){$c_2$}
% 				\rput[bl](3.56,-1.64){\ldots }
% 				\rput[bl](4.5,-1.94){$c_k$}
% 			\end{pspicture}
% 		}

% 	\end{center}
% 	dove possiamo modificare i contatori, che sono numeri naturali, a nostro piacimento "aggiungendo
% 	o togliendo X". Vediamo perché questo modello simula una MdT.
% 	Supponiamo che l'alfabeto del nastro contenga $R-1$ simboli. Allora:
% 	\begin{center}

% 		\psscalebox{1.0 1.0} % Change this value to rescale the drawing.
% 		{
% 			\begin{pspicture}(0,-0.42)(7.17,0.42)
% 				\psline[linecolor=black, linewidth=0.04](0.02,0.4)(0.02,0.4)(3.22,0.4)(3.22,-0.4)(0.02,-0.4)(0.02,0.4)
% 				\psline[linecolor=black, linewidth=0.04](0.82,0.4)(0.82,-0.4)
% 				\psline[linecolor=black, linewidth=0.04](1.62,0.4)(1.62,-0.4)
% 				\psline[linecolor=black, linewidth=0.04](2.42,0.4)(2.42,-0.4)
% 				\psline[linecolor=black, linewidth=0.04, arrowsize=0.05291667cm 2.0,arrowlength=1.4,arrowinset=0.0]{->}(3.22,0.0)(4.02,0.0)
% 				\rput[bl](0.34,-0.1){$x_1$}
% 				\rput[bl](1.18,-0.08){$x_2$}
% 				\rput[bl](1.98,0.0){\ldots }
% 				\rput[bl](2.66,-0.1){$x_k$}
% 				\rput[bl](4.06,-0.14){$x_1+x_2r+x_3r^2+\ldots $}
% 			\end{pspicture}
% 		}

% 	\end{center}
	Così come codifico il nastro posso codificare una pila. Per togliere $x_1$ in cima alla pila devo dividere
	il numero per R e prendere il resto. Per aggiungere un nuovo simbolo, moltiplico per R e aggiungo il
	nuovo simbolo. Per modificare il simbolo, basta aggiungere la differenza tra il nuovo e la cima precedente. È immediato farlo con tre contatori (usandone uno di appoggio).
	Per fare tutto ciò bastano due contatori (e quindi due pile) dove nel primo è codificato come:
	$$2^{c_1}3^{c_2}5^{c_3}$$
	Per i linguaggi si ha quindi:
	\begin{itemize}
		\item linguaggi ricorsivi $\to$ decidibili e MdT si ferma sempre
		\item linguaggi ricorsivamente enumerabili $\to$ semidecidibili (ma comunque indecidibili) e MdT si ferma se accetta ma potrebbe non fermarsi
		\item linguaggi non ricorsivamente enumerabili $\to$ indecidibili e $\not\exists$ MdT
	\end{itemize}
	quindi, insiemisticamente:
	\begin{center}
			\includegraphics{2025-01-16_15-21.png}

		% \psscalebox{1.0 1.0} % Change this value to rescale the drawing.
		% {
		% 	\begin{pspicture}(0,-2.2)(8.0,2.2)
		% 		\psellipse[linecolor=black, linewidth=0.04, dimen=outer](4.0,0.0)(4.0,2.2)
		% 		\psellipse[linecolor=black, linewidth=0.04, dimen=outer](4.2,0.0)(2.2,1.4)
		% 		\psellipse[linecolor=black, linewidth=0.04, dimen=outer](4.4,-0.2)(1.2,0.4)
		% 		\rput[bl](2.48,1.72){non ricorsivamente }
		% 		\rput[bl](2.72,0.64){ricorsivamente }
		% 		\rput[bl](3.88,-0.36){ricorsivi}
		% 		\rput[bl](2.7,0.26){enumerabili}
		% 		\rput[bl](2.54,1.48){enumerabili}
		% 	\end{pspicture}
		% }

	\end{center}
	\begin{example}
		vediamo un linguaggio non ricorsivamente enumerabile.\\
		$w\in\{o,1\}^*$ biiezione tra stringhe e numeri:
		\begin{center}
			\begin{tabular}{c c c c c c c}
				$w_1$         & $w_2$ & $w_3$ & $w_4$ & $w_5$ & $w_6$ & \ldots  \\
				1             & 2     & 3     & 4     & 5     & 6     & \ldots  \\
				$\varepsilon$ & 0     & 1     & 00    & 01    & 10    & ..
			\end{tabular}
		\end{center}
		Consideriamo però le due stringe 00101 e 0101: se le leggessi come numeri binari sono entrambi
		5, quindi abbiamo perso la biiezione.
		Forziamo quindi l'interpretazione delle stringhe mettendoci davanti un 1.
		Codifichiamo anche una MdT in binario:
		$$M=(Q,\{0,1\},\Gamma,\delta,q_1,B,\{q_2\})$$
		$$Q=\{q_1,\ldots ,q_r\}$$
		$$\Gamma=\{x_1,\ldots ,x_s\}=\{0,1,B,\ldots \}$$
		$$direzioni:\,\,L\to D_1\,\,R\to D_2$$
		$$\underbrace{\delta(q_i,x_j)=(q_k,x_1,D_m}_{0^i10^j10^k10^l10^m}$$
		$$Cod_111Cod_211Cod_3\\\ldots =Cpd(\delta)$$
	\end{example}
	\begin{example}
		sia $M=(\{q_1,q_2,q_3\},\{o,1\}.\{0,1,B\},\delta,q_1,B,q_2)$ con:
		$$\overbrace{0}^{q_1}1\overbrace{00}^{x_2}1\overbrace{000}^{q_3}1\overbrace{0}^{x_1}1\overbrace{00}^{D_2}$$
		$$\delta(q_1,1)=(q_3,0,R)$$
		$$\delta(q_3,0)=(q_1,1,R)$$
		$$\delta(q_3,1)=(q_2,0,R)$$
		$$\delta(q_3,B)=(q_3,1,L)$$
		Quindi ora possiamo dire che M è Mi, ovvero l'i-esima macchina di Turing.
		\\
		si ha però un problema: io posso permutare le delta, ottenendo numeri diversi, pur descrivendo la stessa
		macchina. Quindi la stessa macchina compare più volte nella sequenza infinita. Inoltre, data una
		stringa, non è detto che rappresenti una MdT. Inoltre:
		$$Cod((M,W))=Cod(M)111Cod(W)$$
		definisco quindi $L_d$
		$$L_d=\{w_i\in\{0,1\}^*|\,w_L\in L(M_i)\}$$
		e considero la tabella infinita:
		\begin{center}
			\begin{tabular}{c c|c c c c c c}
				       &   &          &          & j        &          &                \\
				       & 1 & 2        & 3        & 4        & 5        & 6              \\
				\hline
				       & 1 & 0        & 1        & 1        & 0        & \ldots  & \ldots       \\
				       & 2 & 1        & 1        & 0        & 0        & \ldots  & \ldots       \\
				       & 3 & 0        & 0        & 1        & 1        & \ldots  & \ldots       \\
				(Mi) i & 4 & 0        & 1        & 0        & 1        & \ldots  & \ldots       \\
				       & 5 & $\vdots$ & $\vdots$ & $\vdots$ & $\vdots$ & \ldots  & \ldots       \\
				       & 6 & $\vdots$ & $\vdots$ & $\vdots$ & $\vdots$ & \ldots  & $\ddots$
			\end{tabular}
		\end{center}
		Osserviamo la diagonale della tabella. Fanno parte di Ld solo le stringhe che hanno 0 sulla
		diagonale. Prendo in sostanza il complemento della diagonale.
		Se esistesse una MdT accettante, allora esisterebbe una riga uguale a tale complemento.
		Però tale complemento non è uguale a nessuna delle righe, perché differisce per l'i-esimo valore
		dall'i-esima riga (essendo il negato di tale valore!).
	\end{example}
	\begin{theorem}
		Se L è ricorsivo, allora il complementare di L è ricorsivo.
	\end{theorem}
	\begin{proof}
		Se L è ricorsivo, allora esiste una MdT M tale che $L = L(M)$. Costruiamo $M^{'}$ tale che
		$L(M^{'}$ = complementare di L.
	\end{proof}
	\begin{theorem}
		$$L,\overline{L}\in RE\to L,\overline{L}\in RIC$$
		se un linguaggio e il suo complementare sono ricorsivamente enumerabili allora quel linguaggio è ricorsivo
	\end{theorem}
	\begin{proof}
		$$L\in RE\to \exists M_L\to L=L(M_L)$$
		$$\overbrace{L}\in RE\to \exists M_{\overline{L}}\to \overline{L}=L(M_L)$$
	\end{proof}
	Poiché ogni stringa o appartiene a L o al complemento di L, costruiamo una nuova macchina M che
	simula $M_L$ e $M_{\overline{L}}$˛\\Prima o poi una delle due macchina deve fermarsi. Allora possiamo accettare o rifiutare.\\
	Si ha quindi la seguente tabella per i vari casi:
	\begin{center}
		\begin{tabular}{c c c c }
			accettabile & Ric              & RE               & non RE           \\
			\hline
			si          & $L,\overline{L}$ &                  &                  \\
			no          & $L$              & $\overline{L}$   &                  \\
			no          & $L$              &                  & $\overline{L}$   \\
			no          & $\overline{L}$   & $L$              &                  \\
			no          &                  & $L,\overline{L}$ &                  \\
			si          &                  & $L$              & $\overline{L}$   \\
			no          & $\overline{L}$   &                  & $L$              \\
			si          &                  & $\overline{L}$   & $L$              \\
			si          &                  &                  & $L,\overline{L}$ \\
		\end{tabular}
	\end{center}
	\section{Macchina di Turing Universale}
	Abbiamo già visto che si può codificare in binario una MdT, possiamo codificare anche la coppia
	MdT con il suo input:
	$$Cod(M)111Cod(M)\,\,[M111W]$$Possiamo quindi pensare a una MdT universale che sappia simulare qualsiasi altra MdT specificata come da codifica:
			\includegraphics{2025-01-16_15-22.png}

	% \begin{center}
	% 	\psscalebox{1.0 1.0} % Change this value to rescale the drawing.
	% 	{
	% 		\begin{pspicture}(0,-1.2)(7.44,1.2)
	% 			\psframe[linecolor=black, linewidth=0.04, dimen=outer](4.36,1.2)(1.96,-1.2)
	% 			\psline[linecolor=black, linewidth=0.04, arrowsize=0.05291667cm 2.0,arrowlength=1.4,arrowinset=0.0]{->}(0.36,0.4)(1.96,0.4)
	% 			\psline[linecolor=black, linewidth=0.04, arrowsize=0.05291667cm 2.0,arrowlength=1.4,arrowinset=0.0]{->}(0.36,-0.4)(1.96,-0.4)
	% 			\psline[linecolor=black, linewidth=0.04, arrowsize=0.05291667cm 2.0,arrowlength=1.4,arrowinset=0.0]{->}(4.36,0.0)(5.96,0.0)
	% 			\rput[bl](0.02,0.28){M}
	% 			\rput[bl](0.0,-0.5){W}
	% 			\rput[bl](2.82,-0.08){$M_U$}
	% 			\rput[bl](6.2,0.1){accetta/}
	% 			\rput[bl](6.2,-0.24){rifiuta}
	% 		\end{pspicture}
	% 	}
	% \end{center}
	$$L_U\{(M,W)|W\in L(M)\}$$
	La macchina universale corrisponde alla nostra idea di computer programmabile. Descriviamo
	questa macchina universale, che ha quattro nastri:
	\begin{itemize}
		\item \textbf{primo nastro:} $M111W$
		\item \textbf{secondo nastro:} nastro/codifica di M
		\item \textbf{terzo nastro:} stato/codifica di M
		\item \textbf{quarto nastro:} ausiliario

	\end{itemize}
	Il linguaggio universale è ricorsivamente enumerabile ma non ricorsivo (non potrebbe esserlo, visto
	che la macchina che simula potrebbe non fermarsi).
	Il diagramma delle classi dei linguaggi è quindi diventato il seguente:
			\includegraphics{2025-01-16_15-23.png}
	% \begin{center}
	% 	\psscalebox{1.0 1.0} % Change this value to rescale the drawing.
	% 	{
	% 		\begin{pspicture}(0,-1.6)(6.0,1.6)
	% 			\psframe[linecolor=black, linewidth=0.04, dimen=outer](6.0,1.6)(0.0,-1.6)
	% 			\psframe[linecolor=black, linewidth=0.04, dimen=outer](4.8,0.8)(0.8,-0.8)
	% 			\psframe[linecolor=black, linewidth=0.04, dimen=outer](3.6,0.4)(2.0,-0.4)
	% 			\rput[bl](0.8,1.2){non RE}
	% 			\rput[bl](1.2,0.4){RE}
	% 			\rput[bl](2.52,-0.14){RIC}
	% 			\rput[bl](4.0,0.4){$L_{ne}$}
	% 			\rput[bl](4.0,0.0){$L_U$}
	% 			\rput[bl](5.2,-1.2){$L_e$}
	% 			\rput[bl](4.4,-1.2){$L_d$}
	% 		\end{pspicture}
	% 	}
	% \end{center}
	Valgono ancora le riduzioni. Sapendo che $P_1$ è indecidibile (non RE o RE), posso fare una riduzione
	dalle istanze di $P_1$  alle istanze di $P_2$ mostrando che $P_2$ è indecidibile.
	Con la riduzione stiamo dicendo che $P_2$ è almeno difficile quanto $P_1$  (e non viceversa!).
	Notiamo che non è nemmeno necessario che tutte le istanze di $P_2$  siano "coperte" dal processo di
riduzione.
\begin{theorem}
	$$L_{ne}\in RE$$
\end{theorem}
\begin{proof}
	Dobbiamo fare una riduzione da $L_u$ a $L_{ne}$ per mostrare che $L_{ne}$ è ricorsivamente enumerabile.
	La riduzione è descritta da questo schema:
	\begin{center}
			\includegraphics{2025-01-16_15-25.png}
		% \psscalebox{1.0 1.0} % Change this value to rescale the drawing.
		% {
		% 	\begin{pspicture}(0,-1.89)(9.2,1.89)
		% 		\psframe[linecolor=black, linewidth=0.04, dimen=outer](6.92,1.89)(0.92,-1.31)
		% 		\psframe[linecolor=black, linewidth=0.04, dimen=outer](4.34,0.97)(3.22,-0.15)
		% 		\psline[linecolor=black, linewidth=0.04, arrowsize=0.05291667cm 2.0,arrowlength=1.4,arrowinset=0.0]{->}(0.5,0.67)(3.24,0.65)
		% 		\psline[linecolor=black, linewidth=0.04, arrowsize=0.05291667cm 2.0,arrowlength=1.4,arrowinset=0.0]{->}(2.04,0.15)(3.2,0.15)
		% 		\psline[linecolor=black, linewidth=0.04, arrowsize=0.05291667cm 2.0,arrowlength=1.4,arrowinset=0.0]{->}(4.36,0.41)(5.22,0.39)
		% 		\psline[linecolor=black, linewidth=0.04, arrowsize=0.05291667cm 2.0,arrowlength=1.4,arrowinset=0.0]{->}(6.68,0.41)(7.94,0.41)
		% 		\rput[bl](5.38,0.33){accetta}
		% 		\rput[bl](8.12,0.33){accetta}
		% 		\rput[bl](1.66,0.17){w}
		% 		\rput[bl](1.34,-0.33){(indovinato)}
		% 		\rput[bl](0.0,0.51){$M_i$}
		% 		\rput[bl](3.52,0.29){$M_u$}
		% 		\rput[bl](3.98,-1.89){M per $L_{ne}$ (nondet)}
		% 	\end{pspicture}
		% }
	\end{center}
\end{proof}
\begin{theorem}
	$$L_{ne}\not\in RIC$$
\end{theorem}
\begin{proof}
	infatti essendo $l_e$ il complementare di $l_{ne}$, per i teoremi visti in precedenza  $l_e$ non può essere ricorsivo
	né può essere ricorsivamente enumerabile. Segue che  $l_e$ è non RE
\end{proof}
\begin{example}
	considero:
	$$L_e=\{M|\,L(M)=\emptyset\}$$
	$$L_{ne}=\{M|\,L(M)=\emptyset\}$$
\end{example}
%%% Local Variables:
%%% mode: LaTeX
%%% TeX-master: ../libro-linguaggi
%%% End:



\parte{Parser}
%%% Local Variables:
%%% mode: LaTeX
%%% TeX-master: "../libro-linguaggi"
%%% End:

\section{Tabella di parsing}
\label{sec:parser-sll}

In questa sezione vedremo un metodo deterministico per il parsing top-down, in grado di costruire sempre lo stesso
albero di derivazione.
Mentre la discesa ricorsiva potrebbe, nel caso peggiore, generare un numero esponenziale di alberi parziali, un parsing
deterministico non ammette più di una possibile estensione ad ogni passo.
L'efficienza pratica della discesa ricorsiva è dovuta al fatto che ogni produzione il cui membro destro inizia con un
terminale $c$ può essere applicata solo se il carattere attuale dell'input è uguale a $c$: ciò permette di non applicare
alcune produzioni.

Il fatto di considerare solo parsing deterministici ha una conseguenza importante: non saremo più in grado di analizzare
una generica grammatica libera dal contesto, ma solo un insieme ristretto.
Ad esempio, non potremo gestire grammatiche ambigue.

Iniziamo con un caso particolarmente ristretto: tutte le produzioni hanno un membro destro che inizia con un terminale.
Se permettiamo di espandere solo produzioni il cui membro sinistro inizia con il simbolo attuale dell'input, limitiamo
fortemente l'insieme delle possibili scelte, riducendo i tempi di calcolo.

Se desideriamo aumentare ulteriormente l'efficienza del parsing, possiamo calcolare una tabella, chiamata
\keyword{tabella di parsing}, che associa ad ogni coppia variable-terminale le produzioni che hanno tale variabile come
membro destro e il cui membro sinistro inizia con il terminale.

Notiamo che la tabella di parsing è sufficiente per determinare tutto il parsing, perchè l'algoritmo semplicemente
guarderà il carattere attuale dell'input per determinare quali produzioni siano applicabili.
Possiamo restringere ulteriormente l'insieme delle grammatiche che siamo in grado di gestire, richiedendo che nessuna
cella della tabella di parsing contenga più di una produzione: ciò corrisponde alle grammatiche \keyword{SLL(1)}, anche
dette grammatiche LL(1) semplici, dove la prima L indica che il parser produrrà un'albero di derivazione sinistro
(leftmost), mentre L(1) indica che il parser potrà vedere un carattere a destra di quello attuale.

\section{Parser LL(1)}
\label{sec:parser-ll}

Procediamo gradualmente verso la rimozione dei principali vincoli, iniziamo non permettendo $\varepsilon$-produzioni.
Ciò implica che ogni variabile produce solo stringhe non vuote.
Per costruire la tabella di parsing, per ogni variabile $A$ costruiamo l'insieme $\First(A)$ che contiene i terminali
$c$ tali che $A \eprod{} t\alpha$, ovvero i terminali che iniziano una forma sentenziale che può essere prodotta a partire
da $A$.
Una volta costruiti tutti gli insiemi $\First(\cdot)$, possiamo costruire la tabella di parsing $TP$ con
l'\Cref{alg:costruzione-tabella-parsing-da-first}, dove assumiamo di conoscere $\First(\alpha)$ per ogni forma
sentenziale $\alpha$.
Ciò non è una vera limitazione, in quanto dobbiamo solo considerare il primo simbolo di $\alpha$: se si tratta di una
variabile $A$, sarà memorizzata nella tabella $\First{A}$.
Altrimenti il primo simbolo di $\alpha$ è un terminale $c$ e in questo caso possiamo imporre $\First(c) = \{c\}$.
Inoltre la tabella $TP$ deve contenere solo il membro destro della produzione, in quanto il membro sinistro è la
variabile $A$.

L'\Cref{alg:costruzione-tabella-parsing-da-first} è un esempio di algoritmo che costruisce il risultato per
  accumulazione (cioè aggiungendo qualcosa, senza mai togliere) e prosegue finchè non si raggiunge un punto fisso,
  ovvero un valore del risultato per cui non vengono più effettuate aggiunte.
La terminazione di questa procedura deriva dal fatto che ad ogni iterazione (tranne l'ultima) deve necessariamente
essere aggiunto almeno un elemento al risultato, e il risultato è un sottoinsieme di un insieme finito.
In questo caso, ogni $TP[\cdot]$ è un sottoinsieme di tutte le produzioni della grammatica.



\begin{algorithm}[tb!]
\SetKwInOut{Input}{Input}\SetKwInOut{Output}{Output}
\caption{Costruzione della tabella di parsing $TP$ in assenza di \epsprod}
\label{alg:costruzione-tabella-parsing-da-first}
\tcp{$c$ è un terminale, $A$ è una variabile}
\ForEach{$A\in V, c\in T$}{$TP[A,c]=\emptyset$}
\ForEach{produzione $A\to \alpha V$}{%
  \eIf{$\alpha$ inizia con il terminale $c$}{%
    Aggiungiamo $\alpha$ all'insieme $TP[A,c]$
  }{\ForEach{$c\in\First(\alpha)$}{%
      Aggiungiamo $\alpha$ all'insieme $TP[A,c]$
    }
  }
  }
\Return $TP$
\end{algorithm}


Rimane da descrivere come costruire gli insiemi $\First$.
Assumiamo inizialmente di non avere $\varepsilon$-produzioni: ciò implica che $\First(A)$ sarà uguale all'unione di
$\First(\alpha_{i})$ per tutte i membri destri delle produzioni che hanno $A$ come membro sinistro.
In altre parole, se $A\to \alpha_{1},\ldots, \alpha_{n}$ sono tutte le produzioni che hanno $A$ come membro sinistro,
allora $\First(A) = \bigcup_{i} \First(\alpha_{i})$.
Notiamo però che ogni $\alpha_{i}$ è una forma sentenziale, pertanto dobbiamo capire come calcolare $\First(\alpha_{i})$.
Siccome supponiamo di non avere \epsprod, per ogni forma sentenziale $\alpha_{i}$ è sufficente considerare la sua prima
variabile o il suo primo terminale.
Quindi, se $\alpha_{i}=x\beta_{i}$ con $x\in V\cup T$, allora $\First(\alpha_{i}) = \First(x)$, arrivando
all'\Cref{alg:costruzione-first-no-eps-produzioni}, dove $\First$ viene costruito incrementalmente, finchè non si arriva
ad un punto fisso.


\begin{algorithm}[tb!]
\SetKwInOut{Input}{Input}\SetKwInOut{Output}{Output}
\caption{Costruzione di $\First$ in assenza di \epsprod}
\label{alg:costruzione-first-no-eps-produzioni}
\tcp{$c$ è un terminale, $A$ è una variabile, $\alpha$ è una forma sentenziale}
$\First[\varepsilon]=\{\varepsilon\}$\;
\ForEach{$c\in T$}{$\First[c]=\{c\}$}
\ForEach{$A\in V$}{$\First[A]=\emptyset$}
$m\gets$ vero\;
\While{$m$}{%
    $m \gets $ falso\;
    \ForEach{produzione $A\to c\alpha$}{%
      \If{$\First[c] \not\subseteq \First[A]$}{
        Aggiungiamo $\First[c]$ all'insieme $\First[A]$\;
        $m\gets$ vero\;
        }
      }
    }
\Return $\First$
\end{algorithm}

Purtroppo non possiamo escludere la presenza di \epsprod.
In particolare, consideriamo cosa succede se vogliamo calcolare $\First(A\alpha)$, dove $A$ è una variabile e $\alpha$
un forma sentenziale, quanto $A\eprod{} \varepsilon$.
In questo caso $\First(A\alpha)$ include non solo $\First(A)$, ma anche $\First(\alpha)$, perchè se
$A\eprod{} \varepsilon$ allora il primo carattere generato da $A\alpha$ potrebbe essere il primo carattere generato da $\alpha$.
Non possiamo però dire che $\First(A\alpha) = \First(A) \cup \First(\alpha)$, perchè non possiamo escludere che anche
$\alpha \eprod{} \varepsilon$.
Per questo motivo introduciamo $\Follow(A)$ che è l'insieme dei terminali che seguono immediatamente la derivazione di $A$.

\begin{algorithm}[tb!]
\SetKwInOut{Input}{Input}\SetKwInOut{Output}{Output}
\caption{Costruzione di $\Follow$}
\label{alg:costruzione-follow}
\tcp{$A, B$ sono variabili, $\alpha, \beta$ sono forme sentenziali}
\ForEach{$A\in V$}{$\Follow[A]=\emptyset$}
$m\gets$ vero\;
\While{$m$}{%
    $m \gets $ falso\;
    \ForEach{produzione $A\to \alpha B\beta$}{%
      \If{$\First[\beta] \not\subseteq \Follow[B]$}{
        Aggiungiamo $\First[\beta]$ all'insieme $\Follow[B]$\;
        $m\gets$ vero\;
        }
        \If{$\beta \eprod{} \varepsilon$ e $\Follow[A] \not\subseteq \Follow[B]$}{%
          Aggiungiamo $\Follow[A]$ all'insieme $\Follow[B]$\;
          $m\gets$ vero\;
        }
      }
    }
\Return $\Follow$
\end{algorithm}

Dobbiamo adesso aggiornare la procedura per calcolare la tabella $TP$, dove ogni produzione $A\to \alpha$ viene
considerata e, se $\alpha$ non inizia con un terminale, allora aggiungiamo $\alpha$ all'insieme $TP[A,c]$ per ogni
terminale $c\in \First(\alpha)$ (esattamente come in \Cref{alg:costruzione-tabella-parsing-da-first}) e per ogni
terminale in $\Follow(A)$, se $\alpha$ può produrre la stringa vuota $\varepsilon$.
Queste operazioni vengono riassunte con i terminali nell'insieme $\First(\alpha\Follow(A))$: infatti se $\alpha$ non può
generare la stringa vuota, allora $\First(\alpha\Follow(A)) = \First(\alpha)$.



\begin{algorithm}[tb!]
\SetKwInOut{Input}{Input}\SetKwInOut{Output}{Output}
\caption{Costruzione della tabella di parsing $TP$}
\label{alg:costruzione-tabella-parsing-da-first-follow}
\tcp{$c$ è un terminale, $A$ è una variabile}
\ForEach{$A\in V, c\in T$}{$TP[A,c]=\emptyset$}
\ForEach{produzione $A\to \alpha$}{%
  \eIf{$\alpha$ inizia con il terminale $c$}{%
    Aggiungiamo $\alpha$ all'insieme $TP[A,c]$
  }{\ForEach{$c\in\First(\alpha\Follow(A))$}{%
      Aggiungiamo $\alpha$ all'insieme $TP[A,c]$
    }
  }
  }
\Return $TP$
\end{algorithm}


%%% Local Variables:
%%% mode: LaTeX
%%% TeX-master: "../libro-linguaggi"
%%% TeX-engine: luatex
%%% End:

\section{Parser LR\($k$\)}

%%% Local Variables:
%%% mode: LaTeX
%%% TeX-master: "../libro-linguaggi"
%%% TeX-engine: luatex
%%% End:

\section{Parsing  Expression Grammars}

%%% Local Variables:
%%% mode: LaTeX
%%% TeX-master: "../libro-linguaggi"
%%% TeX-engine: luatex
%%% End:



\appendix % From here onwards, chapters are numbered with letters, as is the appendix convention

\parte{Appendix}

%\input{chapters/appendix.tex}

%----------------------------------------------------------------------------------------

\backmatter % Denotes the end of the main document content
\setchapterstyle{plain} % Output plain chapters from this point onwards

%----------------------------------------------------------------------------------------
%       BIBLIOGRAPHY
%----------------------------------------------------------------------------------------

% The bibliography needs to be compiled with biber using your LaTeX editor, or on the command line with 'biber main' from the template directory

%\defbibnote{bibnote}{Here are the references in citation order.\par\bigskip} % Prepend this text to the bibliography
\printbibliography[heading=bibintoc, title=Bibliografia] % Add the bibliography heading to the ToC, set the title of the bibliography and output the bibliography note


%----------------------------------------------------------------------------------------
%       NOMENCLATURE
%----------------------------------------------------------------------------------------

% The nomenclature needs to be compiled on the command line with 'makeindex main.nlo -s nomencl.ist -o main.nls' from the template directory

%\nomenclature{$c$}{Speed of light in a vacuum inertial frame}
%\nomenclature{$h$}{Planck constant}
%
%\renewcommand{\nomname}{Notation} % Rename the default 'Nomenclature'
%\renewcommand{\nompreamble}{The next list describes several symbols that will be later used within the body of the document.} % Prepend this text to the nomenclature
%
%\printnomenclature % Output the nomenclature


%----------------------------------------------------------------------------------------
%       GLOSSARY
%----------------------------------------------------------------------------------------

% The glossary needs to be compiled on the command line with 'makeglossaries main' from the template directory
%
%\setglossarystyle{listgroup} % Set the style of the glossary (see https://en.wikibooks.org/wiki/LaTeX/Glossary for a reference)
%\printglossary[title=Special Terms, toctitle=List of Terms] % Output the glossary, 'title' is the chapter heading for the glossary, toctitle is the table of contents heading

%----------------------------------------------------------------------------------------
%       INDEX
%----------------------------------------------------------------------------------------

% The index needs to be compiled on the command line with 'makeindex main' from the template directory

%\printindex % Output the index

%\newpage
%\listoftheorems

\end{document}

%%% Local Variables:
%%% mode: LaTeX
%%% TeX-engine: luatex
%%% End:
