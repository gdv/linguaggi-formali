\documentclass[12pt,aspectratio=169,professionalfont,serif]{beamer}
\usepackage[osf,p]{newpxtext}
\usepackage{newpxmath}
\usepackage{fontawesome5}

\usepackage{fancyvrb}
\fvset{%frame=single,
commandchars=\\\{\},
%framesep=1mm,
fontfamily=helvetica,
fontsize=\normalsize
}
%\usepackage[bitstream-charter]{mathdesign}
\usepackage{listings}
\lstset{commentstyle=\color{orange}, keywordstyle=\color{yellow} }


\usetheme{Boadilla}
\useoutertheme{split}
\usecolortheme{albatross}


\setbeamertemplate{blocks}[rounded][shadow=false]
\setbeamertemplate{navigation symbols}{}
%\beamerdefaultoverlayspecification{<+->}


\setbeamercolor*{structure}{fg=green!60!black,bg=black}
\setbeamercolor*{normal text}{fg=green!60!black,bg=black}
\setbeamercolor{palette primary}{use={structure,normal text},fg=green,bg=structure.bg}
\setbeamercolor{palette secondary}{use={structure,normal text},fg=structure.fg,bg=structure.bg}
\setbeamercolor{palette tertiary}{use={structure,normal text},fg=structure.fg,bg=structure.bg}
\setbeamercolor{palette quaternary}{use={structure,normal text},fg=white,bg=green!30!black}
\setbeamercolor{palette sidebar quaternary}{use={structure,normal text},fg=white,bg=green!30!black}
\setbeamercolor{title in head/foot}{use={structure,normal text},fg=white,bg=green!30!black}

\setbeamercolor*{example text}{use={structure},fg=green!60!black}
\setbeamercolor*{block body}{use={structure},bg=structure.bg}
\setbeamercolor*{block body alerted}{use={structure},bg=structure.bg}
\setbeamercolor*{block body example}{use={structure},bg=structure.bg}
\setbeamercolor*{block title}{parent=structure,bg=structure.bg}
\setbeamercolor*{block title alerted}{use={structure,alerted text},fg=alerted text.fg,bg=structure.bg}
\setbeamertemplate{navigation symbols}{}
\setbeamertemplate{items}[square]
\setbeamercolor{item projected}{fg=white}
\setbeamercolor*{normal text}{use={structure},fg=white!90!blue,bg=structure.bg}
\setbeamercolor*{separation line}{}
\setbeamercolor*{fine separation line}{}
\setbeamercolor{alerted text}{use={structure},fg=structure.fg}

\usepackage[italian]{babel}
\usepackage[utf8]{inputenc}
\usepackage{pgf}
\usepackage{verbatim}
\usepackage{inconsolata}
\usepackage{listings}
\lstset{language=C, frame=single,
	basicstyle=\ttfamily,
	numbers=left, numberstyle=\tiny\color{gray},
	numbersep=5pt, fancyvrb=true
}

\usepackage[noend]{algorithm2e}
\usepackage{pgf}
\usepackage{tikz}
\usepackage{graphicx}
\usetikzlibrary{%
	arrows,
	arrows.meta,
	positioning,
	calc,
	backgrounds,
	chains,
	matrix,
	patterns,
	automata,
	fit,
	graphs,
	decorations,
	decorations.pathmorphing,
	decorations.pathreplacing,
	decorations.markings,
}

\usepgflibrary{shapes,shapes.geometric}


\usepackage{url}


\usepackage{xmpmulti}
\usepackage{transparent}
\graphicspath{{img/}}

\newcommand{\simplebox}[1]{
      \begin{tikzpicture}
              \node[text centered, rectangle, draw=black, thick, text width=\linewidth,align=center, rounded corners, minimum height=1em] (C) {#1};
      \end{tikzpicture}
}
%
% In questo file si mettono i comandi utili solo per questo insegnamento.
%
\author{Gianluca Della Vedova}
\title{Linguaggi e Computabilità}
\institute{Univ. Milano--Bicocca\\
	\texttt{https://www.unimib.it/gianluca-della-vedova}}
\date{\today}

