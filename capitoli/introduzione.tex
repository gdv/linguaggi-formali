\chapter{Introduzione}

Questo testo si prefigge di introdurre i linguaggi formali e descrivere la
rilevanza che questi hanno nel disegno di un compilatore.

In sintesi, un compilatore è un programma che riceve il codice sorgente di un
programma e produce un programma equivalente, detto codice oggetto, in un altro
linguaggio di programmazione: spesso in assembler.

L'anatomia di un compilatore individua tre componenti
principali~\sidecite{engineering-a-compiler}:
\begin{enumerate}
\item
      Frontend: trasforma il codice sorgente in una rappresentazione intermedia
      (IR) indipendente dal linguaggio di programmazione.
\item
      Optimizer: manipola la rappresentazione intermedia per migliorarne le
      caratteristiche (rendere più veloce il programma, usare meno memoria,
      ecc.)
\item
      Backend: riceve la rappresentazione intermedia ed emette il corrispondente
      codice oggetto.
\end{enumerate}

Vedremo che i linguaggi formali sono il fondamento di ogni frontend.

%%% Local Variables:
%%% mode: LaTeX
%%% TeX-master: "../libro-linguaggi"
%%% End:
