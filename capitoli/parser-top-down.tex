\chapter{Parser}\label{cha:parser}

Un parser analizza il risultato dell'analizzatore lessicale e costruire l'albero di derivazione che ha prodotto il testo.
In realtà nulla impedisce di avere solo un parser che agisca sul testo, evitando di avere un analizzatore lessicale
separato e, sempre più spesso, è quello che viene fatto in pratica, soprattutto per linguaggi relativamente semplici.
La motivazione principale per distinguere un passo di analisi lessicale è l'efficienza: un lexer sarà basato su
espressioni regolari (quindi grammatiche di tipo 3), mentre un parser richiede un livello di sofisticazione maggiore.
Sebbene esistano parser per linguaggi liberi dal contesto, normalmente si preferisce restringere l'attenzione a classi
di linguaggi intermedi fra liberi dal contesto e regolari, in quanto i linguaggi di programmazione difficilmente
richiedono tutta la flessibilità introdotta dalle CFG.

\section{Parser Top-down}\label{sec:parser-top-down}

L'idea intuitiva dei \textbf{parser top-down} è estremamente semplice: costruire l'albero di derivazione a partire da un albero $D$ formato dalla
sola radice ed estendere l'albero con l'espansione di una produzione alla volta, finchè non si ottiene esattamente il
testo $T$.

Questa tipologia di parser si contrappone a quella dei \textbf{parser bottom-up} che invece partono dall'analisi del
testo e costruiscono l'albero di derivazione partendo dalle foglie, finchè non viene ricostruita la radice dell'albero.
Questa tipologia era dominante in passato, quando i computer erano decisamente meno potenti e procedure efficienti, ma
complesse, erano necessarie per effettuare il parsing in tempi ragionevoli.
Negli ultimi anni si vede invece una tendenza verso i parser top-down, in quanto gli algoritmi utilizzati sono semplici,
sia da descrivere che da implementare, sebbene non abbiano le stesse garanzie di efficienza di alcuni parser bottom-up.



%%% Local Variables:
%%% mode: LaTeX
%%% TeX-master: "../libro-linguaggi"
%%% TeX-engine: luatex
%%% End:
