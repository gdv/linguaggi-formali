\section{Parser Top-down}\label{sec:parser-top-down}

L'idea intuitiva dei \textbf{parser top-down} è estremamente semplice: costruire l'albero di derivazione a partire da un albero $D$ formato dalla
sola radice ed estendere l'albero con l'espansione di una produzione alla volta, finchè non si ottiene esattamente il
testo $T$.

Questa tipologia di parser si contrappone a quella dei \textbf{parser bottom-up} che invece partono dall'analisi del
testo e costruiscono l'albero di derivazione partendo dalle foglie, finchè non viene ricostruita la radice dell'albero.
Questa tipologia era dominante in passato, quando i computer erano decisamente meno potenti e procedure efficienti, ma
complesse, erano necessarie per effettuare il parsing in tempi ragionevoli.
Negli ultimi anni si vede invece una tendenza verso i parser top-down, in quanto gli algoritmi utilizzati sono semplici,
sia da descrivere che da implementare, sebbene non abbiano le stesse garanzie di efficienza di alcuni parser bottom-up.



%%% Local Variables:
%%% mode: LaTeX
%%% TeX-master: "../libro-linguaggi"
%%% TeX-engine: luatex
%%% End:
