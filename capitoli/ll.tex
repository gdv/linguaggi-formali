\section{Tabella di parsing}
\label{sec:parser-sll}

In questa sezione vedremo un metodo deterministico per il parsing top-down, in grado di costruire sempre lo stesso
albero di derivazione.
Mentre la discesa ricorsiva potrebbe, nel caso peggiore, generare un numero esponenziale di alberi parziali, un parsing
deterministico non ammette più di una possibile estensione ad ogni passo.
L'efficienza pratica della discesa ricorsiva è dovuta al fatto che ogni produzione il cui membro destro inizia con un
terminale $c$ può essere applicata solo se il carattere attuale dell'input è uguale a $c$: ciò permette di non applicare
alcune produzioni.

Il fatto di considerare solo parsing deterministici ha una conseguenza importante: non saremo più in grado di analizzare
una generica grammatica libera dal contesto, ma solo un insieme ristretto.
Ad esempio, non potremo gestire grammatiche ambigue.

Iniziamo con un caso particolarmente ristretto: tutte le produzioni hanno un membro destro che inizia con un terminale.
Se permettiamo di espandere solo produzioni il cui membro sinistro inizia con il simbolo attuale dell'input, limitiamo
fortemente l'insieme delle possibili scelte, riducendo i tempi di calcolo.

Se desideriamo aumentare ulteriormente l'efficienza del parsing, possiamo calcolare una tabella, chiamata
\keyword{tabella di parsing}, che associa ad ogni coppia variable-terminale le produzioni che hanno tale variabile come
membro destro e il cui membro sinistro inizia con il terminale.

Notiamo che la tabella di parsing è sufficiente per determinare tutto il parsing, perchè l'algoritmo semplicemente
guarderà il carattere attuale dell'input per determinare quali produzioni siano applicabili.
Possiamo restringere ulteriormente l'insieme delle grammatiche che siamo in grado di gestire, richiedendo che nessuna
cella della tabella di parsing contenga più di una produzione: ciò corrisponde alle grammatiche \keyword{SLL(1)}, anche
dette grammatiche LL(1) semplici, dove la prima L indica che il parser produrrà un'albero di derivazione sinistro
(leftmost), mentre L(1) indica che il parser potrà vedere un carattere a destra di quello attuale.

\section{Parser LL(1)}
\label{sec:parser-ll}

Procediamo gradualmente verso la rimozione dei principali vincoli, iniziamo non permettendo $\varepsilon$-produzioni.
Ciò implica che ogni variabile produce solo stringhe non vuote.



%%% Local Variables:
%%% mode: LaTeX
%%% TeX-master: "../libro-linguaggi"
%%% TeX-engine: luatex
%%% End:
